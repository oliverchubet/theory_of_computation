\section{Deduction}\label{S-logic-5}


\newcommand{\argument}[2]{
  \begin{tabular}[t]{@{\ }l@{\ }}
   #1\\
   \hline
   $\therefore$\ \ #2\\
  \end{tabular}
}


Logic can be applied to draw conclusions from a set of premises.
A premise\index{premise} is just a proposition that is known to be true or that
has been accepted to be true for the sake of argument, and a conclusion\index{conclusion}
is a proposition that can be deduced\index{deduction} logically from the premises.
The idea is that if you believe that the premises are true,
then logic forces you to accept that the conclusion is true.
An ``argument''\index{argument} is a claim that a certain conclusion follows from
a given set of premises.  Here is an argument laid out in
a traditional format:
\begin{center}
\argument{If today is Tuesday, then this is Belgium\\Today is Tuesday}{This is Belgium}
\end{center}
The premises of the argument are shown above the line, and the conclusion
below.  The symbol $\therefore$ is read ``therefore.''  The claim is that
the conclusion, ``This is Belgium,'' can be deduced logically from the two
premises, ``If today is Tuesday, then this is Belgium'' and ``Today is Tuesday.''
In fact, this claim is true.  Logic forces you to accept this argument.
Why is that?

Let $p$ stand for the proposition ``Today is Tuesday,'' and let $q$ stand for the
proposition ``This is Belgium.''  Then the above argument has the form
\begin{center}
\argument{$p\IMP q$\\$p$}{$q$}
\end{center}
Now, for \emph{any} propositions $p$ and $q$---not just the ones in this particular
argument---if $p\IMP q$ is true and $p$ is true, then $q$ must also be true.
This is easy to check in a truth table:
\begin{center}
  \begin{tabular}{|c|c|c|}
     \hline
     \strut $p$&    $q$&    $p\IMP q$\\
     \hline
     false&  false&  true\\
     false&  true&   true\\
     true&   false&  false\\
     true&   true&   true\\
     \hline
  \end{tabular}
\end{center}
The only case where both $p\IMP q$ and $p$ are true is on
the last line of the table, and in this case, $q$ is also true.
If you believe
$p\IMP q$ and $p$, you have no logical choice but to believe $q$.
This applies no matter what $p$ and $q$ represent.  For example,
if you believe ``If Jill is breathing, then Jill pays taxes,'' 
and you believe that ``Jill is breathing,'' logic forces you to believe that
``Jill pays taxes.''  Note that we can't say for sure that the
conclusion is true, only that \emph{if} the premises are true,
\emph{then} the conclusion must be true.

This fact can be rephrased by saying that $\big((p\IMP q)\AND p\big)\IMP q$
is a tautology.  More generally, for any compound propositions $\mathscr{P}$
and $\mathscr{Q}$, saying ``$\mathscr{P}\IMP \mathscr{Q}$ is
a tautology'' is the same as saying that ``in all cases where $\mathscr{P}$
is true, $\mathscr{Q}$ is also true''.\footnote{Here, ``in all cases'' means
for all combinations of truth values of the propositional variables
in $\mathscr{P}$ and $\mathscr{Q}$.  Saying $\mathscr{P}\IMP \mathscr{Q}$ is
a tautology means it is true in all cases.  But by definition of 
$\IMP$, it is automatically true in cases where $\mathscr{P}$ is
false.  In cases where $\mathscr{P}$ is true, $\mathscr{P}\IMP \mathscr{Q}$
will be true if and only if $\mathscr{Q}$ is true.}
We will use the notation $\mathscr{P}\LOGIMP\mathscr{Q}$ to
mean that $\mathscr{P}\IMP \mathscr{Q}$ is a tautology.
Think of $\mathscr{P}$ as being the premise of an argument or
the conjunction of several premises.  To say $\mathscr{P}\LOGIMP\mathscr{Q}$
is to say that $\mathscr{Q}$ follows logically from~$\mathscr{P}$.
We will use the same notation in both propositional logic and
predicate logic.  (Note that the relation of $\LOGIMP$ to $\IMP$ is
the same as the relation of $\equiv$ to $\IFF$.)


\begin{definition}
Let $\mathscr{P}$ and $\mathscr{Q}$ be any formulas in either
propositional logic or predicate logic.  The notation
$\mathscr{P}\LOGIMP\mathscr{Q}$ is used to mean that
$\mathscr{P}\IMP\mathscr{Q}$ is a tautology.  That is,
in all cases where $\mathscr{P}$ is true, $\mathscr{Q}$ is
also true.  We then say that $\mathscr{Q}$ can be
\nw[deduction]{logically deduced} from $\mathscr{P}$ or that
$\mathscr{P}$ \nw{logically implies} $\mathscr{Q}$.
\end{definition}

An argument in which the conclusion follows logically from the
premises is said to be a \nw{valid argument}.  To test whether
an argument is valid, you have to replace the particular propositions
or predicates that it contains with variables, and then test
whether the conjunction of the premises logically implies the
conclusion.  We have seen that any argument of the form
\begin{center}
\argument{$p\IMP q$\\$p$}{$q$}
\end{center}
is valid, since $\big((p\IMP q)\AND p\big)\IMP q$ is a tautology.
This rule of deduction is called \nw{modus ponens}.  It plays a central
role in logic.  Another, closely related rule is \nw{modus tollens},
which applies to arguments of the form
\begin{center}
\argument{$p\IMP q$\\$\NOT q$}{$\NOT p$}
\end{center}
To verify that this is a valid argument, just check that
$\big((p\IMP q)\AND \NOT q\big)\LOGIMP \NOT p$, that is, that
$\big((p\IMP q)\AND \NOT q\big)\IMP \NOT p$ is a tautology.
As an example, the following argument has the form of \textit{modus tollens}
and is therefore a valid argument:
\begin{center}
\argument{If Keanu Reeves is a good actor, then I'm the king of France\\
I am not the king of France}{Keanu Reeves in not a good actor}
\end{center}
You should note carefully that the validity of this argument has nothing
to do with whether or not Keanu Reeves\index{Reeves, Keanu} can act well.  The argument forces
you to accept the conclusion \emph{only if} you accept the premises.
You can logically believe that the conclusion is false, as long as
you believe that at least one of the premises is false.

Another named rule of deduction is the \nw[syllogism]{Law of Syllogism}, which has the form
\begin{center}
\argument{$p\IMP q$\\$q\IMP r$}{$p\IMP r$}
\end{center}
For example:
\begin{center}
\argument{If you study hard, you do well in school\\
If you do well in school, you get a good job}{If you study hard,
you get a good job}
\end{center}

There are many other rules.  Here are a few that might prove useful.
Some of them might look trivial, but don't underestimate the power
of a simple rule when it is combined with other rules.
\begin{center}
\mbox{
\argument{$p\OR q$\\$\NOT p$}{$q$}\qquad
\argument{$p$\\$q$}{$p\AND q$}\qquad
\argument{$p\AND q$}{$p$}\qquad
\argument{$p$}{$p\OR q$}
}
\end{center}


Logical deduction is related to logical equivalence.\index{logical equivalence!and logical deduction}
We defined $\mathscr{P}$ and $\mathscr{Q}$ to be
logically equivalent if $\mathscr{P}\IFF\mathscr{Q}$ is
a tautology.  Since $\mathscr{P}\IFF\mathscr{Q}$ is equivalent
to $(\mathscr{P}\IMP\mathscr{Q})\AND(\mathscr{Q}\IMP\mathscr{P})$,
we see that  $\mathscr{P}\equiv \mathscr{Q}$ if and only if both
$\mathscr{Q}\LOGIMP\mathscr{P}$ and $\mathscr{P}\LOGIMP\mathscr{Q}$.
Thus, we can show that two statements are logically equivalent if
we can show that each of them can be logically deduced from the
other.  Also, we get a lot of rules about logical deduction for 
free---two rules of deduction for each logical equivalence we know.  For
example, since $\NOT(p\AND q)\equiv (\NOT p\OR \NOT q)$,
we get that $\NOT(p\AND q)\LOGIMP (\NOT p\OR \NOT q)$.
For example, if we know  ``It is not both sunny and warm,'' 
then we can logically deduce ``Either it's not sunny or it's not warm.''
(And vice versa.)


\medskip


In general, arguments are more complicated that those we've considered
so far.  Here, for example, is an argument that has five premises:
\begin{center}
\argument{
  $(p\AND r)\IMP s$\\
  $q\IMP p$\\
  $t\IMP r$\\
  $q$\\
  $t$
}{$s$}
\end{center}
Is this argument valid?  Of course, you could use a truth table
to check whether the conjunction of the premises logically implies
the conclusion.  But with five propositional variables, the table
would have 32 lines, and the size of the table grows quickly when
more propositional variables are used.  So, in general, truth
tables are not practical.  

Fortunately, there is another way to proceed, based on the fact that
it is possible to chain several logical deductions together.
That is, if $\mathscr{P}\LOGIMP\mathscr{Q}$ and
$\mathscr{Q}\LOGIMP\mathscr{R}$, it follows that
$\mathscr{P}\LOGIMP\mathscr{R}$.  This means we can demonstrate the
validity of an argument by deducing the conclusion from the
premises in a sequence of steps.  These steps can be presented
in the form of a proof:

\begin{definition}
A \nw[proof]{formal proof} that an argument is valid consists of a
sequence of propositions such that the last proposition in the
sequence is the conclusion of the argument, and every proposition
in the sequence is either a premise of the argument or follows
by logical deduction from propositions that precede it in the list.
\end{definition}

The existence of such a proof shows that the conclusion follows
logically from the premises, and therefore that the argument is 
valid.\index{valid argument}  Here is a formal proof that the argument given above is valid.
The propositions in the proof are numbered, and each proposition
has a justification.
\begin{center}
  \begin{tabular}{r@{\ \ }l@{\qquad}l}
     1.&$q\IMP p$&    premise\\
     2.&$q$&          premise\\
     3.&$p$&          from 1 and 2 (\textit{modus ponens})\\
     4.&$t\IMP r$&    premise\\
     5.&$t$&          premise\\
     6.&$r$&          from 4 and 5 (\textit{modus ponens})\\
     7.&$p\AND r$&    from 3 and 6\\
     8.&$(p\AND r)\IMP s$& premise\\
     9.&$s$&          from 7 and 8 (\textit{modus ponens})\\
  \end{tabular}
\end{center}
Once a formal proof has been constructed, it is convincing.  Unfortunately,
it's not necessarily easy to come up with the proof.  Usually, the best
method is a combination of working forward (``Here's what I know, what
can I deduce from that?'') and working backwards (``Here's what I
need to prove, what other things would imply that?'').  For this proof,
I might have thought:  I want to prove $s$.  I know that
$p\AND r$ implies $s$, so if I can prove $p\AND r$, I'm OK.
But to prove $p\AND r$, it'll be enough to prove $p$ and $r$ 
separately\dots.

Of course, not every argument is valid, so the question also
arises, how can we show that an argument is invalid?\index{invalid argument}  Let's
assume that the argument has been put into general form, with
all the specific propositions replaced by propositional variables.
The argument is valid if in all cases where all the premises are
true, the conclusion is also true.  The argument is invalid if
there is even one case where all the premises are true and the
conclusion is false.  We can prove that an argument is invalid
by finding an assignment of truth values to the propositional variables
which makes all the premises true but makes the conclusion false.
For example, consider an argument of the form:
\begin{center}
\argument{$p\IMP q$\\$q\IMP (p\AND r)$\\$r$}{$p$}
\end{center}
In the case where $p$ is false, $q$ is false, and $r$ is true,
the three premises of this argument are all true, but the conclusion
is false.  This shows that the argument is invalid.

To apply all this to arguments stated in English, we have to
introduce propositional variables to represent all the propositions
in the argument.  For example, consider:
\begin{quote}
John will be at the party if Mary is there and Bill is not there.
Mary will be at the party if it's on Friday or Saturday.
If Bill is at the party, Tom will be there.  Tom won't be at
the party if it's on Friday.  The party is on Friday.
Therefore, John will be at the party.
\end{quote}
Let $j$ stand for ``John will be at the party,'' $m$ for
``Mary will be there,'' $b$ for ``Bill will be there,''
$t$ for ``Tom will be there,'' $f$ for ``The party is on Friday,''
and $s$ for ``The party is on Saturday.''  Then this argument has
the form
\begin{center}
\argument{$(m\AND \NOT b)\IMP j$\\
  $(f\OR s)\IMP m$\\
  $b\IMP t$\\
  $f\IMP \NOT t$\\
  $f$}{$j$}
\end{center}
This is a valid argument, as the following proof shows:
\begin{center}
  \begin{tabular}{r@{\ \ }l@{\qquad}l}
     1.&$f\IMP\NOT t$&premise\\
     2.&$f$&premise\\
     3.&$\NOT t$&from 1 and 2 (\textit{modus ponens})\\
     4.&$b\IMP t$&premise\\
     5.&$\NOT b$&from 4 and 3 (\textit{modus tollens})\\
     6.&$f\OR s$&from 2\\
     7.&$(f\OR s)\IMP m$&premise\\
     8.&$m$&from 6 and 7 (\textit{modus ponens})\\
     9.&$m\AND\NOT b$&from 8 and 5\\
     10.&$(m\AND\NOT b)\IMP j$&premise\\
     11.&$j$&from 10 and 9 (\textit{modus ponens})\\
  \end{tabular}
\end{center}

\medskip

So far in this section, we have been working mostly with propositional
logic.  But the definitions of valid argument and logical deduction
apply to predicate logic as well.  One of the most basic rules of
deduction in predicate logic says that $(\forall xP(x))\LOGIMP P(a)$
for any element $a$ in the domain of the predicate~$P$.
That is, if a predicate is true of all elements, then it is true of
any given particular element.  This rule can be combined with 
rules of deduction for propositional logic to give the following
valid arguments
\begin{center}
\mbox{
\argument{$\forall x(P(x)\IMP Q(x))$\\$P(a)$}{$Q(a)$}\qquad\qquad
\argument{$\forall x(P(x)\IMP Q(x))$\\$\NOT Q(a)$}{$\NOT P(a)$}
}
\end{center}
These valid arguments go by the names of \textit{modus ponens}\index{modus ponens} and
\textit{modus tollens}\index{modus tollens} for predicate logic.
Note that from the premise $\forall x(P(x)\IMP Q(x))$ we can deduce
$P(a)\IMP Q(a)$.  From this and from the premise that $P(a)$, we
can deduce $Q(a)$ by \textit{modus ponens}.  So the first argument
above is valid.  The second argument is similar, using 
\textit{modus tollens}.

The most famous logical deduction of them all is an application
of \textit{modus ponens} for predicate logic:
\begin{center}
\argument{All humans are mortal\\Socrates is human}{Socrates is mortal}
\end{center}
This has the form of \textit{modus ponens} with $P(x)$ standing
for ``$x$~is human,'' $Q(x)$ standing for ``$x$~is mortal,'' and
$a$ standing for the noted element, Socrates.

There is a lot more to say about logical deduction and
proof in predicate logic, and we'll spend the rest of this chapter
on the subject.

\begin{exercises}

\problem Verify the validity of \textit{modus tollens} and the Law of
Syllogism.

\problem Each of the following is a valid rule of deduction.
For each one, give an example of a valid argument in English that
uses that rule.
\begin{center}
\mbox{\qquad
\argument{$p\OR q$\\$\NOT p$}{$q$}\qquad
\argument{$p$\\$q$}{$p\AND q$}\qquad
\argument{$p\AND q$}{$p$}\qquad
\argument{$p$}{$p\OR q$}
}
\end{center}

\problem There are two notorious invalid arguments that look
deceptively like \textit{modus ponens} and \textit{modus tollens}:
\begin{center}
\mbox{\qquad
\argument{$p\IMP q$\\$q$}{$p$}\qquad
\argument{$p\IMP q$\\$\NOT p$}{$\NOT q$}\qquad
}
\end{center}
Show that each of these arguments is invalid.  Give an English
example that uses each of these arguments.

\problem Decide whether each of the following arguments is valid.
If it is valid, give a formal proof.  If it is invalid, show that
it is invalid by finding an appropriate assignment of truth values
to propositional variables.
\smallskip
\tparts{
   \argument{$p\IMP q$\\$q\IMP s$\\$s$}{$p$}&
   \argument{$p\AND q$\\$q\IMP (r\OR s)$\\$\NOT r$}{$s$}&
   \argument{$p\OR q$\\$q\IMP (r\AND s)$\\$\NOT p$}{$s$}\cr\noalign{\smallskip}
   \argument{$(\NOT p)\IMP t$\\$q\IMP s$\\$r\IMP q$\\$\NOT(q\OR t)$}{$p$}&
   \argument{$p$\\$s\IMP r$\\$q\OR r$\\$q\IMP\NOT p$}{$\NOT s$}&
   \argument{$q\IMP t$\\$p\IMP(t\IMP s)$\\$p$}{$q\IMP s$}
}


\problem For each of the following English arguments, express the
argument in terms of propositional logic and determine whether the
argument is valid or invalid.
\ppart If it is Sunday, it rains or snows.  Today, it is Sunday
and it's not raining.  Therefore, it must be snowing.
\ppart If there are anchovies on the pizza, Jack won't eat it.
If Jack doesn't eat pizza, he gets angry.  Jack is angry.
Therefore, there were anchovies on the pizza.
\ppart At 8:00, Jane studies in the library or works at home.
It's 8:00 and Jane is not studying in the library.  So she must
be working at home.



\end{exercises}




