\chapter{Turing Machines and Computability}\label{C-turing}

\renewcommand{\b}{{\tt\#}}
\newcommand{\at}{{\tt\char`\@}}

\startchapter{We saw hints} at the end of the previous chapter that
``computation'' is a more general concept than we might have thought.
General grammars, which at first encounter don't seem to have much
to do with algorithms or computing, turn out to be able to do things
that are similar to the tasks carried out by computer programs.
In this chapter, we will see that general grammars are precisely
equivalent to computer programs in terms of their computational
power, and that both are equivalent to a particularly simple model
of computation known as a \nw{Turing machine}.  We shall also see
that there are limits to what can be done by computing.















%\endinput

