\section{Application: Recursion and Induction}

In computer programming, there is a technique called \nw{recursion}
that is closely related to induction.  In a computer program, a
\nw{subroutine} is a named sequence of instructions for performing
a certain task.  When that task needs to be performed in a program,
the subroutine can be \nw[calling a subroutine]{called} by name.
A typical way to organize a program is to break down a large task
into smaller, simpler subtasks by calling subroutines to perform
each of the subtasks.  A subroutine can perform its task by
calling other subroutines to perform subtasks of the overall task.
A subroutine can also call itself.  That is, in the process of
performing some large task, a subroutine can call itself to perform
a subtask.  This is known as recursion, and a subroutine that does
this is said to be a \nw{recursive subroutine}.  Recursion is
appropriate when a large task can be broken into subtasks where
some or all of the subtasks are smaller, simpler versions of the
main task.

Like induction, recursion is often considered to be a ``hard''
topic by students.  Professors, on the other hand, often say
that they can't see what all the fuss is about, since induction
and recursion are elegant methods which ``obviously'' work.
In fairness, students have a point, since induction and recursion
both manage to pull infinite rabbits out of very finite hats.
But the magic is indeed elegant, and learning the trick is
very worthwhile.

\medbreak

A simple example of a recursive subroutine is a function that
computes $n!$ for a non-negative integer $n$.  $n!$, which is read ``$n$ factorial,''
is defined as follows:
\begin{align*}
0! &= 1\\
n! &= \prod_{i=1}^n\,i\text{\qquad for $n>0$}\\
\intertext{For example, $5!=1\cdot2\cdot3\cdot4\cdot5=120$.  Note that for $n>1$,}
n! &= \prod_{i=1}^n\,i = \left(\prod_{i=1}^{n-1}\,i\right)\cdot n = \big((n-1)!\big)\cdot n
\end{align*}
It is also true that $n!=\big((n-1)!\big)\cdot n$ when $n=1$.  This observation
makes it possible to write a recursive function to compute $n!$.
(All the programming examples in this section are written in the Java
programming language.)
\begin{verbatim}
        int factorial( int n ) {
              // Compute n!.  Assume that n >= 0.
           int answer;
           if ( n == 0 ) {
              answer = 1;
           }
           else {
              answer = factorial( n-1 ) * n;
           } 
           return answer;
        }
\end{verbatim}
In order to compute \textit{factorial}($n$) for $n>0$, this function
first computes \textit{factorial}($n-1$) by calling itself recursively.
The answer from that computation is then multiplied by $n$ to give the
value of~$n!$.  The recursion has a base case,\index{base case} namely the case when
$n=0$.  For the base case, the answer is computed directly rather
than by using recursion.  The base case prevents the recursion from
continuing forever, in an infinite chain of recursive calls.

Now, as it happens, recursion is not the best way to compute $n!$.
It can be computed more efficiently using a loop.  Furthermore,
except for small values of $n$, the value of $n!$ is outside the
range of numbers that can be represented as 32-bit \textit{ints}.
However, ignoring these problems, the \textit{factorial} function
provides a nice first example of the interplay between recursion and
induction.  We can use induction to prove that \textit{factorial}($n$)
does indeed compute $n!$ for $n\ge 0$.  (In the proof, we pretend that
the data type \textit{int} is not limited to 32 bits.  In reality,
the function only gives the correct answer when the answer can be
represented as a 32-bit binary number.)

\begin{theorem}
Assume that the data type \textit{int} can represent arbitrarily large
integers.  Under this assumption, the \textit{factorial} function
defined above correctly computes $n!$ for any natural number $n$.
\end{theorem}
\begin{proof}
Let $P(n)$ be the statement ``\textit{factorial}($n$) correctly computes $n!$.''
We use induction to prove that $P(n)$ is true for all natural numbers~$n$.

Base case:  In the case $n=0$, the \textit{if} statement in the function
assigns the value 1 to the answer.  Since 1 is the correct value of
$0!$, \textit{factorial}(0) correctly computes $0!$.

Inductive case:  Let $k$ be an arbitrary natural number, and assume that
$P(k)$ is true.  From this assumption, we must show that $P(k+1)$ is true.
The assumption is that \textit{factorial}($k$) correctly computes $k!$,
and we want to show that \textit{factorial}($k+1$) correctly computes
$(k+1)!$.

When the function computes \textit{factorial}($k+1$), the value of
the parameter $n$ is $k+1$.
Since $k+1>0$, the \textit{if} statement in the function computes the
value of \textit{factorial}($k+1$) by applying the computation
\textit{factorial}$(k)*(k+1)$.  We know, by the induction hypothesis,
that the value computed by \textit{factorial}($k$) is $k!$.
It follows that the value computed by \textit{factorial}($k+1$)
is $(k!)\cdot(k+1)$.  As we observed above, for any $k+1>0$,
$(k!)\cdot(k+1)=(k+1)!$.  We see that \textit{factorial}($k+1$)
correctly computes $(k+1)!$.  This completes the induction.
\end{proof}

In this proof, we see that the base case of the induction corresponds
to the base case of the recursion, while the inductive case corresponds
to a recursive subroutine call.  A recursive subroutine call,
like the inductive case of an induction, reduces a problem to
a ``simpler'' or ``smaller'' problem, which is closer to the base
case.

Another standard example of recursion is the Towers of Hanoi problem.\index{Towers of Hanoi}
Let $n$ be a positive integer.  Imagine a set of $n$ disks of decreasing
size, piled up in order of size, with the largest disk on the bottom
and the smallest disk on top.  The problem is to move this tower of
disks to a second pile, following certain rules:  Only one disk
can be moved at a time, and a disk can only be placed on top of
another disk if the disk on top is smaller.  While the disks are
being moved from the first pile to the second pile, disks can be
kept in a third, spare pile.   All the disks must at all times be
in one of the three piles.  For example, if there are two disks,
the problem can be solved by the following sequence of moves:
\begin{verbatim}
         Move disk 1 from pile 1 to pile 3
         Move disk 2 from pile 1 to pile 2
         Move disk 1 from pile 3 to pile 2
\end{verbatim}
A simple recursive subroutine can be used to write out the list
of moves to solve the problem for any value of $n$.  The recursion is
based on the observation that for $n>1$, the problem can be
solved as follows:  Move $n-1$ disks from pile number 1 to pile number 3
(using pile number 2 as a spare).  Then move the largest disk, disk number $n$,
from pile number 1 to pile number 2.  Finally, move the $n-1$ disks from
pile number 3 to pile number 2, putting them on top of the $n^{th}$ disk
(using pile number 1 as a spare).  In both cases, the problem of
moving $n-1$ disks is a smaller version of the original problem and
so can be done by recursion.  Here is the subroutine, written in Java:
\begin{verbatim}
         void Hanoi(int n, int A, int B, int C) {
               // List the moves for moving n disks from
               // pile number A to pile number B, using
               // pile number C as a spare.  Assume n > 0.
            if ( n == 1 ) {
               System.out.println("Move disk 1 from pile "
                            + A + " to pile " + B);
             }
            else {
               Hanoi( n-1, A, C, B );
               System.out.println("Move disk " + n
                     + " from pile " + A + " to pile " + B);
               Hanoi( n-1, C, B, A );
            }
         }
\end{verbatim}
We can use induction to prove that this subroutine does in
fact solve the Towers of Hanoi problem.
\begin{theorem}
The sequence of moves printed by the \textit{Hanoi} subroutine as given above
correctly solves the Towers of Hanoi problem for any integer $n\ge1$.
\end{theorem}
\begin{proof}
We prove by induction that whenever $n$ is a positive integer and
$A$, $B$, and $C$ are the numbers 1, 2, and 3 in some order, 
the subroutine call \textit{Hanoi}($n,A,B,C$)
prints a sequence of moves that will move $n$ disks from pile $A$ to
pile $B$, following all the rules of the Towers of Hanoi problem.

In the base case, $n=1$,
the subroutine call \textit{Hanoi}($1,A,B,C$) prints out the single
step ``Move disk 1 from pile A to pile B,'' and this move does solve
the problem for 1 disk.

Let $k$ be an arbitrary positive integer, and suppose that
\textit{Hanoi}($k,A,B,C$) correctly solves the problem 
of moving the $k$ disks from pile $A$ to pile $B$ using pile $C$ as the spare,
whenever $A$, $B$, and $C$ are the numbers 1, 
2, and 3 in some order.  We need to show that 
\textit{Hanoi}($k+1,A,B,C$) correctly solves the problem for
$k+1$ disks.  Since $k+1>1$, \textit{Hanoi}($k+1,A,B,C$) begins by
calling \textit{Hanoi}($k,A,C,B$).  By the induction hypothesis,
this correctly moves $k$ disks from pile $A$ to pile $C$.  Disk number
$k+1$ is not moved during this process.
At that point, pile $C$ contains the $k$ smallest disks and
pile $A$ still contains the $(k+1)^{st}$ disk, which has not
yet been moved.  So the next move printed by the subroutine,
``Move disk $(k+1)$ from pile A to pile B,'' is legal because pile $B$ is empty.
Finally, the subroutine calls \textit{Hanoi}($k,C,B,A$),
which, by the induction hypothesis, correctly moves the $k$ smallest disks from 
pile $C$ to pile $B$, putting
them on top of the $(k+1)^{\text{st}}$ disk, which does not move during this process.
At that point, all $(k+1)$
disks are on pile $B$, so the problem for
$k+1$ disks has been correctly solved.
\end{proof}

Recursion is often used with linked data structures, which are
data structures that are constructed by linking several objects
of the same type together with pointers.  (If you don't already
know about objects and pointers, you will not be able to follow
the rest of this section.)  For an example, we'll look at
the data structure known as a \nw{binary tree}.\index{tree}
A binary tree consists of nodes linked together in a tree-like
structure.  The nodes can contain any type of data, but we will
consider binary trees in which each node contains an integer.
A binary tree can be empty, or it can consist of a node (called
the \nw{root} of the tree) and two smaller binary trees (called the
\nw{left subtree} and the \nw{right subtree} of the tree).
You can already see the recursive structure:  A tree can contain
smaller trees.  In Java, the nodes of a tree can be represented
by objects belonging to the class
\begin{verbatim}
         class BinaryTreeNode {
            int item;   // An integer value stored in the node.
            BinaryTreeNode left;   // Pointer to left subtree.
            BinaryTreeNode right;  // Pointer to right subtree.
         }
\end{verbatim}
An empty tree is represented by a pointer that has the special
value \nw[null pointer]{null}.  If \textit{root} is
a pointer to the root node of a tree, then \textit{root.left}
is a pointer to the left subtree and \textit{root.right} is a
pointer to the right subtree.  Of course, both \textit{root.left}
and \textit{root.right} can be \textit{null} if the corresponding
subtree is empty.  Similarly, \textit{root.item} is a name
for the integer in the root node.

Let's say that we want a function that will find the
sum of all the integers in all the nodes of a binary tree.
We can do this with a simple recursive function.  The base
case of the recursion is an empty tree.  Since there are no
integers in an empty tree, the sum of the integers in an
empty tree is zero.  For a non-empty tree, we can use recursion
to find the sums of the integers in the left and right subtrees,
and then add those sums to the integer in the root node of the
tree.  In Java, this can be expressed as follows:
\begin{verbatim}
         int TreeSum( BinaryTreeNode root ) {
               // Find the sum of all the integers in the
               // tree that has the given root.
            int answer;
            if ( root == null ) {  // The tree is empty.
               answer = 0;
            }
            else {
               answer = TreeSum( root.left );
               answer = answer + TreeSum( root.right );
               answer = answer + root.item;
            }
            return answer;
         }
\end{verbatim}
We can use the second form of the principle of mathematical induction
to prove that this function is correct.

\begin{theorem}
The function \textit{TreeSum}, defined above, correctly
computes the sum of all the integers in a binary tree.
\end{theorem}
\begin{proof}
We use induction on the number of nodes in the tree.
Let $P(n)$ be the statement ``\textit{TreeSum}
correctly computes the sum of the nodes in any binary tree
that contains exactly $n$ nodes.''  We show that $P(n)$ is true
for every natural number~$n$.

Consider the case $n=0$.  A tree with zero nodes is empty,
and an empty tree is represented by a \textit{null} pointer.
In this case, the \textit{if} statement in the definition of
\textit{TreeSum} assigns the value 0 to the answer, and this is
the correct sum for an empty tree.  So, $P(0)$ is true.

Let $k$ be an arbitrary natural number, with $k>0$.  Suppose we already
know $P(x)$ for each natural number $x$ with $0\le x < k$.  That is,
\textit{TreeSum} correctly computes the sum of all the integers in
any tree that has fewer than $k$ nodes.  We must show that it follows
that $P(k)$ is true, that is, that \textit{TreeSum} works for 
a tree with $k$ nodes.  Suppose that \textit{root} is a pointer
to the root node of a tree that has a total of $k$ nodes.
Since the root node counts as a node, that leaves a total of
$k-1$ nodes for the left and right subtrees, so each subtree
must contain fewer than $k$ nodes.  By the induction hypothesis,
we know that \textit{TreeSum}(\textit{root.left}) correctly
computes the sum of all the integers in the left subtree, and
\textit{TreeSum}(\textit{root.right}) correctly computes the
sum of all the integers in the right subtree.  The sum of all
the integers in the tree is \textit{root.item} plus the
sums of the integers in the subtrees, and this is the value
computed by \textit{TreeSum}.  So, \textit{TreeSum} does
work for a tree with $k$ nodes.  This completes the induction.
\end{proof}

Note how closely the structure of the inductive proof follows the 
structure of the recursive function.  In particular, the
second principle of mathematical induction is very natural here, since
the size of subtree could be anything up to one less than
the size of the complete tree.  It would be very difficult
to use the first principle of induction in a proof about
binary trees.

\begin{exercises}

\problem The \textit{Hanoi} subroutine given in this section does
not just solve the Towers of Hanoi problem.  It solves the
problem using the minimum possible number of moves.  Use induction
to prove this fact.

\problem Use induction to prove that the \textit{Hanoi} subroutine
uses $2^n-1$ moves to solve the Towers of Hanoi problem for $n$ disks.
(There is a story that goes along with the Towers of Hanoi problem.
It is said that on the day the world was created, a group of monks in Hanoi
were set the task of solving the problem for 64 disks.  They can
move just one disk each day.  On the day the problem is solved, 
the world will end.  However, we shouldn't worry too much,
since $2^{64}-1$ days is a very long time---about 50 million billion years.)

\problem Consider the following recursive function:
\begin{verbatim}
         int power( int x, int n ) {
               // Compute x raised to the power n.
               // Assume that n >= 0.
            int answer;
            if ( n == 0 ) {
               answer = 1;
            }
            else if (n % 2 == 0) {
               answer = power( x * x, n / 2);
            }
            else {
               answer = x * power( x * x, (n-1) / 2);
            }
            return answer;
         }
\end{verbatim}
Show that for any integer $x$ and any non-negative integer $n$,
the function \textit{power}($x$,$n$) correctly computes the value
of $x^n$.  (Assume that the \textit{int} data type can represent
arbitrarily large integers.)  Note that the test
``\verb/if (n % 2/ \verb/== 0)/'' tests whether $n$ is evenly divisible by 2.
That is, the test is true if $n$ is an even number.  (This function is
actually a very efficient way to compute $x^n$.)

\problem A \nw{leaf node} in a binary tree is a node in which 
both the left and the right subtrees are empty.  Prove that
the following recursive function correctly counts the number
of leaves in a binary tree:
\begin{verbatim}
         int LeafCount( BinaryTreeNode root ) {
               // Counts the number of leaf nodes in
               // the tree with the specified root.
            int count;
            if ( root == null ) {
               count = 0;
            }
            else if ( root.left == null && root.right == null ) {
               count = 1;
            }
            else {
               count = LeafCount( root.left );
               count = count + LeafCount( root.right );
            }
            return count;
         }
\end{verbatim}
\smallskip

\problem A \nw{binary sort tree}\index{sort tree} satisfies the
following property:  If \textit{node} is a pointer to any
node in the tree, then all the integers in the left subtree
of \textit{node} are less than \textit{node.item} and
all the integers in the right subtree of \textit{node} are
greater than or equal to \textit{node.item}.  Prove that the
following recursive subroutine prints all the integers in
a binary sort tree in non-decreasing order:
\begin{verbatim}
         void SortPrint( BinaryTreeNode root ) {
               // Assume that root is a pointer to the
               // root node of a binary sort tree.  This
               // subroutine prints the integers in the
               // tree in non-decreasing order.
            if ( root == null ) {
                  // There is nothing to print.
            }
            else {
               SortPrint( root.left );
               System.out.println( root.item );
               SortPrint( root.right );
            }
         }
\end{verbatim}
\smallskip


\end{exercises}



