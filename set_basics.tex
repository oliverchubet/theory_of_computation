\section{Basic Concepts}\label{S-sets-1}

A \nw{set} is a collection of \nw[element (of a set)]{elements}.
A set is defined entirely by the elements that it contains.
An element can be anything, including another set.
A set can be specified by listing the elements, enclosed by left ``$\{$'' and right ``$\}$'' braces.
For example, 
\[\{\ 17,\ \pi,\ \texttt{New York City},\ \texttt{Barack Obama},\ \texttt{Big Ben}\ \}.\]

It's important to understand that a set is defined by the elements that it contains, and not by the order in which those elements might be listed.
For example, the notations $\{a,b,c,d\}$ and $\{b,c,a,d\}$ define the same set.
Furthermore, a set can only contain one copy of a given element (even if the notation that specifies the set lists the element twice).
This means that $\{a,b,a,a,b,c,a\}$ and $\{a,b,c\}$ specify exactly the same set.
Note in particular that it's incorrect to say that the set $\{a,b,a,a,b,c,a\}$ contains seven elements, since some of the elements in the list are identical.

The symbol $\in$ is used to express the relation ``is an element of.''
The assertion that $a$ is not an element of $A$ is expressed by the notation $a\not\in A$.
Note that both $a\in A$ and $a\not\in A$ are statements in the sense of propositional logic.
That is, they are assertions which can be either true or false.
The statement $a\not\in A$ is equivalent to $\NOT(a\in A)$.

Since a set is completely determined by the elements that it contains, there is only one empty set, denoted $\emptyset$.
Note that for any element $a$, the statement $a\in\emptyset$ is false.
The empty set, $\emptyset$, can also be denoted by an empty pair of braces, $\{$~$\}$.

If $A$ and $B$ are sets, then, by definition, $A$ is equal to $B$ if and only
if they contain exactly the same elements.  In this case, we write $A=B$.
Suppose now that $A$ and $B$ are sets such that every element of $A$ is an element of $B$.  
In that case, we say that $A$ is a \nw{subset} of~$B$.
 The fact that $A$ is a subset of $B$ is denoted by $A\SUB B$.
If $A\SUB B$ but $A\not= B$, we say that $A$ is a \nw{proper subset} of $B$.
Note that $\emptyset$ is a subset of every set $B$: $x \in \emptyset$ is false for any $x$.
If $A=B$, then it is automatically true that $A\SUB B$ and that
$B\SUB A$.
The converse is also true: If $A\SUB B$ and $B\SUB A$, then $A=B$.

\begin{theorem}\label{T-setequality}
Let $A$ and $B$ be sets.  Then $A=B$ if and only if both $A\SUB B$ and $B\SUB A$.
\end{theorem}

This theorem implies that one can prove that two sets, $A$ and $B$, are equal, in two steps.
First check that every element of $A$ is also an element of $B$, and then check that every element of $B$ is also an element of $A$.
 
\medbreak

A set can contain an infinite number of elements.
In such a case, it is not possible to list all the elements in the set.
Sometimes the ellipsis ``\dots'' is used to indicate a list that continues on infinitely.
For example, $\N$, the set of natural numbers, can be specified as
\[\N = \{ 0, 1, 2, 3, \dots \}\]
However, this is an informal notation, which is not really well-defined, and it should only be used in cases where it is clear what it means.
It's not very useful to say that ``the set of prime numbers is $\{2,3,5,7,11,13,\dots\}$,'' and It is completely meaningless to talk about ``the set $\{17,42,105,\dots\}$.''
Clearly, we need another way to specify sets besides listing their elements.
The need is fulfilled by set-builder notation.

If $P(x)$ is a predicate, then we can form the set that contains all elements $a$ such that $P(a)$ is true.
The notation $\{x \st P(x)\}$ is used to denote this set.
The notation $\{x\st P(x)\}$ can be read ``the set of $x$ such that $P(x)$.''
For example, if $E(x)$ is the predicate ``$x$ is an even number,'' then in set-builder notation $\{x\st E(x)\}$ specifies the set of even numbers.
That is, 
\[\{x\st E(x)\} = \{0,2,4,6,8,\dots\}\]
It turns out, for deep and surprising reasons that we will discuss later in this
section, that we have to be a little careful about what counts as a predicate.  
In order for the set $\{x\st P(x)\}$ to be valid, we have to assume that the domain of $P$ is in fact a set.  
(You might wonder how it could be anything else.  That's the surprise!)
Often, in set-builder notation it is useful to specify the domain explicitly.
For example, to make it clear that $x$ must be a natural number, we could write the set as $\{x\in\N \st  E(x)\}$.
This notation can be read as ``the set of all $x$ in $\N$ such that $E(x)$.''
More generally, if $X$ is a set and $P$ is a predicate whose domain includes all the elements of $X$, then 
\[\{x\in X\st P(x)\}\]
is the set that consists of all elements $a$ that are members of the set $X$ and for which $P(a)$ is true.
In set-builder notation, we don't have to assume that the domain for $P$ is a set, since we are effectively limiting the domain to the set $X$.
The set denoted by $\{x\in X\st P(x)\}$ could also be written as $\{x\st x\in X\AND P(x)\}$.

%We can use set-builder notation to define the set of prime numbers in a rigorous way.
%A prime number is a natural number $n$ which is greater than 1 and which satisfies the property that for any factorization $n=xy$, where $x$ and $y$ are natural numbers, either $x$ or $y$ must be $n$.
%We can express this definition as a predicate and define the set of prime numbers as
%\[\{n\in\N\st (n>1)\AND\forall x\forall y\big((x\in\N\AND y\in\N\AND n=xy)\IMP(x=n\OR y=n)\big)\}.\]
%Admittedly, this definition is hard to take in in one gulp.
%But this example
%shows that it is possible to define complex sets using set-builder notation.



\medbreak

Now that we have a way to express a wide variety of sets, we turn to operations that
can be performed on sets.
The most basic operations on sets are \nw{union} and \nw{intersection}.
If $A$ and $B$ are sets, then we define the union of $A$ and $B$ to be the set that contains all the elements of $A$ together with all the elements of $B$.
The union of $A$ and $B$ is denoted by $A\cup B$.
The union can be defined formally as
\[A\cup B = \{x\st x\in A \OR x\in B\}.\]
The intersection of $A$ and $B$ is defined to be the set that contains every element that is both a member of $A$ and a member of $B$.
The intersection of $A$ and $B$ is denoted by $A\cap B$.
Formally,
\[A\cap B = \{x\st x\in A \AND x\in B\}.\]
An element gets into $A\cup B$ if it is in \emph{either} $A$ or $B$.
It gets into $A\cap B$ if it is in \emph{both} $A$ and $B$.
Note that the symbol for the logical ``or'' operator, $\OR$, is similar to the symbol for the union operator,~$\cup$, while the logical ``and'' operator, $\AND$, is similar to the intersection operator,~$\cap$.

The \nw{set difference} of two sets, $A$ and $B$, is defined to be
the set of all elements that are members of $A$ but are not members
of $B$.
The set difference of $A$ and $B$ is denoted $A\SETDIFF B$.
The idea is that $A\SETDIFF B$ is formed by starting with $A$ and then removing any element that is also found in $B$.
Formally,
\[A\SETDIFF B = \{x\st x\in A \AND x\not\in B\}.\]
Union and intersection are clearly commutative operations.
That is, $A\cup B=B\cup A$ and $A\cap B=B\cap A$ for any sets $A$ and $B$.
However, set difference is not commutative.
In general, $A\SETDIFF B \not= B\SETDIFF A$.

Suppose that $A=\{a,b,c\}$, that $B=\{b,d\}$, and
that $C=\{d,e,f\}$.  Then we can apply the definitions of
union, intersection, and set difference to compute, for example,
that:
\begin{align*}
   A\cup B &= \{a,b,c,d\} &  A\cap B &= \{b\}   &  A\SETDIFF B &= \{a,c\}\\
   A\cup C &= \{a,b,c,d,e,f\}&A\cap C&= \emptyset & A\SETDIFF C &= \{a,b,c\}
\end{align*}
In this example, the sets $A$ and $C$ have no elements in common, so
that $A\cap C=\emptyset$.  There is a term for this:
Two sets are said to be \nw[disjoint sets]{disjoint} if they
have no elements in common.  That is, for any sets $A$ and $B$,
$A$ and $B$ are said to be disjoint if and only if $A\cap B=\emptyset$.


Of course, the set operations can also be applied to sets that are defined by set-builder notation.
For example, let $L(x)$ be the predicate ``$x$ is lucky,'' and let $W(x)$ be the predicate ``$x$ is wise,'' where the domain for each predicate is the set of people.
Let $X = \{x\st L(x)\}$, and let $Y=\{x\st W(x)\}$.
Then
\begin{align*}
  X\cup Y &= \{x\st L(x)\OR W(x)\} = \{\text{people who are lucky or wise}\}\\
  X\cap Y &= \{x\st L(x)\AND W(x)\} = \{\text{people who are lucky and wise}\}\\
  X\SETDIFF Y &= \{x\st L(x)\AND \NOT W(x)\} = \{\text{people who are lucky but not wise}\}\\
  Y\SETDIFF X &= \{x\st W(x)\AND \NOT L(x)\} = \{\text{people who are wise but not lucky}\}
\end{align*}
You have to be a little careful with the English word ``and.''
We might say that the set $X\cup Y$ contains people who are lucky \emph{and} people who are wise.
But what this means is that a person gets into the set $X\cup Y$ either by being lucky \emph{or} by being wise, so $X\cup Y$ is defined using the logical ``or'' operator,~$\OR$.

\medbreak

\fig
  {F-setops}
  {Some of the notations that are defined in this section.  $A$ and $B$ are
   sets, and $a$ is an element.}
  {\begin{tabular}{|c|l|}
        \hline
        \strut\textbf{Notation} & \textbf{Definition}\\
        \hline
        \strut $a\in A$      & $a$ is a member (or element) of $A$\\ 
        \strut $a\not\in A$  & $\NOT(a\in A)$, $a$ is not a member of $A$\\
        \strut $\emptyset$   & the empty set, which contains no elements\\
        \strut $A\SUB B$     & $A$ is a subset of $B$, $\forall x$ if $x\in A$ then $x\in B$\\
        \strut $A\SUP B$     & $A$ is a superset of $B$, same as $B\SUB A$\\
        \strut $A=B$         & $A$ and $B$ have the same members, $A\SUB B\AND B\SUB A$\\
        \strut $A\cup B$     & union of $A$ and $B$, $\{x\st x\in A\OR x\in B\}$\\
        \strut $A\cap B$     & intersection of $A$ and $B$, $\{x\st x\in A\AND x\in B\}$\\
        \strut $A\SETDIFF B$ & set difference of $A$ and $B$, $\{x\st x\in A\AND x\not\in B\}$\\
        \strut $\POW(A)$     & power set of $A$, $\{X\st X\SUB A\}$\\
        \hline
     \end{tabular}
   }

Sets can contain other sets as elements.
For example, the notation $\{a,\{b\}\}$ defines a set that contains two elements, the element $a$ and the set $\{b\}$.
Since the set $\{b\}$ is a member of the set $\{a,\{b\}\}$, we have that $\{b\}\in\{a,\{b\}\}$.
On the other hand, provided that $a\not=b$, the statement $\{b\}\SUB\{a,\{b\}\}$ is false, since saying $\{b\}\SUB\{a,\{b\}\}$ is equivalent to saying that $b\in\{a,\{b\}\}$, and the element $b$ is not one of the two members of $\{a,\{b\}\}$.
For the element $a$, it is true that $\{a\}\SUB\{a,\{b\}\}$.

Given a set $A$, we can construct the set that contains all the subsets of $A$.
This set is called the \nw{power set} of $A$, and is denoted $\POW(A)$.
Formally, we define
\[\POW(A)=\{X\st X\SUB A\}.\]
For example, if $A=\{a,b\}$, then the subsets of $A$ are the empty set, $\{a\}$, $\{b\}$, and $\{a,b\}$, so the power set of $A$ is set given by
\[\POW(A) = \{\,\emptyset,\,\{a\},\,\{b\},\,\{a,b\}\,\}.\]
Note that since the empty set is a \emph{subset} of any set, the empty set is an \emph{element} of the power set of any set.
That is, for any set $A$, $\emptyset\SUB A$ and $\emptyset\in\POW(A)$.
Since the empty set is a subset of itself, and is its only subset, we have that $\POW(\emptyset) = \{\emptyset\}$.
The set $\{\emptyset\}$ is not empty.
It contains one element, namely~$\emptyset$.


\medbreak

We remarked earlier in this section that constructing a set, $\{x \st P(x)\}$, with set-builder notation is only valid if the domain of $P$ is a set.
This might seem a rather puzzling thing to say---after all, why and how would the domain be anything else?
The answer is related to Russell's Paradox, which we mentioned briefly in Chapter~\ref{C-proof} and which shows that it is logically impossible for the set of all sets to exist.
This impossibility can be demonstrated using a proof by contradiction.
In the proof, we use the existence of the set of all sets to define another set which cannot exist because its existence would lead to a logical contradiction.

\begin{theorem}\label{T-Russell}
There is no set of all sets.
\end{theorem}
\begin{proof}
Suppose that the set of all sets exists.
We will show that this assumption leads to a contradiction.
Let $V$ be the set of all sets.
We can then define the set $R$ to be the set which contains every set that does not contain itself.
That is,
\[R=\{X\in V\st X\not\in X\}\]
Now, we must have either $R\in R$ or $R\not\in R$.
We will show that either case leads to a contradiction.

Consider the case where $R\in R$.
Since $R\in R$, $R$ must satisfy the condition for membership in $R$.
A set $X$ is in $R$ iff $X\not\in X$.
To say that $R$ satisfies this condition means that $R\not\in R$.
That is, from $R\in R$, we deduce the contradiction that $R\not\in R$.

Now consider the remaining case, where $R\not\in R$.
Since $R\not\in R$, $R$ does not satisfy the condition for membership in $R$.
Since the condition for membership is that $R\not\in R$, and this condition is false, the statement $R\not\in R$ must be false.
But this means that the statement $R\in R$ is true.
From $R\not\in R$, we deduce the contradiction that $R\in R$.

Since both possible cases, $R\in R$ and $R\not\in R$, lead to contradictions, we see that it is not possible for $R$ to exist.
Since the existence of $R$ follows from the existence of $V$, we see that $V$ also cannot exist.
\end{proof}

To avoid Russell's paradox, we must put limitations on the construction of new sets.
We can't force the set of all sets into existence simply by thinking of it.
We can't form the set $\{x\st P(x)\}$ unless the domain of $P$ is a set.
Any predicate $Q$ can be used to form a set $\{x\in X\st Q(x)\}$, but set-builder notation requires a pre-existing set $X$.
Predicates can be used to form subsets of existing sets, but they can't be used to form new sets completely from scratch.

\medskip

The notation $\{x\in A\st P(x)\}$ is a convenient way to effectively limit the domain of a predicate, $P$, to members of a set, $A$, that we are actually interested in.
We will use a similar notation with the quantifiers $\forall$ and~$\exists$.
The proposition $(\forall x\in A)(P(x))$ is true if and only if $P(a)$ is true for every element $a$ of the set $A$.
And the proposition $(\exists x\in A)(P(x))$ is true if and only if there is some element $a$ of the set $A$ for which $P(a)$ is true.
These notations are valid only when $A$ is contained in the domain for~$P$.
As usual, we can leave out parentheses when doing so introduces no ambiguity.
So, for example, we might write $\forall x\in A\;P(x)$.

\medskip

We end this section with proofs of the two forms of the principle of mathematical induction.
These proofs were omitted from the previous chapter, but only for the lack of a bit of set notation.
In fact, the principle of mathematical induction is valid only because it follows from one of the basic axioms that define the natural numbers, namely any non-empty set of natural numbers has a smallest element.
Given this axiom, we can use it to prove the following two theorems:


\begin{theorem}\label{T-induction}
Let $P$ be a predicate whose domain includes the natural numbers.
Suppose that $P(0)$ is true and for all $k\in \N$ if $P(k)$ then $P(k+1)$.
Then for all $n\in\N$ we have $P(n)$.
\end{theorem}
\begin{proof}
Suppose that $P(0)$ is true. 
Additionally, suppose that for all $k\in \N$ if $P(k)$ then $P(k+1)$.
However, suppose for all $n\in \N$ that $P(n)$ is false.
We show that this assumption leads to a contradiction.

Since the statement $\forall n\in\N,\,P(n)$ is false, its negation, $\NOT(\forall n\in\N,\,P(n))$, is true.
The negation is equivalent to $\exists n\in\N,\,\NOT P(n)$.
Let $X=\{n\in\N\st \NOT P(n)\}$.
Since $\exists n\in\N,\,\NOT P(n)$ is true, we know that $X$ is not empty.
Since $X$~is a non-empty set of natural numbers, it has a smallest element.
Let $x$ be the smallest element of $X$.
That is, $x$ is the smallest natural number such that $P(x)$ is false.
Since we know that $P(0)$ is true, $x$ cannot be~0.
Let $y=x-1$.
Since $x\not=0$, $y$ is a natural number.
Since $y<x$, we know, by the definition of $x$, that $P(y)$ is true.
We also know that for all $k\in\N$ if $P(k)$ then $P(k+1)$.
In particular, taking $k=y$, we know that if $P(y)$ then $P(y+1)$.
Since $P(y)$ and $P(y)\IMP P(y+1)$, we deduce by \textit{modus ponens} that $P(y+1)$ is true.
But $y+1=x$, so we have deduced that $P(x)$ is true.
This contradicts that $P(x)$ is false.
This contradiction proves the theorem.
\end{proof}


\begin{theorem}
Let $P$ be a predicate whose domain includes
the natural numbers.  Suppose that $P(0)$ is true and that
\[(P(0)\AND P(1)\AND\cdots\AND P(k))\IMP P(k+1)\]
is true for each natural number $k\geq 0$.  
Then it is true that $\forall n\in\N,\,P(n)$.
\end{theorem}
\begin{proof}
Suppose that $P$ is a predicate that satisfies the hypotheses of the theorem, and suppose that the statement $\forall n\in\N,\,P(n)$ is false.
We show that this assumption leads to a contradiction.

Let $X=\{n\in\N\st \NOT P(n)\}$.
Because of the assumption that $\forall n\in\N,\,P(n)$ is false, $X$ is non-empty.
It follows that $X$ has a smallest element.
Let $x$ be the smallest element of $X$.
The assumption that $P(0)$ is true means that $0\not\in X$, so we must have $x>0$.
Since $x$ is the smallest natural number for which $P(x)$ is false, we know that $P(0)$, $P(1)$, \dots, and $P(x-1)$ are all true.
From this and that $(P(0)\AND P(1)\AND\cdots\AND P(x-1))\IMP P(x)$, we deduce that $P(x)$ is true.
But this contradicts that $P(x)$ is false.
This contradiction proves the theorem.
\end{proof}



