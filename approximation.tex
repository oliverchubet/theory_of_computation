\chapter{Approximation Algorithms}\label{C-approx}

There are many problems for which we do not have efficient algorithms.
Furthermore, we do not know whether efficient solutions exist in general for these problems.
However, in some of these cases, we can efficiently compute approximate solutions.
In this section we discuss some well-known optimization problems.
We define what an approximate solution is and give some examples of approximation algorithms.

\section{Decision Problems vs. Optimization Problems}

Decision problems are problems that can be answered with a "yes" or "no".
A solution to a decision problem is an algorithm that decides a language corresponding to the given problem.

For example:
\begin{itemize}
\item \textbf{Vertex Cover} asks whether a given graph $G$ has a vertex cover of cardinality $k$.
The corresponding language is $VC = \{\langle G, k\rangle \mid G \text{ has a vertex cover of size }k\}$.
\item \textbf{Traveling Sales Person} asks whether a given graph $G$ (with weighted edges) has a tour of cost $c$.
The corresponding language is $TSP = \{\langle G, c\rangle \mid G \text{ has a tour of cost }c\}$.
\end{itemize}

Some decision problems can be reframed as optimization problems.
Instead of asking for "yes" or "no", an optimization problem asks for a minimum or maximum value solution.

For example:
\begin{itemize}
\item \textbf{Vertex Cover}, given a graph $G$, asks for the minimum cardinality over all vertex covers for $G$.
\item \textbf{Traveling Sales Person}, given a graph $G$ with weighted edges, asks for the minimum cost $c$ over all tours of $G$.
\end{itemize}

An optimization problem consists of the following:
\begin{itemize}
\item instances $I$,
\item a set of feasible solutions $F$, and
\item a cost function $c: F \to \R$.
\end{itemize}

A solution $x \in F$ is considered optimal if $c(x)$ has the minimum cost over all solutions in $F$.
We denote the optimal solution for a given instance as $\textsf{opt}(I)$.

Let's look one more time at Traveline Sales Person:
\begin{itemize}
\item An instance of Traveling Sales Person is a graph with weighted edges.
\item The set of feasible solutions is the set of all possible tours over $G$.
\item The cost is a function that takes a tour and adds the sum of the weights of the edges in that tour.
\end{itemize}

\section{Approximation Algorithms}

Approximation algorithms only make sense in the context of optimization problems.
It isn't clear what an approximation to a decision procedure would be.
However, one may guess that an approximation algorithm to an optimization problem would find a solution that was "close" to the optimal solution.
Now we only need to define precisely what we mean by "close".

Let $\alpha \ge 1$.
An $\alpha$-approximate algorithm to an optimization problem $P$ is an algorithm $A$ that on instance $I$ returns a solution $x$ such that $c(x) \le \alpha c(\textsf{opt}(I))$.

In otherwords, the cost of the solution given by the approximation algorithm is at most $\alpha$ times larger than the cost of the optimal solution.

In the case of Traveling Sales Person, a $2$-approximation algorithm would find a tour with a cost no more than twice the cost of the optimal tour.
Similarly, a $2$-approximation for Vertex Cover would be a vertex cover with cardinality at most twice of the optimal.

\section{Vertex Cover}

A vertex cover for a graph $G = (V, E)$ is a subset of the vertices $S \subseteq V$ such that every edge $e \in E$ has at least one endpoint in $S$.
We have already established that as a decision problem that Vertex Cover is NP-hard.
Does that necessarily mean that Vertex Cover is NP-hard as an optimization problem?

\textbf{Claim} Vertex Cover as an optimization problem is NP-hard.

The claim is easy to see by a reduction from the decision problem to the optimization problem.

In this section we look at a $2$-approximation algorithm for Vertex Cover.
The algorithm is simple once we understand maximal matchings.

Let $G = (V,E)$ be a graph.
A matching $M$ in $G$ is a set of edges where no two edges share an endpoint.
The matching $M$ is maximal if we cannot add any edges of $G$ and still have a matching.

TODO:
\begin{itemize}
\item Hint: maximal vs maximum
\end{itemize}

\includegraphics{figs/maximum_vs_maximal.pdf}

\includegraphics{figs/maximum_vs_maximal_2.pdf}

%![maximum vs maximal fig](../figs/maximum_vs_maximal_2.pdf)

We can compute maximal matchings efficiently using a greedy algorithm: keep adding edges until there are none left that can be added and still give a matching.

\textbf{Claim}: A maximal matching gives us a $2$-approximation for a minimum cardinality vertex cover.

TODO: prove claim
\begin{itemize}
\item If a vertex is in an optimal vertex cover then it is an endpoint of an edge in the maximal matching.
\item For every edge in the vertex cover, at least one vertex is in the optimal solution.
\item There are at most $2$-times as many vertices in the maximal matching as in the optimal solution.
\item Give things names
\item picture example
\end{itemize}

\section{Metric Traveling Sales Person}

