
\chapter{Grammars}\label{C-grammars}


\startchapter{Both natural languages, }such as English and the
artificial languages used for programming have a structure
known as grammar or syntax.  In order to form legal sentences
or programs, the parts of the language must be fit together
according to certain rules.  For natural languages, the
rules are somewhat informal (although high-school English
teachers might have us believe differently).  For programming
languages, the rules are absolute, and programs that violate
the rules will be rejected by a compiler.

In this chapter, we will study formal grammars and languages
defined by them.  The languages we look at will, for the most part,
be ``toy'' languages, compared to natural languages or even
to programming languages, but the ideas and techniques are basic
to any study of language.  In fact, many of the ideas arose
almost simultaneously in the 1950s in the work of linguists who were
studying natural language and programmers who were looking for
ways to specify the syntax of programming languages.

The grammars in this chapter are \nw[generative grammar]{generative grammars}.
A generative grammar is a set of rules that can be used to generate
all the legal strings in a language.  We will also consider the closely
related idea of \nw{parsing}.  To parse a string means to determine
how that string can be generated according to the rules.

This chapter is a continuation of the preceding chapter.  
Like a regular expression, a grammar is a way to specify a possibly
infinite language with a finite amount of information.  In fact,
we will see that every regular language can be specified
by a certain simple type of grammar.  We will also see that some languages
that can be specified by grammars are not regular.

\endinput

