\usepackage{amsmath}
\usepackage{amsfonts}
\usepackage{amssymb}
\usepackage[mathscr]{eucal}
\usepackage{amsthm}
\usepackage{makeidx}
\usepackage{ifthen}
%\usepackage[dvips]{graphics}    %change to dvips for production
\usepackage{graphics}    %change to dvips for production
\makeindex

%\usepackage[dvips]{hyperref}
\usepackage{hyperref}
\hypersetup{colorlinks=true,
   bookmarksnumbered=true,
   filecolor=blue,
   breaklinks=true,
   urlcolor=blue
}

\usepackage[margin=1in]{geometry}

%\setlength{\headsep}{0.4truein}             % Sizes for a 6 inch by 9 inch version
%\setlength{\topmargin}{-0.3 true in}
%\setlength{\topskip}{0 true in}
%\setlength{\textwidth}{4.5 true in}
%\setlength{\oddsidemargin}{-0.1 true in}
%\setlength{\evensidemargin}{-0.4 true in}
%\setlength{\textheight}{6.8 true in}

%\special{papersize=6in,9in}      % This is processed by dvips/dvipdf

%\newcommand{\breakSixByNine}{\newpage}  % for inserting a forced line break in the 6-by-9 version
                                     % Version 2.1.  (Also see definition of \eps below)
                                     % Used only once, in turing.tex
                                     
\newcommand{\breakSixByNine}{}

% FOR 6-BY-9 version, an extra scaling factor of 0.8 is added to the \eps command.

%\newcommand*{\eps}[1]{\scalebox{0.8}{\includegraphics{figures/#1.eps}}}

\newcommand*{\eps}[1]{\includegraphics{figures/#1.eps}}

\newcommand*{\scaledeps}[2]{\resizebox{#1}{!}{\eps{#2}}}

\newtheorem{theorem}{Theorem}[chapter]       % theorems,lemmas corrolaries
\newtheorem{lemma}[theorem]{Lemma}           % numbered together, by chapter
\newtheorem{corrolary}[theorem]{Corollary}

\theoremstyle{definition}
\newtheorem{definition}{Definition}[chapter]
\newtheorem{example}{Example}[chapter]



%\nw is "new word"
%\nw{text} sets text in slanted font and puts it in the index
%\nw[none]{text} sets text in slanted font, but does not put it in the index
%\nw[index-entry]{text} puts text in slanted font and puts index-entry in the index
\newcommand*{\nw}[2][]{\textsl{\textbf{#2}}\ifthenelse{\equal{#1}{}}{\index{#2}}{\ifthenelse{\equal{#1}{none}}{}{\index{#1}}}}


\newcommand*{\largesize}[1]{{\Large#1}}
\newcommand*{\startchapter}[1]{\textsc{\largesize #1}}  % for first few words


%\newcommand{\AND}{\wedge}
\newcommand{\AND}{\text{ and }}
%\newcommand{\OR}{\vee}
\newcommand{\OR}{\text{ or }}
\newcommand{\NOT}{\lnot}
\newcommand{\IMP}{\rightarrow}
\newcommand{\IFF}{\leftrightarrow}
\newcommand{\XOR}{\oplus}
\newcommand{\T}{\mathbb{T}}
\newcommand{\F}{\mathbb{F}}
\newcommand{\LOGIMP}{\Longrightarrow}

%For Chapter 2
\newcommand{\N}{{\mathbb N}}
\newcommand{\Z}{{\mathbb Z}}
\newcommand{\Q}{{\mathbb Q}}
\newcommand{\R}{{\mathbb R}}
\newcommand{\Zpos}{{{\mathbb Z}^+}}
\newcommand{\SUB}{\subseteq}
\newcommand{\SUP}{\supseteq}
\newcommand{\PSUB}{\varsubsetneq}
\newcommand{\PSUP}{\varsupsetneq}
\newcommand{\SETDIFF}{\setminus}   % the set difference operator.
\newcommand{\POW}{{\mathscr P}}   % For the power set of a set
\newcommand{\st}{\,|\,}   % for the | in sets { x | P(x) }


\newcounter{problemcounter}
\newcounter{partcounter}[problemcounter]

\newcommand{\Item}[1]{\par\hangafter=0
                         \hangindent=15pt
                         \noindent\llap{#1}\ignorespaces}
\newcommand{\IItem}[1]{\par\hangafter=0
                         \hangindent=40pt
                         \noindent\llap{#1}\ignorespaces}

\newcommand{\problem}{\smallskip\stepcounter{problemcounter}\Item{\bfseries\arabic{problemcounter}.\ }}
\newcommand{\ppart}{\stepcounter{partcounter}\IItem{\bfseries\alph{partcounter})\ }}
\newcommand{\pparts}[1]{\vskip\parskip
   \halign{\hskip40pt\stepcounter{partcounter}\llap{\bfseries\alph{partcounter})\ }$##$\qquad\hfil&&
            \hskip30pt\stepcounter{partcounter}\llap{\bfseries\alph{partcounter})\ }$##$\qquad\hfil\cr
   #1\crcr}}
%\tparts is same as \pparts, but doesn't use math mode
\newcommand{\tparts}[1]{\vskip\parskip
   \halign{\hskip40pt\stepcounter{partcounter}\llap{\bfseries\alph{partcounter})\ }##\qquad\hfil&&
            \hskip30pt\stepcounter{partcounter}\llap{\bfseries\alph{partcounter})\ }##\qquad\hfil\cr
   #1\crcr}}
\newenvironment{exercises}
   {\setcounter{problemcounter}{0}
    \bigbreak\medbreak
    \leftline{\bfseries\large Exercises}
    \medskip
    \small}
   {}
    
    
\renewcommand{\strut}{\vrule width 0pt depth 1ex height 2.5ex}
\newcommand{\bigstrut}{\vrule width 0pt depth 1ex height 3ex}

%arguments are: label,caption,figure data
\newcommand{\fig}[3]{\begin{figure}[t]
  \begin{center}
    #3
  \parbox{0.9\textwidth}{\textit{\caption{\label{#1}#2}}}
  \end{center}
  \end{figure}
}

%Added for Chapter 6

\newcommand{\EMPTYSTRING}{\varepsilon}
\newcommand{\PRODUCES}{\longrightarrow}
\newcommand{\YIELDS}{\Longrightarrow}
\newcommand{\YIELDSTAR}{{\Longrightarrow}^*}
\newcommand{\NT}[1]{\hbox{$\langle$\textit{#1}$\rangle$}}
\newcommand{\BNFPRODUCES}{\hbox{\texttt{::=}}}
\newcommand{\BNFALT}{\hbox{$|$}}



