\begin{exercises}

\problem The language $L=\{a^m\st m>0\}$ is the range of the function
$f(a^n)=a^{n+1}$.  Design a Turing machine that computes this function,
and find the grammar that generates the language $L$ by
imitating the computation of that machine.

\problem Complete the proof of Theorem \ref{T-re} by proving
the following:  If $L$ is a recursive language over an
alphabet $\Sigma$, then both
$L$ and $\Sigma^*\SETDIFF L$ are recursively enumerable.

\problem Show that a language $L$ over an alphabet $\Sigma$
is recursive if and only if there are grammars $G$
and $H$ such that the language generated by $G$ is $L$ and the
language generated by $H$ is $\Sigma^*\SETDIFF L$.

\problem This section discusses recursive languages and recursively
enumerable languages.  How could one define recursive subsets of
$\N$ and recursively enumerable subsets of $\N$?

\problem Give an informal argument to show that a subset $X\SUB\N$ is
recursive if and only if there is a computer program
that prints out the elements of $X$ {\it in increasing order}.

\end{exercises}

