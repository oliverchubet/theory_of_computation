\section{Non-context-free Languages}\label{S-grammars-4}

We have seen that there are context-free languages that are not
regular.  The natural question arises, are there languages that
are not context-free?  It's easy to answer this question in the
abstract:  For a given alphabet $\Sigma$, there are uncountably
many languages over $\Sigma$, but there are only
countably many context-free languages over $\Sigma$.  It follows
that most languages are not context-free.  However, this answer
is not very satisfying since it doesn't give us any example of
a specific language that is not context-free.

As in the case of regular languages, one way to show that
a given language $L$ is not context-free is to find some property
that is shared by all context-free languages and then to show that
$L$ does not have that property.  For regular languages, the
Pumping Lemma gave us such a property.  It turns out that
there is a similar Pumping Lemma for context-free languages.
The proof of this lemma uses parse trees.  In the proof, we
will need a way of representing abstract parse trees, without
showing all the details of the tree.  The picture

\medskip
\centerline{\scaledeps{0.5truein}{fig-5-4}}
\smallskip

\noindent represents a parse tree which has the non-terminal symbol
$A$ at its root and the string $x$ along the ``bottom'' of the tree.
(That is, $x$ is the string made up of all the symbols at the
endpoints of the tree's branches, read from left to right.)  Note that
this could be a partial parse tree---something that could be a part of a
larger tree.  That is, we do not require $A$ to be the start symbol
of the grammar and we allow $x$ to contain both terminal and
non-terminal symbols.  The string $x$, which is along the bottom
of the tree, is referred to as the \nw[yield of a parse tree]{yield}
of the parse tree.  Sometimes, we need to show more explicit detail in
the tree.  For example, the picture

\medskip
\centerline{\scaledeps{1truein}{fig-5-5}}

\noindent represents a parse tree in which the yield is the
string $xyz$.  The string $y$ is the yield of a smaller tree, with
root $B$, which is contained within the larger tree.
Note that any of the strings $x$, $y$, or $z$ could be the
empty string.  

We will also need the concept of the \nw[height of
a parse tree]{height} of a parse tree.  The height of a parse tree is
the length of the longest path from the root of the tree to the
tip of one of its branches.

Like the version for regular languages, the Pumping Lemma for
context-free languages shows that any sufficiently long string
in a context-free language contains a pattern that can be repeated
to produce new strings that are also in the language.  However,
the pattern in this case is more complicated.  For regular
languages, the pattern arises because any sufficiently long path
through a given DFA must contain a loop.   For context-free
languages, the pattern arises because in a sufficiently
large parse tree, along a path from the root of the tree to the
tip of one of its branches, there must be some non-terminal
symbol that occurs more than once.

\begin{theorem}[Pumping Lemma for Context-free Languages]
Suppose that $L$ is a context-free language.
Then there is an integer $K$ such that any string $w\in L(G)$
with $|w|\ge K$ has the property that $w$ can be written
in the form $w=uxyzv$ where
\Item{$\bullet\,$}$x$ and $z$ are not both equal to the empty string;
\Item{$\bullet\,$}$|xyz|< K$; and
\Item{$\bullet\,$}For any $n\in\N$, the string $ux^nyz^nv$ is in $L$.
\end{theorem}
\begin{proof}
Let $G=(V,\Sigma,P,S)$ be a context-free grammar for the language $L$.
Let $N$ be the number of non-terminal symbols in $G$, plus 1.  That is,
$N=|V|+1$.  Consider all possible parse trees for the grammar $G$
with height less than or equal to $N$.  (Include parse trees with any
non-terminal symbol as root, not just parse trees with root $S$.) 
There are only finitely many such parse trees, and therefore there
are only finitely many different strings that are the yields of
such parse trees.  Let $K$ 
be an integer which is greater than the length of any such string.

Now suppose that $w$ is any string in $L$ whose length is greater
than or equal to $K$.  Then \textit{any} parse tree for $w$ must have height
greater than $N$.  (This follows since $|w|\ge K$ and the yield of 
any parse tree of height $\le N$ has length less than $K$.)
Consider a parse tree for $w$ of minimal size, that is one that contains
the smallest possible number of nodes.  Since the height of this parse
tree is greater than $N$, there is at least one path from the
root of the tree to tip of a branch of the tree that has length
greater than $N$.  Consider the longest such path.  The symbol at
the tip of this path is a terminal symbol, but all the other symbols
on the path are non-terminal symbols.  There are at least $N$ such
non-terminal symbols on the path.  Since the number of different
non-terminal symbols is $|V|$ and since $N=|V|+1$, some non-terminal
symbol must occur twice on the path.  In fact, some non-terminal
symbol must occur twice among the bottommost $N$ non-terminal
symbols on the path.  Call this symbol $A$.  Then we see that
the parse tree for $w$ has the form shown here:

\medskip
\centerline{\scaledeps{2truein}{fig-5-6}}

\noindent The structure of this tree breaks the string $w$ into
five substrings, as shown in the above diagram.
We then have $w=uxyzv$.  It only remains to show that $x$,
$y$, and $z$ satisfy the three requirements stated in the 
theorem.

Let $T$ refer to the entire parse tree, let $T_1$ refer to the 
parse tree whose root is the upper $A$ in the diagram, and 
let $T_2$ be the parse tree whose root is the lower $A$ in
the diagram.  Note that the height of $T_1$ is less than 
or equal to $N$.  (This follows from two facts:  The path shown in
$T_1$ from its root to its base has length less than or equal to
$N$, because we chose the two occurrences of $A$ to be among the $N$
bottommost non-terminal symbols along the path in $T$ from its
root to its base.  We know that there is no longer path from
the root of $T_1$ to its base, since we chose the path in $T$
to be the longest possible path from the root of $T$ to its base.)
Since any parse tree with height less than or equal to $N$ has
yield of length less than $K$, we see that $|xyz|<K$.

If we remove $T_1$ from $T$ and replace it with
a copy of $T_2$, the result is a parse tree with
yield $uyv$, so we see that the string $uyv$ is in the language
$L$.  
Now, suppose that both $x$ and $z$ are equal to the
empty string.  In that case, $w=uyv$, so the tree we have
created would be another parse tree for $w$.  But this tree
is smaller than $T$, so this would contradict the fact that
$T$ is the smallest parse tree for $w$.  We see that
$x$ and $z$ cannot both be the empty string.

If we remove $T_2$ from $T$ and replace it with a copy
of $T_1$, the result is a parse tree with yield $ux^2yz^2v$,
so we see that $ux^2yz^2v\in L$.  The two parse trees that
we have created look like this:

\medskip
\centerline{\scaledeps{4truein}{fig-5-7}}
\smallskip

\noindent Furthermore, we can apply the process of replacing
$T_2$ with a copy of $T_1$ to the tree on the right above to
create a parse tree with yield $ux^3yz^3v$.  Continuing in this
way, we see that $ux^nyz^nv\in L$ for all $n\in\N$.
This completes the proof of the theorem.
\end{proof}


Since this theorem guarantees that all context-free languages
have a certain property, it can be used to show that specific
languages are not context-free.  The method is to show that
the language in question does not have the property that is
guaranteed by the theorem.  We give two examples.

\begin{corrolary}\label{T-CFG-anbncn}
Let $L$ be the language $\{a^nb^nc^n \st n\in\N\}$.  Then
$L$ is not a context-free language.
\end{corrolary}
\begin{proof}
We give a proof by contradiction.  Suppose that $L$ is
context-free.  Then, by the Pumping Lemma for Context-free Languages,
there is an integer $K$ such that every string $w\in L$ with 
$|w|\ge K$ can be written in the form $w=uxyzv$ where
$x$ and $z$ are not both empty, $|xyz|<K$, and $ux^nyz^nv\in L$
for every $n\in\N$.

Consider the string $w=a^Kb^Kc^K$, which is in $L$,
and write $w=uxyzv$, where $u$, $x$, $y$, $z$, and $v$ satisfy the
stated conditions.  Since $|xyz|<K$, we see that if $xyz$
contains an $a$, then it cannot contain a $c$.  And if it
contains a $c$, then it cannot contain an $a$.  It is also possible
that $xyz$ is made up entirely of $b\text{'}s$.  In any of these cases,
the string $ux^2yz^2v$ cannot be in $L$, since it does not contain
equal numbers of $a\text{'}s$, $b\text{'}s$, and $c\text{'}s$.  But this contradicts
the fact that $ux^nyz^nv\in L$ for all $n\in\N$.  This contradiction
shows that the assumption that $L$ is context-free is incorrect.
\end{proof}
\smallskip

\begin{corrolary}
Let $\Sigma$ be any alphabet that contains at least two symbols.
Let $L$ be the language over $\Sigma$ defined by $L=\{ss\st s\in\Sigma^*\}$.
Then $L$ is not context-free.
\end{corrolary}
\begin{proof}
Suppose, for the sake of contradiction, that $L$ is
context-free.  Then, by the Pumping Lemma for Context-free Languages,
there is an integer $K$ such that every string $w\in L$ with 
$|w|\ge K$ can be written in the form $w=uxyzv$ where
$x$ and $z$ are not both empty, $|xyz|<K$, and $ux^nyz^nv\in L$
for every $n\in\N$.

Let $a$ and $b$ represent distinct symbols in $\Sigma$.
Let $s=a^Kba^Kb$ and let $w=ss=a^Kba^Kba^Kba^Kb$, which is in $L$.
Write $w=uxyzv$, where $u$, $x$, $y$, $z$, and $v$ satisfy the
stated conditions.

Since $|xyz|<K$, $x$ and $z$ can, together, contain no more than one $b$.  If
either $x$ or $y$ contains a $b$, then $ux^2yz^2v$ contains exactly five
$b\text{'}s$.  But any string in $L$ is of the form $rr$ for some string $r$
and so contains an even number of $b\text{'}s$.  The fact that
$ux^2yz^2z$ contains five $b\text{'}s$ contradicts the fact that $ux^2yz^2v\in L$.
So, we get a contradiction in the case where $x$ or $y$ contains a $b$.

Now, consider the case where $x$ and $y$ consist entirely of $a\text{'}s$.
Again since $|xyz|<K$, we must have either that $x$ and $y$ are both
contained in the same group of $a\text{'}s$ in the string $a^Kba^Kba^Kba^Kb$,
or that $x$ is contained in one group of $a\text{'}s$ and $y$ is contained in
the next.  In either case, it is easy to check that the string
$ux^2yz^2v$ is no longer of the form $rr$ for any string $r$,
which contradicts the fact that $ux^2yz^2v\in L$.

Since we are led to a contradiction in every case, we see that the
assumption that $L$ is context-free must be incorrect.
\end{proof}


Now that we have some examples of languages that are not context-free,
we can settle some other questions about context-free languages.
In particular, we can show that the intersection of two context-free
languages is not necessarily context-free and that the complement of
a context-free language is not necessarily context-free.

\begin{theorem}
The intersection of two context-free languages is not necessarily a
context-free language.
\end{theorem}
\begin{proof}
To prove this, it is only necessary to produce an example of two
context-free languages $L$ and $M$ such that $L\cap M$ is not a
context-free languages.  Consider the following languages, defined
over the alphabet $\Sigma=\{a,b,c\}$:
\begin{align*}
   L&=\{a^nb^nc^m \st n\in\N\text{ and }m\in\N\}\\
   M&=\{a^nb^mc^m \st n\in\N\text{ and }m\in\N\}
\end{align*}
Note that strings in $L$ have equal numbers of $a\text{'}s$ and $b\text{'}s$ while
strings in $M$ have equal numbers of $b\text{'}s$ and $c\text{'}s$.  It follows that
strings in $L\cap M$ have equal numbers of $a\text{'}s$, $b\text{'}s$, and $c\text{'}s$.
That is,
\begin{align*}
   L\cap M=\{a^nb^nc^n\st n\in\N\}
\end{align*}
We know from the above theorem that $L\cap M$ is not context-free.
However, both $L$ and $M$ are context-free.  The language $L$
is generated by the context-free grammar
\begin{align*}
   S &\PRODUCES TC\\
   C &\PRODUCES cC\\
   C &\PRODUCES \EMPTYSTRING\\
   T &\PRODUCES aTb\\
   T &\PRODUCES \EMPTYSTRING
\end{align*}
and $M$ is generated by a similar context-free grammar.
\end{proof}
\smallskip

\begin{corrolary}
The complement of a context-free language is not necessarily context-free.
\end{corrolary}
\begin{proof}
  Suppose for the sake of contradiction that the complement of every
context-free language is context-free.  

Let $L$ and $M$ be two context-free languages over the alphabet $\Sigma$.
By our assumption, the complements $\overline{L}$ and $\overline{M}$
are context-free.  By Theorem~\ref{T-CFG-closures}, it follows
that $\overline{L}\cup\overline{M}$ is context-free.  Applying our
assumption once again, we have that $\overline{\overline{L}\cup\overline{M}}$
is context-free.
But $\overline{\overline{L}\cup\overline{M}}=L\cap M$, so we have
that $L\cap M$ is context-free.

We have shown, based on our assumption that the complement of any
context-free language is context-free, that the intersection of any
two context-free languages is context-free.  But this contradicts
the previous theorem, so we see that the assumption cannot be true.
This proves the theorem.
\end{proof}

Note that the preceding theorem and corollary say only that
$L\cap M$ is not context-free for \textit{some} context-free languages
$L$ and $M$ and that $\overline{L}$ is not context-free for 
\textit{some} context-free language $L$.  There are, of course, many 
examples of context-free languages $L$ and $M$ for which
$L\cap M$ and $\overline{L}$ \textit{are} in fact context-free.

Even though the intersection of two context-free languages is not necessarily
context-free, it happens that the intersection of a context-free language
with a \textit{regular} language is always context-free.  This is not difficult
to show, and a proof is outlined in Exercise~4.4.8.
I state it here without proof:

\begin{theorem}
Suppose that $L$ is a context-free language and that $M$ is a regular
language.  Then $L\cap M$ is a context-free language.
\end{theorem}

For example, let $L$ and $M$ be the languages
defined by $L=\{w\in \{a,b\}^*\st w=w^R\}$ and
$M=\{w\in\{a,b\}^*\st$ the length of $w$ is a multiple of 5$\}$.  Since 
$L$ is context-free and $M$ is regular, we know that
$L\cap M$ is context-free.  The language $L\cap M$ consists of
every palindrome over the alphabet $\{a,b\}$ whose length is a
multiple of five.

This theorem can also be used to show that certain languages are
\textit{not} context-free.  For example, consider the language
$L=\{w\in\{a,b,c\}^*\st n_a(w)$ $=n_b(w)=n_c(w)\}$.  (Recall that
$n_x(w)$ is the number of times that the symbol $x$ occurs in the
string $w$.)  We can use a proof by contradiction to show that
$L$ is not context-free. 
Let $M$ be the regular language defined by the regular
expression $a^*b^*c^*$.  It is clear that
$L\cap M=\{a^nb^nc^n\st n\in\N\}$.  If $L$ were context-free,
then, by the previous theorem, $L\cap M$ would be context-free.
However, we know from Theorem~\ref{T-CFG-anbncn} that $L\cap M$
is not context-free.  So we can conclude that $L$ is not
context-free.


\begin{exercises}

\problem Show that the following languages are not context-free:
\ppart $\{a^nb^mc^k\st n>m>k\}$
\ppart $\{w\in\{a,b,c\}^*\st n_a(w)>n_b(w)>n_c(w)\}$
\ppart $\{www\st w\in\{a,b\}^*\}$
\ppart $\{a^nb^mc^k\st n,m\in\N\text{ and } k=m*n\}$
\ppart $\{a^nb^m\st m=n^2\}$

\problem Show that the languages $\{a^n\st \,\text{n is a prime number}\}$
and $\{a^{n^2}\st n\in \N\}$ are not context-free.  (In fact, it 
can be shown that a language over the alphabet $\{a\}$ is
context-free if and only if it is regular.)

\problem Show that the language $\{w\in\{a,b\}^*\st n_a(w)=n_b(w)$
and $w$ contains the string $baaab$ as a substring$\}$ is context-free.

\problem Suppose that $M$ is any finite language and that
$L$ is any context-free language.  Show that the language
$L\SETDIFF M$ is context-free.  (Hint: Any finite language is a
regular language.)

\end{exercises}




