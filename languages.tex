\section{Languages}
In formal language theory, an \nw{alphabet} is a finite, non-empty set.
The elements of the set are called \nw{symbols}.
A finite sequence of symbols $a_1a_2\ldots a_n$ from an alphabet is called a \nw{string} over that alphabet.  

\smallskip

\begin{example}
$\Sigma = \{0,1\}$ is an alphabet, and {\em 011}, 
{\em 1010}, and {\em 1} are all strings over $\Sigma$.
\end{example}

\smallskip

Note that strings really are \emph{sequences} of symbols, which implies that order matters.
Thus {\em 011}, {\em 101}, and {\em 110} are all different strings, though they are made up of the same symbols.
The strings $x=\aetc{a}{n}$ and $y=\aetc{b}{m}$ are \nw{equal} only if $m=n$ (i.e.\ the strings contain the same number of symbols) and $a_i=b_i$ for all $1 \leq i \leq n$.

Just as there are operations defined on numbers, truth values, sets, and other mathematical elements, there are operations defined on strings.
Some important operations are:
\begin{enumerate}
\item {\em length}: the \nw{length} of a string $x$ is the number of symbols in it.
The notation for the length of $x$ is $|x|$.
Note that this is consistent with other uses of $|\ |$, all of which involve some notion of size: $|number|$ measures how big a number is (in terms of its distance from 0);  $|set|$ measures the size of a set (in terms of the number of elements).

A {\em length-$n$ string} is simply ``a string whose length is $n$".

\item {\em concatenation}: the \nw{concatenation} of two strings $x=a_1 a_2\ldots a_m$ and $y=b_1b_2\ldots b_n$ is the sequence of symbols $a_1\ldots a_mb_1\ldots b_n$.
Sometimes $\cdot$ is used to denote concatenation, but it is far more usual to see the concatenation of $x$ and $y$ denoted by $xy$ than by $x\cdot y$.
You can easily convince yourself that concatenation is associative (i.e.\ $(xy)z = x(yz)$ for all strings $x,y$ and $z$.)
Concatenation is not commutative (i.e.\ it is not always true that $xy = yx$:
for example, if $x=a$ and $y=b$ then $xy=ab$ while $yx=ba$ and, as discussed above, these strings are not equal.)

\item {\em reversal}: the \nw{reverse} of a string $x=a_1a_2\ldots a_n$ is the string $x^R = a_na_{n-1}\ldots a_2a_1$.
\end{enumerate}

\begin{example}
Let $\Sigma = \ab$, $x=a$, $y=abaa$, and $z=bab$.
Then $|x| = 1$, $|y| = 4$, and $|z|=3$.  Also, $xx = aa$, $xy =
aabaa$, $xz = abab$, and $zx = baba$.  Finally, $x^R = a$,
$y^R = aaba$, and $z^R=bab$.
\end{example}

\smallskip

By the way, the previous example illustrates a naming convention standard
throughout language theory texts: if a letter is
intended to represent a single symbol in an alphabet, the convention
is to use a letter from the beginning of the English alphabet ({\em a,
b, c, d }); if a letter is intended to represent a string, the 
convention is to use a letter from the end of the English alphabet
({\em u, v, } etc).

\bigskip

In set theory, we have a special symbol to designate the set that 
contains no elements.  Similarly, language theory has a special 
symbol $\varepsilon$ which is used to represent the \nw{empty string}, the
string with no 
symbols in it.  (Some texts use the symbol $\lambda$ instead.)
It is worth noting that $|\varep| = 0$, that $\varep^R = \varep$,
and that $\varep \cdot x = x \cdot \varep = x$ for all strings $x$.
(This last fact may appear a bit confusing.  Remember that $\varep$
is not a symbol in a string with length 1, but rather the name given
to the string made up of 0 symbols.  Pasting those 0 symbols onto the
front or back of a string $x$ still produces $x$.) 

\bigskip

The set of all strings over an alphabet $\Sigma$ is denoted $\Sigma^*$.
(In language theory, the symbol $^*$ is typically used to denote ``zero
or more'', so $\Sigma^*$ is the set of strings made up of zero or 
more symbols from $\Sigma$.)  Note that while an alphabet 
$\Sigma$ is by 
definition a \emph{finite} set of symbols, and strings are by
definition \emph{finite} sequences of those symbols, the set $\Sigma^*$
is \emph{always infinite}.  Why is this?  Suppose $\Sigma$ contains $n$
elements.  Then there is one string over $\Sigma$ with 0 symbols,
$n$ strings with 1 symbol, $n^2$ strings with 2 symbols (since there
are $n$ choices for the first symbol and $n$ choices for the second),
$n^3$ strings with 3 symbols, etc.

\smallskip

\begin{example} If $\Sigma = \{1\}$, then $\Sigma^* = \{\varep,
1, 11, 111, \ldots\}$.  If $\Sigma = \ab$, then $\Sigma^* = \{
\varep, a, b, aa, ab, ba, bb, aaa, aab, \ldots\}$.
\end{example}

\smallskip

Note that $\Sigma^*$ is \emph{countably} infinite: if we list the strings as in
the preceding example (length-0 strings, length-1 strings in ``alphabetical"
order, length-2 strings similarly ordered, etc) then any string over $\Sigma$
will eventually appear.  (In fact, if $|\Sigma| = n \geq 2$ and $x \in \Sigma^*$ has
length $k$, then $x$ will appear on the list within the first $\frac{n^{k+1} -
1}{n-1}$ entries.)

\bigskip

We now come to the definition of a \nw{language} in the formal language
theoretical sense.


\begin{definition} A language over an alphabet $\Sigma$ is a subset
of $\Sigma^*$.  Thus, a language over $\Sigma$ is an element of
${\cal P}(\Sigma^*)$, the power set of $\Sigma^*$.
\end{definition}

\smallskip
In other words, any set of strings (over alphabet $\Sigma$) constitutes a
language (over alphabet $\Sigma$).

\smallskip

\begin{example} Let $\Sigma = \{0,1\}$.  Then the following are all
languages over $\Sigma$:

$L_1 = \{011, 1010, 111\}$

$L_2 = \{0, 10, 110, 1110, 11110, \ldots\}$

$L_3 = \{x \in \Sigma^* \ | \ n_0(x) = n_1(x) \}$, where the notation 
\nw{$n_0(x)$}
stands for the 

\ \ \ number of 0's in the string $x$, and similarly for $n_1(x)$.

$L_4 = \{x \ | \ \mbox{\ $x$ represents a multiple of 5 in binary}\}$
\end{example}

\smallskip

Note that languages can be either finite or infinite.
Because $\Sigma^*$ is infinite, it clearly has an
infinite number of subsets, and so there are an infinite number of languages
over $\Sigma$.  But are there countably or uncountably many such languages?

\smallskip

\begin{theorem}
For any alphabet $\Sigma$, the number of languages over $\Sigma$ is
uncountable.
\end{theorem}

\smallskip
This fact is an immediate consequence of the result, proved in a previous
chapter, that the power set of a countably infinite set is uncountable.  Since
the elements of ${\cal P}(\Sigma)$ are exactly the languages over $\Sigma$,
there are uncountably many such languages.

\medskip

Languages are sets and therefore, as for any sets, it makes sense to talk about
the union, intersection, and complement of languages.  (When taking the complement
of a language over an alphabet $\Sigma$, we always consider the univeral set
to be $\Sigma^*$, the set of all strings over~$\Sigma$.)
Because languages are
sets of strings, there are additional operations that can be defined on
languages, operations that would be meaningless on more general sets.  For
example, the idea of concatenation can be extended from strings to languages.

For two sets of strings $S$ and $T$, we define the \nw{concatenation} of $S$ and
$T$ (denoted $S\cdot T$ 
or just $ST$) to
be the set $ST = \{ st \ | \ s \in S \AND t \in T \}$.  For example, if $S =
\{ab, aab\}$ and $T=\{\varep, 110, 1010\}$, then 
$ST = \{ab,  ab110,  ab1010,  aab,  aab110,  aab1010\}$.  
Note in particular that $ab \in ST$, because $ab \in S$, $\varep \in T$, and
$ab \cdot \varep = ab$.
Because 
concatenation of sets is defined in terms of the
concatenation of the
strings that the sets contain, concatenation of sets is associative
and not commutative.  (This can easily be verified.)  

When a set $S$
is concatenated with itself, the notation $SS$ is usually scrapped
in favour of $S^2$; if $S^2$ is concatenated with $S$, we write
$S^3$ for the resulting set, etc.  So $S^2$ is the set of all strings formed by
concatenating two (possibly different, possibly identical) strings from $S$,
$S^3$ is the set of strings formed by concatenating three strings from $S$,
etc.  Extending this notation, we take $S^1$ to be the set of strings formed
from one string in $S$ (i.e.\ $S^1$ is $S$ itself), and $S^0$ to be the set of
strings formed from zero strings in $S$ (i.e.\ $S^0 = \{\varep\}$).  If we take
the union $S^0 \cup S^1 \cup S^2 \cup \ldots$, then the resulting set is the set of
all strings formed by concatenating zero or more strings from $S$, and is
denoted $S^*$.  The set $S^*$ is called the \nw{Kleene closure} of $S$, and
the $^*$ operator is called the \nw{Kleene star} operator.

\smallskip

\begin{example}
Let $S = \{01, ba\}$.  Then

$S^0 = \{\varep\}$

$S^1 = \{01, ba\}$

$S^2 = \{0101, 01ba, ba01, baba\}$

$S^3 = \{010101, 0101ba, 01ba01, 01baba, ba0101, ba01ba, baba01, bababa\}$

etc, so

$S^* =\{\varep,01,ba,0101,01ba,ba01,baba,010101,0101ba,\ldots\}.$
\end{example}
 
\smallskip

Note that this is the second time we have seen the notation $something^*$.  We
have previously seen that for an alphabet $\Sigma$, $\Sigma^*$ is defined to be 
the set of all
strings over $\Sigma$.  If you think of $\Sigma$ as being a set of length-1
strings, and take its Kleene closure, the result is once again the set of all
strings over $\Sigma$, and so the two notions of $^*$ coincide.

\smallskip

\begin{example}
Let $\Sigma = \ab$.  Then

$\Sigma^0 = \{\varep\}$

$\Sigma^1 = \ab$

$\Sigma^2 = \{aa, ab, ba, bb\}$

$\Sigma^3 = \{aaa, aab, aba, abb, baa, bab, bba, bbb\}$

etc, so

$\Sigma^* =\{\varep,a,b,aa,ab,ba,bb,aaa,aab,aba,abb,baa,bab,\ldots\}.$
\end{example}

\begin{exercises}
\problem Let $S = \{\varep, ab, abab\}$ and $T = \{aa, aba, abba, abbba,
\ldots\}$.  Find the following:
\pparts{ S^2 & S^3 & S^* & ST & TS }
\problem The \nw{reverse} of a language $L$ is defined to be 
$L^R = \{ x^R \ | \ x \in L\}$.  Find $S^R$ and $T^R$ for the $S$ and $T$ in the
preceding problem.
\problem Give an example of a language $L$ such that $L=L^*$.

\end{exercises}



