\section{Application: Relational Databases}\label{S-sets-8}

One of the major uses of computer systems is to store and manipulate
collections of data.  A \nw{database} is a collection of data
that has been organized so that it is possible to add and delete
information, to update the data that it contains, and to
retrieve specified parts of the data.  A \nw{Database Management
System}, or DBMS\index{DBMS}, is a computer program that makes it possible
to create and manipulate databases.  A DBMS must be able to
accept and process commands that manipulate the data in the databases
that it manages.  These commands are called \nw[query, database]{queries},
and the languages in which they are written are called 
\nw[query language]{query languages}.  A query language is
a kind of specialized programming language.


\fig
  {F-DBMS}
  {Tables that could be part of a relational database.
   Each table has a name, shown above the table.
   Each column in the table also has a name, shown in the top row
   of the table.  The remaining rows hold the data.}
  {\textbf{Members}  
 
   \smallskip
   \begin{tabular}{|l|l|l|l|}
        \hline
        \strut \textbf{MemberID}& \textbf{Name}& \textbf{Address} &\textbf{City}\\
        \hline
        \strut 1782& Smith, John& 107 Main St& New York, NY \\
        2889&  Jones, Mary& 1515 Center Ave& New York, NY\\
        378& Lee, Joseph& 90 Park Ave& New York, NY\\
        4277& Smith, John& 2390 River St& Newark, NJ\\
        5704& O'Neil, Sally& 89 Main St& New York, NY\\
        \hline
     \end{tabular}
     
   \bigskip
   \textbf{Books}
   
   \smallskip
   \begin{tabular}{|l|l|l|}
        \hline
        \strut \textbf{BookID}& \textbf{Title}& \textbf{Author}\\
        \hline
        \strut 182& I, Robot& Isaac Asimov\\
        221& The Sound and the Fury& William Faulkner\\
        38&  Summer Lightning& P.G. Wodehouse\\
        437& Pride and Prejudice& Jane Austen\\
        598& Left Hand of Darkness& Ursula LeGuin\\
        629& Foundation Trilogy& Isaac Asimov\\
        720& Mirror Dance& Lois McMaster Bujold\\
        \hline
     \end{tabular}
     
   \bigskip
   \textbf{Loans}
   
   \smallskip
   \begin{tabular}{|l|l|l|}
        \hline
        \strut \textbf{MemberID}& \textbf{BookID}& \textbf{DueDate}\\
        \hline
        \strut 378&  221& October 8, 2010\\
        2889& 182& November 1, 2010\\
        4277& 221& November 1, 2010\\
        1782& 38&  October 30, 2010\\
        \hline
     \end{tabular}
   }

There are many different ways that the data in a database could
be represented.  Different DBMS's use various data representations
and various query languages.  However, data is most commonly stored
in relations.  A relation\index{relation} in a database is a relation
in the mathematical sense.
That is, it is a subset of a cross product of sets.  A database
that stores its data in relations is called a \nw{relational
database}.  The query language for most relational database management
systems is some form of the language known as \nw{Structured Query
Language}, or SQL.\index{SQL}  In this section, we'll take a very brief look
at SQL, relational databases, and how they use relations.

\medbreak

A relation is just a subset of a cross product of sets.  Since we
are discussing computer representation of data, the sets are
data types.  As in Section~\ref{S-sets-5}, we'll use data
type names such as \textit{int} and \textit{string} to refer to
these sets.  A relation that is a subset of the
cross product $\textit{int}\times\textit{int}\times\textit{string}$
would consist of ordered 3-tuples such as 
(17, 42, ``hike'').  In a relational database, the data is stored in
the form of one or more such relations.  The relations are called
tables, and the tuples that they contain are called rows or records.

As an example, consider a lending library that wants to
store data about its members, the books that it owns, and
which books the members have out on loan.
This data could be represented in three tables, as illustrated in
Figure~\ref{F-DBMS}.  The relations are shown as tables rather than
as sets of ordered tuples, but each table is, in fact, a relation.
The rows of the table are the tuples.  The \texttt{Members} table,
for example, is a subset of $\textit{int}\times\textit{string}\times\textit{string}\times\textit{string}$,
and one of the tuples is (1782, ``Smith, John'', ``107 Main St'', ``New York, NY'').
A table does have one thing that ordinary relations in mathematics
do not have.  Each column in the table has a name.  These names
are used in the query language to manipulate the data in the tables.

The data in the \texttt{Members} table is the basic information that
the library needs in order to keep track of its members, namely the name and
address of each member.  A member also has a \texttt{MemberID} number,
which is presumably assigned by the library.  Two different members
can't have the same \texttt{MemberID}, even though they might
have the same name or the same address.  The \texttt{MemberID}
acts as a \nw{primary key} for the \texttt{Members} table.
A given value of the primary key uniquely identifies one of
the rows of the table.  Similarly, the \texttt{BookID} in
the \texttt{Books} table is a primary key for that table.
In the \texttt{Loans} table, which holds information about which
books are out on loan to which members, a \texttt{MemberID} 
unambiguously identifies the member who has a given book on loan,
and the \texttt{BookID} says unambiguously which book that is.
Every table has a primary key, but the key can consist of more
than one column.  The DBMS enforces the uniqueness
of primary keys.  That is, it won't let users make a modification
to the table if it would result in two rows having the same
primary key.

The fact that a relation is a set---a set of tuples---means that
it can't contain the same tuple more than once.  In terms of tables, 
this means that a table shouldn't contain two identical rows.  But
since no two rows can contain the same primary key, it's
impossible for two rows to be identical.  So tables are in fact
relations in the mathematical sense.

\medbreak

The library must have a way to add and delete members and books
and to make a record when a book is borrowed or returned.
It should also have a way to change the address of a member
or the due date of a borrowed book.  Operations such as
these are performed using the DBMS's query language.
SQL has commands named \texttt{INSERT}, \texttt{DELETE},
and \texttt{UPDATE} for performing these operations.
The command for adding Barack Obama as a member of the
library with \texttt{MemberID} 999 would be
\begin{verbatim}
       INSERT INTO Members
       VALUES (999, "Barack Obama",
                 "1600 Pennsylvania Ave", "Washington, DC")
\end{verbatim}
When it comes to deleting and modifying rows, things become
more interesting because it's necessary to specify which
row or rows will be affected.  This is done by specifying
a condition that the rows must fulfill.  For example,
this command will delete the member with ID 4277:
\begin{verbatim}
       DELETE FROM Members
       WHERE MemberID = 4277
\end{verbatim}
It's possible for a command to affect multiple rows.  For
example,
\begin{verbatim}
       DELETE FROM Members
       WHERE Name = "Smith, John"
\end{verbatim}
would delete every row in which the name is ``Smith, John.''
The update command also specifies what changes are to be
made to the row:
\begin{verbatim}
       UPDATE Members
       SET Address="19 South St", City="Hartford, CT"
       WHERE MemberID = 4277
\end{verbatim}

Of course, the library also needs a way of retrieving
information from the database.  SQL provides the
\texttt{SELECT} command for this purpose.  For example,
the query
\begin{verbatim}
       SELECT Name, Address
       FROM Members
       WHERE City = "New York, NY"
\end{verbatim}
asks for the name and address of every member who lives in 
New York City.  The last line of the query is a condition
that picks out certain rows of the ``Members'' relation,
namely all the rows in which the \texttt{City} is
``New York, NY''.  The first line specifies which data
from those rows should be retrieved.  The data is actually
returned in the form of a table.  For example, given the
data in Figure~\ref{F-DBMS}, the query would return this
table:
\begin{center}
\begin{tabular}{|l|l|}
        \hline
        \strut Smith, John& 107 Main St\\
        Jones, Mary& 1515 Center Ave\\
        Lee, Joseph& 90 Park Ave\\
        O'Neil, Sally& 89 Main St\\
        \hline
\end{tabular}
\end{center}
The table returned by a \texttt{SELECT} query can even be used
to construct more complex queries.  For example, if the table returned 
by \texttt{SELECT} has only one column, then it can be
used with the \texttt{IN} operator to specify any value
listed in that column.  The following query will find the
\texttt{BookID} of every book that is out on loan to a
member who lives in New York City:
\begin{verbatim}
       SELECT BookID
       FROM Loans
       WHERE MemberID IN (SELECT MemberID
                          FROM Members
                          WHERE City = "New York, NY")
\end{verbatim}

More than one table can be listed in the \texttt{FROM}
part of a query.  The tables that are listed are joined
into one large table, which is then used for the query.
The large table is essentially the cross product of
the joined tables, when the tables are understood as
sets of tuples.  For example, suppose that we want the
titles of all the books that are out on loan to members who
live in New York City.  The titles are in the \texttt{Books}
table, while information about loans is in the \texttt{Loans}
table.  To get the desired data, we can join the tables
and extract the answer from the joined table:
\begin{verbatim}
       SELECT Title
       FROM Books, Loans
       WHERE MemberID IN (SELECT MemberID
                          FROM Members
                          WHERE City = "New York, NY")
\end{verbatim}
In fact, we can do the same query without using the nested
\texttt{SELECT}.  We need one more bit of notation:  If two
tables have columns that have the same name, the columns can
be named unambiguously by combining the table name with the
column name.  For example, if the \texttt{Members} table
and \texttt{Loans} table are both under discussion, then
the \texttt{MemberID} columns in the two tables can be referred
to as \texttt{Members}.\texttt{MemberID} and 
\texttt{Loans}.\texttt{MemberID}.  So, we can say:
\begin{verbatim}
       SELECT Title
       FROM Books, Loans
       WHERE City ="New York, NY"
             AND Members.MemberID = Loans.MemberID
\end{verbatim}

This is just a sample of what can be done with SQL and
relational databases.  The conditions in \texttt{WHERE}
clauses can get very complicated, and there are other
operations besides the cross product for combining tables.
The database operations that are needed to complete a
given query can be complex and time-consuming.  Before
carrying out a query, the DBMS tries to optimize it.
That is, it manipulates the query into a form that
can be carried out most efficiently.  The rules for
manipulating and simplifying queries form an \emph{algebra}\index{algebra}
of relations, and the theoretical study of relational
databases is in large part the study of the algebra
of relations.

\begin{exercises}

\problem Using the library database from Figure~\ref{F-DBMS},
what is the result of each of the following SQL commands?
\begin{verbatim}
   a)   SELECT Name, Address
        FROM Members
        WHERE Name = "Smith, John"
        
   b)   DELETE FROM Books
        WHERE Author = "Isaac Asimov"

   c)   UPDATE Loans
        SET DueDate = "November 20"
        WHERE BookID = 221

   d)   SELECT Title
        FROM Books, Loans
        WHERE Books.BookID = Loans.BookID
        
   e)   DELETE FROM Loans
        WHERE MemberID IN (SELECT MemberID
                           FROM Members
                           WHERE Name = "Lee, Joseph")
\end{verbatim}

\problem Using the library database from Figure~\ref{F-DBMS},
write an SQL command to do each of the following database
manipulations:
\ppart Find the \texttt{BookID} of every book that is due on
November 1, 2010.
\ppart Change the \texttt{DueDate} of the book with \texttt{BookID} 221
to November 15, 2010.
\ppart Change the \texttt{DueDate} of the book with title
``Summer Lightning'' to November 14, 2010.  Use a nested
\texttt{SELECT}.
\ppart Find the name of every member who has a book out on loan.
Use joined tables in the \texttt{FROM} clause of a \texttt{SELECT}
command.

\problem Suppose that a college wants to use a database to store
information about its students, the courses that are offered in
a given term, and which students are taking which courses.
Design tables that could be used in a relational
database for representing this data.  Then
write SQL commands to do each of the following database
manipulations.  (You should design your tables so that they
can support all these commands.)
\ppart Enroll the student with ID number 1928882900 in ``English 260''.
\ppart Remove ``John Smith'' from ``Biology 110''.
\ppart Remove the student with ID number 2099299001 from every course
in which that student is enrolled.
\ppart Find the names and addresses of the students who are taking ``Computer Science 229''.
\ppart Cancel the course ``History 101''.


\end{exercises}



