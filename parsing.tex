\section{Parsing and Parse Trees}\label{S-grammars-3}

Suppose that $G$ is a grammar for the language $L$.  That is, 
$L=L(G)$.  The grammar $G$ can be used to generate strings in
the language~$L$.  In practice, though, we often start with a string
which might or might not be in~$L$, and the problem is
to determine whether the string is in the language and, if so,
how it can be generated by~$G$.  The goal is to find a derivation
of the string, using the production rules of the grammar, or to
show that no such derivation exists.  This is known as \nw{parsing}
the string.  When the string is a computer program or a sentence
in a natural language, parsing the string is an essential step
in determining its meaning.

As an example that we will use throughout
this section, consider the language that consists of arithmetic
expressions containing parentheses, the binary operators $+$ and $*$,
and the variables $x$, $y$, and $z$.  Strings in this language
include $x$, $x+y*z$, and $((x+y)*y)+z*z$.  Here is a context-free
grammar that generates this language:
\begin{align*}
   E&\PRODUCES E+E\\
   E&\PRODUCES E*E\\
   E&\PRODUCES (E)\\
   E&\PRODUCES x\\
   E&\PRODUCES y\\
   E&\PRODUCES z
\end{align*}
Call the grammar described by these production rules $G_1$.
The grammar $G_1$ says that $x$, $y$, and $z$ are expressions, and that
you can make new expressions by adding two expressions, by multiplying
two expressions, and by enclosing an expression in parentheses.
(Later, we'll look at other grammars for the same language---ones that
turn out to have certain advantages over $G_1$.)

Consider the string $x+y*z$.  To show that this string is in the
language $L(G_1)$, we can exhibit a derivation of the string
from the start symbol $E$.  For example:
\begin{align*}
   E &\YIELDS E+E\\
     &\YIELDS E+E*E\\
     &\YIELDS E+y*E\\
     &\YIELDS x+y*E\\
     &\YIELDS x+y*z
\end{align*}
This derivation shows that the string $x+y*z$ is in fact in $L(G_1)$.
Now, this string has many other derivations.  At each step in the
derivation, there can be a lot of freedom about which rule in the
grammar to apply next.  Some of this freedom is clearly not very
meaningful.  When faced with the string $E+E*E$ in the above example,
the order in which we replace the $E\text{'}s$ with the variables $x$, $y$,
and $z$ doesn't much matter.  To cut out some of this meaningless
freedom, we could agree that in each step of a derivation, the
non-terminal symbol that is replaced is the leftmost non-terminal
symbol in the string.  A derivation in which this is true is
called a \nw{left derivation}.  The following left derivation
of the string $x+y*z$ uses the same production rules as the previous
derivation, but it applies them in a different order:
\begin{align*}
   E &\YIELDS E+E\\
     &\YIELDS x+E\\
     &\YIELDS x+E*E\\
     &\YIELDS x+y*E\\
     &\YIELDS x+y*z
\end{align*}
It shouldn't be too hard to convince yourself that any string that
has a derivation has a left derivation (which can be obtained
by changing the order in which production rules are applied).

We have seen that the same string might have several different derivations.
We might ask whether it can have several different left derivations.
The answer is that it depends on the grammar.  A context-free
grammar $G$ is said to be \nw[ambiguous grammar]{ambiguous}\index{grammar!ambiguous}
if there is a string $w\in L(G)$ such that $w$ has more than one
left derivation according to the grammar $G$.

Our example grammar $G_1$ is ambiguous.  In fact, in addition to the
left derivation given above, the string $x+y*z$ has the alternative
left derivation
\begin{align*}
   E &\YIELDS E*E\\
     &\YIELDS E+E*E\\
     &\YIELDS x+E*E\\
     &\YIELDS x+y*E\\
     &\YIELDS x+y*z
\end{align*}
In this left derivation of the string $x+y*z$, the first production
rule that is applied is $E\PRODUCES E*E$.  The first $E$ on the right-hand
side eventually yields ``$x+y$'' while the second yields ``$z$''.
In the previous left derivation, the first production rule that was
applied was $E\PRODUCES E+E$, with the first $E$ on the right yielding
``$x$'' and the second $E$ yielding ``$y*z$''.  If we think in terms
of arithmetic expressions, the two left derivations lead to
two different interpretations of the expression $x+y*z$.  In one
interpretation, the $x+y$ is a unit that is multiplied by $z$.
In the second interpretation, the $y*z$ is a unit that is added to $x$.
The second interpretation is the one that is correct according to
the usual rules of arithmetic.  However, the grammar allows either
interpretation.  The ambiguity of the grammar allows the string to
be parsed in two essentially different ways, and only one of the
parsings is consistent with the meaning of the string.  Of course,
the grammar for English is also ambiguous.  In a famous example,
it's impossible to tell whether a ``pretty girls' camp'' is
meant to describe a pretty camp for girls or a camp for pretty girls.

When dealing with artificial languages such as programming languages,
it's better to avoid ambiguity.
The grammar $G_1$ is perfectly correct in that it generates the correct
set of strings, but in a practical situation where we are interested
in the meaning of the strings, $G_1$ is not the right grammar for
the job.  There are other grammars that generate the same language
as $G_1$.  Some of them are unambiguous grammars that better reflect
the meaning of the strings in the language.  For example, the
language $L(G_1)$ is also generated by the BNF grammar
\begin{align*}
   E\ &\BNFPRODUCES\ T\ [\ +\ T\ ]\dots\\
   T\ &\BNFPRODUCES\ F\ [\ *\ F\ ]\dots\\
   F\ &\BNFPRODUCES\ \text{``(''}\ E\ \text{``)''}\ \BNFALT\ x\ \BNFALT\ y\ \BNFALT\ z
\end{align*} 
This grammar can be translated into a standard context-free grammar, which
I will call $G_2$:
\begin{align*}
   E &\PRODUCES TA\\
   A &\PRODUCES +TA\\
   A &\PRODUCES \EMPTYSTRING\\
   T &\PRODUCES FB\\
   B &\PRODUCES *FB\\
   B &\PRODUCES \EMPTYSTRING\\
   F &\PRODUCES (E)\\
   F &\PRODUCES x\\
   F &\PRODUCES y\\
   F &\PRODUCES z
\end{align*}
The language generated by
$G_2$ consists of all legal arithmetic expressions made up of 
parentheses, the operators $+$ and $-$, and the variables $x$, $y$,
and $z$.  That is, $L(G_2)=L(G_1)$.  However, $G_2$ is an unambiguous
grammar.  Consider, for example, the string $x+y*z$.  Using the
grammar $G_2$, the only left derivation for this string is:
\begin{align*}
   E &\YIELDS TA\\
     &\YIELDS FBA\\
     &\YIELDS xBA\\
     &\YIELDS xA\\
     &\YIELDS x+TA\\
     &\YIELDS x+FBA\\
     &\YIELDS x+yBA\\
     &\YIELDS x+y*FBA\\
     &\YIELDS x+y*zBA\\
     &\YIELDS x+y*zA\\
     &\YIELDS x+y*z
\end{align*}
There is no choice about the first step in this derivation, since the
only production rule with $E$ on the left-hand side is $E\PRODUCES TA$.
Similarly, the second step is forced by the fact that there is only
one rule for rewriting a $T$.  In the third step, we must replace
an $F$.  There are four ways to rewrite $F$, but only one way to produce
the $x$ that begins the string $x+y*z$, so we apply the rule $F\PRODUCES x$.
Now, we have to decide what to do with the $B$ in $xBA$.  There two rules
for rewriting $B$, $B\PRODUCES *FB$ and $B\PRODUCES\EMPTYSTRING$.  However,
the first of these rules introduces a non-terminal, $*$, which does not
match the string we are trying to parse.  So, the only choice is to
apply the production rule $B\PRODUCES\EMPTYSTRING$.  In the next step
of the derivation, we must apply the rule $A\PRODUCES +TA$ in order to 
account for the $+$ in the string $x+y*z$.  Similarly, each of the 
remaining steps in the left derivation is forced.

\medbreak

The fact that $G_2$ is an unambiguous grammar means that at each
step in a left derivation for a string $w$, there is only one production
rule that can be applied which will lead ultimately to a correct
derivation of~$w$.  However, $G_2$ actually satisfies a much stronger
property:  at each step in the left derivation of $w$, we can tell which
production rule has to be applied by looking ahead at the next
symbol in~$w$.  We say that $G_2$ is an \nw{LL(1) grammar}.
(This notation means that we can read a string from \textbf{L}eft to
right and construct a \textbf{L}eft derivation of the string by
looking ahead at most \textbf{1} character in the string.)
Given an LL(1) grammar for a language, it is fairly straightforward
to write a computer program that can parse strings in that language.
If the language is a programming language, then parsing is one of the
essential steps in translating a computer program into machine language.
LL(1) grammars and parsing programs that use them are often studied
in courses in programming languages and the theory of compilers.

Not every unambiguous context-free grammar is an LL(1) grammar.  Consider, for
example, the following grammar, which I will call $G_3$:
\begin{align*}
   E &\PRODUCES E + T\\
   E &\PRODUCES T\\
   T &\PRODUCES T*F\\
   T &\PRODUCES F\\
   F &\PRODUCES (E)\\
   F &\PRODUCES x\\
   F &\PRODUCES y\\
   F &\PRODUCES z
\end{align*}
This grammar generates the same language as $G_1$ and $G_2$,
and it is unambiguous.  However, it is not possible to construct
a left derivation for a string according to the grammar $G_3$ by
looking ahead one character in the string at each step.  
The first step in any left derivation must be either
$E\YIELDS E+T$ or $E\YIELDS T$.  But how can we decide which of
these is the correct first step?
Consider the strings $(x+y)*z$ and $(x+y)*z+z*x$, which are both
in the language $L(G_3)$.  For the string $(x+y)*z$, the
first step in a left derivation must be $E\YIELDS T$, while 
the first step in a left derivation of $(x+y)*z+z*x$ must be
$E\YIELDS E+T$.  However, the first seven characters of the strings
are identical, so clearly looking even seven characters ahead is not
enough to tell us which production rule to apply.  In fact,
similar examples show that looking ahead any given finite number of
characters is not enough.

However, there is an alternative parsing procedure that will work
for $G_3$.  This alternative method of parsing a string produces
a \nw{right derivation} of the string, that is, a derivation in
which at each step, the non-terminal symbol that is replaced is
the rightmost non-terminal symbol in the string.  Here, for example,
is a right derivation of the string $(x+y)*z$ according to the
grammar $G_3$:
\begin{align*}
  E &\YIELDS T\\
    &\YIELDS T*F\\
    &\YIELDS T*z\\
    &\YIELDS F*z\\
    &\YIELDS (E)*z\\
    &\YIELDS (E+T)*z\\
    &\YIELDS (E+F)*z\\
    &\YIELDS (E+y)*z\\
    &\YIELDS (T+y)*z\\
    &\YIELDS (F+y)*z\\
    &\YIELDS (x+y)*z
\end{align*}
The parsing method that produces this right derivation produces
it from ``bottom to top.''  That is, it begins with
the string $(x+y)*z$ and works backward to the start symbol $E$,
generating the steps of the right derivation in reverse order.
The method works because $G_3$ is what is called an
\nw{LR(1) grammar}.  That is, roughly, it is possible to read
a string from \textbf{L}eft to right and produce a \textbf{R}ight
derivation of the string, by looking ahead at most \textbf{1} symbol at
each step.  Although LL(1) grammars are easier for people to work
with, LR(1) grammars turn out to be very suitable for machine
processing, and they are used as the basis for the parsing
process in many compilers.

LR(1) parsing uses a \nw{shift/reduce} algorithm.  Imagine a
cursor or current position that moves through the string that
is being parsed.  We can visualize the cursor as a vertical
bar, so for the string $(x+y)*z$, we start with the
configuration $|(x+y)*z$.  A \textit{shift} operation simply
moves the cursor one symbol to the right.  For example,
a shift operation would convert $|(x+y)*z$ to $(|x+y)*z$,
and a second shift operation would convert that to
$(x|+y)*z$.  In a reduce
operation, one or more symbols immediately to the left of
the cursor are recognized as the right-hand side of one of
the production rules in the grammar.  These symbols are removed
and replaced by the left-hand side of the production rule.
For example, in the configuration $(x|+y)*z$, the $x$ to the left
of the cursor is the right-hand side of the production rule
$F\PRODUCES x$, so we can apply a reduce operation and replace
the $x$ with $F$, giving $(F|+y)*z$.  This first reduce operation
corresponds to the last step in the right derivation of the
string, $(F+y)*z\YIELDS (x+y)*z$.  Now the $F$ can be recognized
as the right-hand side of the production rule $T\PRODUCES F$,
so we can replace the $F$ with $T$, giving $(T|+y)*z$.
This corresponds to the next-to-last step in the right
derivation, $(T+y)*z\YIELDS (F+y)*z$.

At this point, we have the configuration $(T|+y)*z$.  The $T$
could be the right-hand side of the production rule $E\PRODUCES T$.
However, it could also conceivably come from the rule $T\PRODUCES T*F$.
How do we know whether to reduce the $T$ to $E$ at this point or to
wait for a $*F$ to come along so that we can reduce $T*F\,$?
We can decide by looking ahead at the next character after the
cursor.  Since this character is a $+$ rather than a $*$,
we should choose the reduce operation that replaces $T$ with $E$,
giving $(E|+y)*z$.  What makes $G_3$ an LR(1) grammar is the fact
that we can always decide what operation to apply by looking
ahead at most one symbol past the cursor.

After a few more shift and reduce operations, the configuration
becomes $(E)|*z$, which we can reduce to $T|*z$ by applying the
production rules $F\PRODUCES (E)$ and $T\PRODUCES F$.
Now, faced with $T|*z$, we must once again decide between
a shift operation and a reduce operation that applies the
rule $E\PRODUCES T$.  In this case, since the next character is
a $*$ rather than a $+$, we apply the shift operation, giving
$T*|z$.  From there we get, in succession, $T*z|$,
$T*F|$, $T|$, and finally $E|$.  At this point, we have reduced
the entire string $(x+y)*z$ to the start symbol of the grammar.
The very last step, the reduction of $T$ to $E$ corresponds to
the first step of the right derivation, $E\YIELDS T$.

In summary, LR(1) parsing transforms a string into the
start symbol of the grammar by a sequence of shift and
reduce operations.  Each reduce operation corresponds to a
step in a right derivation of the string, and these steps
are generated in reverse order.  Because the steps in the
derivation are generated from ``bottom to top,'' LR(1)
parsing is a type of \nw{bottom-up parsing}.  LL(1) parsing,
on the other hand, generates the steps in a left derivation
from ``top to bottom'' and so is a type of \nw{top-down parsing}.

\medbreak

Although the language generated by a context-free grammar
is defined in terms of derivations, there is another way of
presenting the generation of a string that is often more useful.
A \nw{parse tree} displays the generation of a string from
the start symbol of a grammar as a two dimensional diagram.
Here are two parse trees that show two derivations of the
string x+y*z according to the grammar $G_1$, which was given
at the beginning of this section:
\bigskip
\centerline{\scaledeps{2.5truein}{fig-5-1}}

\noindent A parse tree is made up of terminal and non-terminal symbols,
connected by lines.  The start symbol is at the top, or ``root,'' of
the tree.  Terminal symbols are at the lowest level, or ``leaves,'' of
the tree.  (For some reason, computer scientists traditionally
draw trees with leaves at the bottom and root at the top.)
A production rule $A\PRODUCES w$ is represented
in a parse tree by the symbol $A$ lying above all the symbols in $w$,
with a line joining $A$ to each of the symbols in $w$.  For
example, in the left parse tree above, the root,
$E$, is connected to the symbols $E$, $+$, and $E$, and this
corresponds to an application of the production rule
$E\PRODUCES E+E$.

It is customary to draw a parse tree with the string of non-terminals
in a row across the bottom, and with the rest of the tree built on
top of that base.  Thus, the two parse trees shown above might
be drawn as:

\bigskip
\centerline{\scaledeps{2.5truein}{fig-5-2}}

Given any derivation of a string, it is possible to construct
a parse tree that shows each of the steps in that derivation.
However, two different derivations can give rise to the same
parse tree, since the parse tree does not show the order in
which production rules are applied.  For example, the parse
tree on the left, above, does not show whether the production
rule $E\PRODUCES x$ is applied before or after the production
rule $E\PRODUCES y$.  However, if we restrict our attention to left
derivations, then we find that each parse tree corresponds to
a unique left derivation and \textit{vice versa}.  I will state this
fact as a theorem, without proof.  A similar result holds for
right derivations.

\begin{theorem}
Let $G$ be a context-free grammar.  There is a one-to-one correspondence
between parse trees and left derivations based on the grammar $G$.
\end{theorem}

Based on this theorem, we can say that a context-free grammar $G$
is ambiguous if and only if there is a string $w\in L(G)$ which has
two parse trees.



\begin{exercises}

\problem Show that each of the following grammars is ambiguous by finding
a string that has two left derivations according to the grammar:
\tparts{
   \vtop{\halign{$#$\hfil\cr
      S\PRODUCES SS\cr
      S\PRODUCES aSb\cr
      S\PRODUCES bSa\cr
      S\PRODUCES\EMPTYSTRING\cr
   }}\quad&
   \vtop{\halign{$#$\hfil\cr
      S\PRODUCES ASb\cr
      S\PRODUCES \EMPTYSTRING\cr
      A\PRODUCES aA\cr
      A\PRODUCES a\cr
   }}\cr
}

\problem Consider the string $z+(x+y)*x$.  Find a left derivation
of this string according to each of the grammars $G_1$, $G_2$, and
$G_3$, as given in this section.

\problem Draw a parse tree for the string $(x+y)*z*x$ according to
each of the grammars $G_1$, $G_2$, and $G_3$, as given in this section.

\problem Draw three different parse trees for the string
$ababbaab$ based on the grammar given in part a) of exercise 1.

\problem Suppose that the string $abbcabac$ has the following parse
tree, according to some grammar $G$:

\centerline{\scaledeps{2in}{fig-5-3}}
\medskip

\ppart List five production rules that must be rules in the grammar $G$,
given that this is a valid parse tree.
\ppart Give a left derivation for the string $abbcabac$ according to the
grammar $G$.
\ppart Give a right derivation for the string $abbcabac$ according to the
grammar $G$.

\problem Show the full sequence of shift and reduce operations
that are used in the LR(1) parsing of the string $x+(y)*z$ according
to the grammar $G_3$, and give the corresponding right derivation
of the string.

\problem This section showed how to use LL(1) and LR(1) parsing to
find a derivation of a string in the language $L(G)$ generated by
some grammar $G$.  How is it possible to use LL(1) or LR(1) parsing
to determine for an arbitrary string $w$ whether $w\in L(G)\,$?
Give an example.

\end{exercises}



