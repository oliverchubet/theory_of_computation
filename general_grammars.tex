\section{General Grammars}\label{S-grammars-5}

At the beginning of this chapter the general idea of a grammar as a set of
rewriting or production rules was introduced.  For most of the chapter, however,
we have restricted our attention to context-free grammars, in which production
rules must be of the form $A\PRODUCES x$ where $A$ is a non-terminal symbol.
In this section, we will consider general grammars, that is, grammars in which
there is no such restriction on the form of production rules.  For a general
grammar, a production rule has the form $u\PRODUCES x$, where $u$ is string
that can contain both terminal and non-terminal symbols.  For convenience, we
will assume that $u$ contains at least one non-terminal symbol, although
even this restriction could be lifted without changing the class of languages
that can be generated by grammars.  Note that a context-free grammar is, in fact,
an example of a general grammar, since production rules in a general grammar
are allowed to be of the form $A\PRODUCES x$.  They just don't have to be of
this form.  I will use the unmodified term \nw{grammar}\index{general grammar} to
refer to general grammars.\footnote{There is another special type of grammar that
is intermediate between context-free grammars and general grammars.  In a
so-called \nw{context-sensitive grammar}, every production rule is of the form
$u\PRODUCES x$ where $|x|\ge|u|$.  We will not cover context-sensitive grammars in
this text.}  The definition of grammar is identical to the
definition of context-free grammar, except for the form of the production rules:

\begin{definition}
A \nw{grammar} is a 4-tuple $(V,\Sigma,P,S)$,
where:

\Item{1.\ }$V$ is a finite set of symbols.  The elements of $V$
are the non-terminal symbols of the grammar.

\Item{2.\ }$\Sigma$ is a finite set of symbols such that $V\cap\Sigma=\emptyset$.
The elements of $\Sigma$ are the terminal symbols of the grammar.

\Item{3.\ }$P$ is a set of production rules.  Each rule is of the
form $u\PRODUCES x$ where $u$ and $x$ are strings in $(V\cup \Sigma)^*$
and $u$ contains at least one symbol from $V$.

\Item{4.\ }$S\in V$.  $S$ is the start symbol of the grammar.
\end{definition}

Suppose $G$ is a grammar.  Just as in the context-free case,
the language generated by $G$ is denoted by $L(G)$ and is defined
as $L(G)=\{x\in\Sigma^*\st S \YIELDS_G^* x\}$.  That is, a string
$x$ is in $L(G)$ if and only if $x$ is a string of terminal symbols
and there is a derivation that produces $x$ from the start symbol,
$S$, in one or more steps.

The natural question is whether there are languages that can be generated
by general grammars but that cannot be generated by context-free languages.
We can answer this question immediately by giving an example of such
a language.  Let $L$ be the language $L=\{w\in\{a,b,c\}^*\st n_a(w)=n_b(w)=n_c(w)\}$.
We saw at the end of the last section that $L$ is not context-free.
However, $L$ is generated by the following grammar:
\begin{align*}
  S&\PRODUCES SABC\\
  S&\PRODUCES \EMPTYSTRING\\
  AB&\PRODUCES BA\\
  BA&\PRODUCES AB\\
  AC&\PRODUCES CA\\
  CA&\PRODUCES AC\\
  BC&\PRODUCES CB\\
  CB&\PRODUCES BC\\
  A&\PRODUCES a\\
  B&\PRODUCES b\\
  C&\PRODUCES c
\end{align*}
For this grammar, the set of non-terminals is $\{S,A,B,C\}$ and the set of
terminal symbols is $\{a,b,c\}$.  Since both terminals and non-terminal
symbols can occur on the left-hand side of a production rule in a general
grammar, it is not possible, in general, to determine which symbols are non-terminal and
which are terminal just by looking at the list of production rules.
However, I will follow the convention that uppercase letters are always
non-terminal symbols.  With this convention, I can continue to specify a
grammar simply by listing production rules.

The first two rules in the above grammar make it possible to produce the strings
$\EMPTYSTRING$, $ABC$, $ABCABC$, $ABCABCABC$, and so on.  Each of these strings
contains equal numbers of $A\text{'}s$, $B\text{'}s$, and $C\text{'}s$.
The next six rules allow the order of the non-terminal symbols in the string
to be changed.  They make it possible to arrange the $A\text{'}s$, $B\text{'}s$,
and $C\text{'}s$ into any arbitrary order.  Note that these rules could not occur
in a context-free grammar.  The last three rules convert the non-terminal symbols
$A$, $B$, and $C$ into the corresponding terminal symbols $a$, $b$, and $c$.
Remember that all the non-terminals must be eliminated in order to produce
a string in $L(G)$.  Here, for example, is a derivation of the string
$baabcc$ using this grammar.  In each line, the string that will be replaced
on the next line is underlined.
\begin{align*}
   S&\YIELDS \underline{S}ABC\\
    &\YIELDS \underline{S}ABCABC\\
    &\YIELDS \underline{AB}CABC\\
    &\YIELDS BA\underline{CA}BC\\
    &\YIELDS BAA\underline{CB}C\\
    &\YIELDS \underline{B}AABCC\\
    &\YIELDS b\underline{A}ABCC\\
    &\YIELDS ba\underline{A}BCC\\
    &\YIELDS baa\underline{B}CC\\
    &\YIELDS baab\underline{C}C\\
    &\YIELDS baabc\underline{C}\\
    &\YIELDS baabcc
\end{align*}
We could produce any string in $L$ in a similar way.  Of course,
this only shows that $L\SUB L(G)$.  To show that $L(G)\SUB L$,
we can observe that for any string $w$ such that $S\;\YIELDSTAR w$,
$n_A(w)+n_a(w) = n_B(w)+n_b(w) = n_C(w)+n_c(w)$.  This follows since the
rule $S\YIELDS SABC$ produces strings in which $n_A(w)=n_B(w)=n_C(w)$,
and no other rule changes any of the quantities
$n_A(w)+n_a(w)$, $n_B(w)+n_b(w)$, or $n_C(w)+n_c(w)$.
After applying these rules to produce a string $x\in L(G)$, we must
have that $n_A(x)$, $n_B(x)$, and $n_C(x)$ are zero.  The fact that
$n_a(x)=n_b(x)=n_c(x)$ then follows from the fact that
$n_A(x)+n_a(x) = n_B(x)+n_b(x) = n_C(x)+n_c(x)$.  That is, $x\in L$.
\medskip

Our first example of a non-context-free language was $\{a^nb^nc^n\st n\in \N\}$.
This language can be generated by a general grammar similar to the previous
example.  However, it requires some cleverness to force the $a\text{'}s$,
$b\text{'}s$, and $c\text{'}s$ into the correct order.  To do this, instead of
allowing $A\text{'}s$, $B\text{'}s$, and $C\text{'}s$ to transform themselves
spontaneously into $a\text{'}s$, $b\text{'}s$, and $c\text{'}s$, we use additional
non-terminal symbols to transform them only after they are in the correct position.
Here is a grammar that does this:
\begin{align*}
  S&\PRODUCES SABC\\
  S&\PRODUCES X\\
  BA&\PRODUCES AB\\
  CA&\PRODUCES AC\\
  CB&\PRODUCES BC\\
  XA&\PRODUCES aX\\
  X&\PRODUCES Y\\
  YB&\PRODUCES bY\\
  Y&\PRODUCES Z\\
  ZC&\PRODUCES cZ\\
  Z&\PRODUCES \EMPTYSTRING
\end{align*}
Here, the first two rules produce one of the strings $X$, $XABC$, $XABCABC$,
$XABCABCABC$, and so on.  The next three rules allow $A\text{'s}$ to move to the
left and $C\text{'}s$ to move to the right, producing a string of the form $XA^nB^nC^n$,
for some $n\in\N$.  The rule $XA\PRODUCES aX$ allows the
$X$ to move through the $A\text{'}s$ from left to right, converting $A\text{'}s$
to $a\text{'}s$ as it goes.  After converting the $A\text{'}s$, the $X$ can be
transformed into a $Y$.  The $Y$ will then move through the $B\text{'}s$, converting
them to $b\text{'}s$.  Then, the $Y$ is transformed into a $Z$, which is responsible
for converting $C\text{'}s$ to $c\text{'}s$.  Finally, an application of the
rule $Z\PRODUCES\EMPTYSTRING$ removes the $Z$, leaving the string $a^nb^nc^n$.

Note that if the rule $X\PRODUCES Y$ is applied before all the $A\text{'}s$ have
been converted to $a\text{'}s$, then there is no way for the remaining $A\text{'}s$
to be converted to $a\text{'}s$ or otherwise removed from the string.  This means
that the derivation has entered a dead end, which can never produce a string
that consists of terminal symbols only.  The only derivations that can produce
strings in the language generated by the grammar are derivations in which the
$X$ moves past all the $A\text{'}s$, converting them all to $a\text{'}s$.  At this
point in the derivation, the string is of the form $a^nXu$ where $u$ is a string
consisting entirely of $B\text{'}s$ and $C\text{'}s$.  At this point, the
rule $X\PRODUCES Y$ can be applied, producing the string $a^nYu$.  Then, if a string
of terminal symbols is ever to be produced, the $Y$ must move past all the $B\text{'}s$,
producing the string $a^nb^nYC^n$.  You can see that the use of three separate
non-terminals, $X$, $Y$, and $Z$, is essential for forcing the symbols in
$a^nb^nc^n$ into the correct order.

\medbreak

For one more example, consider the language $\{a^{n^2}\st n\in\N\}$.  Like the other
languages we have considered in this section, this language is not context-free.
However, it can be generated by a grammar.  Consider the grammar
\begin{align*}
  S&\PRODUCES DTE\\
  T&\PRODUCES BTA\\
  T&\PRODUCES \EMPTYSTRING\\
  BA&\PRODUCES AaB\\
  Ba&\PRODUCES aB\\
  BE&\PRODUCES E\\
  DA&\PRODUCES D\\
  Da&\PRODUCES aD\\
  DE&\PRODUCES \EMPTYSTRING
\end{align*}
The first three rules produce all strings of the form $DB^nA^nE$, for $n\in\N$.
Let's consider what happens to the string $DB^nA^nE$ as the remaining rules are applied.
The next two rules allow a $B$ to move to the right until it reaches the $E$.
Each time the $B$ passes an $A$, a new $a$ is generated, but a $B$ will simply
move past an $a$ without generating any other characters.  Once the $B$ reaches
the $E$, the rule $BE\PRODUCES E$ makes the $B$ disappear.  Each $B$ from the
string $DB^nA^nE$ moves past $n$ $A\text{'}s$ and generates $n$ $a\text{'}s$.
Since there are $n$ $B\text{'}s$, a total of $n^2$ $a\text{'}s$ are generated.
Now, the only way to get rid of the $D$ at the beginning of the string is for
it to move right through all the $A\text{'}s$ and $a\text{'}s$ until it reaches
the $E$ at the end of the string.  As it does this, the rule $DA\PRODUCES D$
eliminates all the $A\text{'}s$ from the string, leaving the string $a^{n^2}DE$.
Applying the rule $DE\PRODUCES\EMPTYSTRING$ to this gives $a^{n^2}$.  This
string contains no non-terminal symbols and so is in the language generated
by the grammar.  We see that every string of the form $a^{n^2}$ is generated
by the above grammar.  Furthermore, only strings of this form can be generated
by the grammar.  

\medbreak

Given a fixed alphabet $\Sigma$, there are only countably many different
languages over $\Sigma$ that can be generated by grammars.  Since there
are uncountably many different languages over $\Sigma$, we know that
there are many languages that cannot be generated by grammars.
However, it is surprisingly difficult to find an actual example of
such a language.

As a first guess, you might suspect that just as $\{a^nb^n \st n\in\N\}$
is an example of a language that is not regular and
$\{a^nb^nc^n\st n\in\N\}$ is an example of a language that is not
context-free, so $\{a^nb^nc^nd^n\st n\in\N\}$ might be an example
of a language that cannot be generated by any grammar.  However,
this is not the case.  The same technique that was used
to produce a grammar that generates $\{a^nb^nc^n\st n\in\N\}$ can
also be used to produce a grammar for $\{a^nb^nc^nd^n\st n\in\N\}$.
In fact, the technique extends to similar languages based on any
number of symbols.

Or you might guess that there is no grammar for the
language $\{a^n\st\,$ $n$ is a prime number$\,\}$.  Certainly, producing
prime numbers doesn't seem like the kind of thing that we would
ordinarily do with a grammar.  Nevertheless, there is a grammar that
generates this language.  We will not actually write down the grammar,
but we will eventually have a way to prove that it exists.

The language $\{a^{n^2}\st n\in\N\}$ really doesn't seem all that
``grammatical'' either, but we produced a grammar for it above.
If you think about how this grammar works, you might get the feeling
that its operation is more like ``computation'' than ``grammar.''
This is our clue.  A grammar can be thought of as a kind of program,
albeit one that is executed in a nondeterministic fashion.  It turns
out that general grammars are precisely as powerful as any other
general-purpose programming language, such as Java or C++.  More
exactly, a language can be generated by a grammar if and only if
there is a computer program whose output consists of a list 
containing all the strings and only the
strings in that language.  Languages that have this property
are said to be \nw[recursively enumerable language]{recursively 
enumerable languages}.  (This term
as used here is {\it not\/} closely related to the idea of a recursive
subroutine.)  The languages that can be generated by general
grammars are precisely the recursively enumerable languages.
We will return to this topic in the next chapter.

It turns out that there are many forms of computation that are
precisely equivalent in power to grammars and to computer programs,
and no one has ever found any form of computation that is more
powerful.  This is one of the great discoveries of the twentieth
century, and we will investigate it further in the next chapter.


\begin{exercises}

\problem Find a derivation for the string $caabcb$, according to the first example
grammar in this section.
Find a derivation for the string $aabbcc$, according to the second example
grammar in this section.
Find a derivation for the string $aaaa$, according to the third example
grammar in this section.

\problem Consider the third sample grammar from this section, which generates
the language $\{a^{n^2}\st n\in\N\}$.  Is the non-terminal symbol $D$ necessary
in this grammar?  What if the first rule of the grammar were replaced by
$S\PRODUCES TE$ and the last three rules were replaced by $A\PRODUCES\EMPTYSTRING$
and $E\PRODUCES\EMPTYSTRING\,$?  Would the resulting grammar still generate
the same language?  Why or why not?

\problem Find a grammar that generates the language $L=\{w\in\{a,b,c,d\}^*\st
n_a(w)=n_b(w)=n_c(w)=n_d(w)\}$.  Let $\Sigma$ be any alphabet.
Argue that the language $\{w\in\Sigma^*\st\,$ all symbols in $\Sigma$ occur equally
often in $w\,\}$ can be generated by a grammar.

\problem For each of the following languages, find a grammar that generates
the language.  In each case, explain how your grammar works.
\pparts{
   \{a^nb^nc^nd^n\st n\in\N\}&
   \{a^nb^mc^{nm}\st n\in\N\text{ and }m\in\N\}\cr
   \{ww\st w\in\{a,b\}^*\}&
   \{www\st w\in\{a,b\}^*\}\cr
   \{a^{2^n}\st n\in\N\}&
   \{w\in\{a,b,c\}^*\st n_a(w)>n_b(w)>n_c(w)\}\cr
}


\end{exercises}





