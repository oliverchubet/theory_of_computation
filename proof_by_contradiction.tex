\section{Proof by Contradiction}

Suppose that we start with some set of assumptions and apply rules
of logic to derive a sequence of statements that can be proved from
those assumptions, and suppose that we derive a statement that we
know to be false.  When the laws of logic are applied to true
statements, the statements that are derived will also be true.
If we derive a false statement by applying rules of logic to a set
of assumptions, then at least one of the assumptions must be false.
This observation leads to a powerful proof technique, which
is known as \nw{proof by contradiction}\index{contradiction}.

Suppose that you want to prove some proposition, $p$.
To apply proof by contradiction, assume that $\NOT p$ is true,
and apply the rules of logic to derive conclusions based on this
assumption.  If it is possible to derive a statement that is known
to be false, it follows that the assumption, $\NOT p$, must be false.
(Of course, if the derivation is based on several assumptions, then you only
know that at least \emph{one} of the assumptions must be false.)
The fact that $\NOT p$ is false proves that $p$ is true.
Essentially, you are arguing that $p$ must be true, because if it 
weren't, then some statement that is known to be false could be proved to be true.
Generally, the false statement that is derived in a proof by
contradiction is of the form $q\AND \NOT q$.  This statement
is a contradiction in the sense that it is false no matter what
the value of $q$.  Note that deriving the contradiction $q\AND \NOT q$
is the same as showing that the two statements, $q$ and $\NOT q$, both
follow from the assumption that $\NOT p$.

As a first example of proof by contradiction, consider the
following theorem:

\begin{theorem}
The number $\sqrt{3}$ is irrational.
\end{theorem}

\begin{proof}
Assume for the sake of contradiction that $\sqrt{3}$ is rational.
Then $\sqrt{3}$ can be written as the ratio of two integers,
$\sqrt{3} = \frac{m'}{n'}$ for some integers $m'$ and $n'$.
Furthermore, the fraction $\frac{m'}{n'}$ can be reduced to lowest
terms by canceling all common factors of $m'$ and $n'$.  So
$\sqrt{3} = \frac{m}{n}$ for some integers $m$ and $n$ which have no
common factors.  Squaring both sides of this equation gives
$3 = \frac{m^2}{n^2}$ and re-arranging gives $3n^2 = m^2.$  From
this equation we see that $m^2$ is divisible by~3; you proved in
the previous section (Exercise~6) that $m^2$ is divisible by 3 iff $m$ is
divisible by 3.  Therefore $m$ is divisible by 3 and we can write
$m=3k$ for some integer $k$.  Substituting $m=3k$ into the last equation
above gives $3n^2 = (3k)^2$ or $3n^2 = 9k^2$, which in turn becomes
$n^2 = 3k^2.$  From this we see that $n^2$ is divisible by 3, and
again we know that this implies that $n$ is divisible by 3.  But
now we have (i) $m$ and $n$ have no common factors, and (ii) $m$ and
$n$ have a common factor, namely 3.  It is impossible for both these
things to be true, yet our argument has been logically correct.  
Therefore our original assumption, namely that $\sqrt{3}$ is rational,
must be incorrect.  Therefore $\sqrt{3}$ must be irrational.
\end{proof}

\medbreak

One of the oldest mathematical proofs, which goes all the
way back to Euclid\index{Euclid}, is a proof by contradiction.
Recall that a prime number is an integer $n$, greater than 1, 
such that the only positive integers that evenly divide $n$ are
1 and $n$.  We will show that there are infinitely many primes.
Before we get to the theorem, we need a lemma.
(A \nw{lemma} is a theorem that is introduced only because it
is needed in the proof of another theorem.  Lemmas help to
organize the proof of a major theorem into manageable chunks.)

\begin{lemma}\label{helpful}
If $N$ is an integer and $N>1$, then there is a prime number
which evenly divides $N$.
\end{lemma}
\begin{proof}
Let $D$ be the smallest integer which is greater than 1 and
which evenly divides $N$.  ($D$ exists since there is at least one
number, namely $N$ itself, which is greater than 1 and which
evenly divides $N$.  We use the fact that any non-empty subset
of $\N$ has a smallest member.)  I claim that $D$ is prime, so that
$D$ is a prime number that evenly divides $N$.

Suppose that $D$ is not prime.  We show that this assumption
leads to a contradiction.  Since $D$ is not prime, then, by definition,
there is a number $k$ between 2 and $D-1$, inclusive, such that
$k$ evenly divides $D$.  But since $D$ evenly divides $N$, we also
have that $k$ evenly divides $N$ (by exercise 5 in the previous section).  That is, $k$ is an integer
greater than one which evenly divides $N$.  But since $k$ is
less than $D$, this contradicts the fact that $D$ is the \emph{smallest}
such number.  This contradiction proves that $D$ is a prime number.
\end{proof}

\begin{theorem}
There are infinitely many prime numbers.
\end{theorem}
\begin{proof}
Suppose that there are only finitely many prime numbers.  We will
show that this assumption leads to a contradiction.

Let $p_1$, $p_2$, \dots, $p_n$ be a complete list of all prime numbers
(which exists under the assumption that there are only finitely many
prime numbers).  Consider the number $N$ obtained by multiplying
all the prime numbers together and adding one.  That is,
\[N=(p_1\cdot p_2\cdot p_3\cdots p_n) + 1.\]
Now, since $N$ is larger than any of the prime numbers $p_i$, and
since $p_1$, $p_2$, \dots, $p_n$ is a \emph{complete} list of prime numbers,
we know that $N$ cannot be prime.  By the lemma, there is a prime
number $p$ which evenly divides $N$.  Now, $p$ must be one of the
numbers $p_1$, $p_2$, \dots, $p_n$.  But in fact, none of these numbers evenly
divides $N$, since dividing $N$ by any $p_i$ leaves a remainder of 1.
This contradiction proves that the assumption that there are only
finitely many primes is false.
\end{proof}

This proof demonstrates the power of proof by contradiction.
The fact that is proved here is not at all obvious, and yet it can
be proved in just a few paragraphs.



\begin{exercises}

\problem Suppose that $a_1$, $a_2$, \dots, $a_{10}$ are real numbers,
and suppose that $a_1+a_2+\cdots+a_{10}>100$.  Use a proof by contradiction
to conclude that at least one of the numbers $a_i$ must be greater than~10.


\problem Prove that each of the following statements is true. 
In each case, use a proof by contradiction.  Remember that the
negation of $p\IMP q$ is $p \AND \NOT q$.
\ppart Let $n$ be an integer.  If $n^2$ is an even integer, then 
$n$ is an even integer.  
\ppart $\sqrt{2}$ is irrational.
\ppart If $r$ is a rational number and $x$ is an
irrational number, then $r+x$ is an irrational number. (That is, the
sum of a rational number and an irrational number is irrational.)
\ppart If $r$ is a non-zero rational number and $x$ is an
irrational number, then $rx$ is an irrational number.    
\ppart If $r$ and $r+x$ are both rational, then $x$ is rational.

\problem The \nw{pigeonhole principle} is the following obvious
observation:  If you have $n$ pigeons in $k$ pigeonholes and if $n>k$,
then there is at least one pigeonhole that contains more than
one pigeon.  Even though this observation seems obvious, it's a
good idea to prove it.  Prove the pigeonhole principle using a
proof by contradiction.

\end{exercises}




