\section{Predicates and Quantifiers}\label{S-logic-4}

In propositional logic,\index{propositional logic} we can let $p$ stand for ``Roses are red'' and
$q$ stand for ``Violets are blue.''  Then $p\AND q$ will stand for
``Roses are red and violets are blue.''  But we lose a lot in the
translation into logic.  Since propositional logic only deals with
truth values, there's nothing we can do with $p$ and $q$ in propositional
logic that has anything to do with roses, violets, or color.
To apply logic to such things, we need \nw[predicate]{predicates}.
The type of logic that uses predicates is called \nw{predicate
logic}, or, when the emphasis is on manipulating and reasoning
with predicates, \nw{predicate calculus}.

A predicate is a kind of incomplete proposition, which becomes
a proposition when it is applied to some element (or, as we'll see later,
to several elements).  In the proposition ``the rose is red,'' the
predicate is \emph{is red}.  By itself, ``is red'' is not a proposition.
Think of it as having an empty slot, that needs to be filled in
to make a proposition: ``---~is red.''  In the proposition
``the rose is red,'' the slot is filled by the element ``the rose,''
but it could just as well be filled by other elements:
``the barn is red''; ``the wine is red''; ``the banana is red.''
Each of these propositions uses the same predicate, but they are
different propositions and they can have different truth values.

If $P$ is a predicate and $a$ is an element, then $P(a)$ stands for
the proposition that is formed when $P$ is applied to $a$.  If $P$
represents ``is red'' and $a$ stands for ``the rose,'' then
$P(a)$ is ``the rose is red.''  If $M$ is the predicate
``is mortal'' and $s$ is ``Socrates,'' then $M(s)$ is the proposition
``Socrates is mortal.'' 

Now, you might be asking, just what is an \emph{element}\index{element} anyway?
I am using the term here to mean some specific, identifiable thing
to which a predicate can be applied.  Generally, it doesn't make
sense to apply a given predicate to every possible element, but only
to elements in a certain category.  For example, it probably doesn't
make sense to apply the predicate ``is mortal'' to your living room
sofa.  This predicate only applies to elements in the category of
living things, since there is no way something can be mortal unless it
is alive.  This category is called the domain for
the predicate.\footnote{In the language
of set theory, which will be introduced in the next chapter,
we would say that a domain is a set, $U$, and
a predicate is a function from $U$ to the set of truth values.
The definition should be clear enough without the formal language
of set theory, and in fact you should think of this definition---and
many others---as motivation for that language.}

We are now ready for a formal definition of one-place
predicates.  A one-place
predicate, like all the examples we have seen so far, has a single
slot which can be filled in with one element:


\begin{definition}
A \nw{one-place predicate}\index{predicate} associates a proposition with each element in some
collection of elements.  This collection is called the \nw{domain
} for the predicate.  If $P$ is a predicate and $a$ is
an element in the domain for $P$, then $P(a)$ denotes
the proposition that is associated with $a$ by~$P$.  We say that $P(a)$
is the result of \nw[none]{applying} $P$ to~$a$.
\end{definition}

We can obviously extend this to predicates that can be applied to
two or more elements.  In the proposition ``John loves Mary,''
\emph{loves} is a two-place predicate.  Besides John and Mary,
it could be applied to other pairs of elements:  ``John loves Jane,''
``Bill loves Mary,'' ``John loves Bill,''  ``John loves John.''
If $Q$ is a two-place
predicate, then $Q(a,b)$ denotes the proposition that is obtained
when $Q$ is applied to the elements $a$ and~$b$.  Note that each of
the ``slots'' in a two-place predicate can have its own domain of
discourse.  For example, if $Q$ represents the predicate ``owns,''
then $Q(a,b)$ will only make sense when $a$ is a person and $b$ is an
inanimate object.  An example of a three-place predicate is
``$a$~gave $b$ to~$c$,'' and a four-place predicate would be
``$a$~bought $b$ from $c$ for $d$ dollars.''  But keep in mind that
not every predicate has to correspond to an English sentence.

When predicates are applied to elements, the results are propositions,
and all the operators of propositional logic can be applied to these
propositions just as they can to any propositions.  Let $R$ be the
predicate ``is red,'' and let $L$ be the two-place predicate ``loves.''
If $a$, $b$, $j$, $m$, and $b$ are elements belonging to the 
appropriate categories, then we can form compound propositions such
as:
\[
\begin{array}{l@{\qquad}l}
   R(a)\AND R(b)         &\text{$a$ is red and $b$ is red}\\
   \NOT R(a)             &\text{$a$ is not red}\\
   L(j,m)\AND\NOT L(m,j) &\text{$j$ loves $m$, and $m$ does not love $j$}\\
   L(j,m)\IMP L(b,m)     &\text{if $j$ loves $m$ then $b$ loves $m$}\\
   R(a)\IFF L(j,j)       &\text{$a$ is red if and only if $j$ loves $j$}\\
\end{array}
\]


\medbreak
Let's go back to the proposition with which we started this section:
``Roses are red.''  This sentence is more difficult to handle than
it might appear.  We still can't express it properly in logic.
The problem is that this proposition is not saying something about
some particular element.  It really says that \emph{all} roses are red 
(which happens to be a false statement, but that's what it means).
Predicates can only be applied to individual elements.

Many other sentences raise similar difficulties:
``All persons are mortal.''  ``Some roses are red, but no roses are black.''
``All math courses are interesting.''  ``Every prime number greater than two
is odd.''  Words like \emph{all}, \emph{no}, \emph{some}, and \emph{every}
are called \nw{quantifiers}.\index{quantifier!in English}  We need to be able to express similar concepts
in logic.

Suppose that $P$ is a predicate, and we want to express the proposition that
$P$ is true when applied to any element in the domain.
That is, we want to say ``for any element $x$ in the domain,
$P(x)$ is true.''  In predicate logic, we write this in symbols as
$\forall x(P(x))$.  The $\forall$ symbol, which looks like an
upside-down~A, is usually read ``for all,'' so that $\forall x(P(x))$
is read as ``for all $x$, $P(x)$.''  (It is understood that this means
for all $x$ in the domain for~$P$.)  For example,
if $R$ is the predicate ``is red'' and the domain consists
of all roses, then $\forall x(R(x))$ expresses the proposition
``All roses are red.''  Note that the same proposition could be
expressed in English as ``Every rose is red'' or ``Any rose is red.''
 
Now, suppose we want to say that a predicate, $P$, is true for \emph{some}
element in its domain.  This is expressed in predicate
logic as $\exists x(P(x))$.  The $\exists$ symbol, which looks like a
backwards~E, is usually read ``there exists,'' but a more exact reading
would be ``there is at least one.'' Thus, $\exists x(P(x))$ is read
as ``There exists an $x$ such that $P(x)$,'' and it means ``there is
at least one $x$ in the domain for $P$ for which $P(x)$
is true.''  If, once again, $R$ stands for ``is red'' and the domain
is ``roses,'' then $\exists x(R(x))$ could be expressed
in English as ``There is a red rose'' or ``At least one rose is red''
or ``Some rose is red.''  It might also be expressed as ``Some roses
are red,'' but the plural is a bit misleading since $\exists x(R(x))$
is true even if there is only one red rose.
We can now give the formal definitions:

\begin{definition}
Suppose that $P$ is a one-place predicate.  Then $\forall x(P(x))$ is
a proposition, which is true if and only if $P(a)$ is true for every
element $a$ in the domain for~$P$.  And $\exists x(P(x))$
is a proposition which is true if and only if there is at least one
element, $a$, in the domain for $P$ for which $P(a)$ is
true.  The $\forall$ symbol is called the \nw{universal quantifier},
and $\exists$ is called the \nw{existential quantifier}.
\end{definition}

The $x$ in $\forall x(P(x))$ and $\exists x(P(x))$ is a variable.\index{variable}
(More precisely, it is an \emph{element} variable, since its value
can only be an element.)\index{element!variable}
Note that a plain $P(x)$---without the $\forall x$ or $\exists x$---is
not a proposition.  $P(x)$~is neither true nor false because $x$
is not some particular element, but just a placeholder in a slot that
can be filled in with an element.  $P(x)$~would stand for
something like the statement ``$x$~is red,'' which is not really a
statement in English at all.  But it becomes a statement when
the $x$ is replaced by some particular element, such as ``the rose.''
Similarly, $P(x)$ becomes a proposition if some element $a$ is substituted
for the~$x$, giving $P(a)$.\footnote{There is certainly room for confusion
about names here.  In this discussion, $x$ is a variable and $a$ is 
an element.  But that's only because I said so.  Any letter could be used
in either role, and you have to pay attention to the context to
figure out what is going on.  Usually, $x$, $y$, and $z$ will be variables.}

An \nw{open statement} is an expression that contains one or more element 
variables, which becomes a proposition when elements are substituted
for the variables.  (An open statement has open ``slots'' that need to
be filled in.)  $P(x)$ and ``$x$ is red'' are examples of open 
statements that contain one variable.  If $L$ is a two-place predicate
and $x$ and $y$ are variables, then $L(x,y)$ is an open statement
containing two variables.  An example in English would be
``$x$~loves~$y$.''  The variables in an open statement are called 
\nw[free variable]{free variables}.  An open statement that contains $x$ as a free
variable can be quantified with $\forall x$ or $\exists x$.
The variable $x$ is then said to be \nw[bound variable]{bound}.  For example,
$x$ is free in $P(x)$ and is bound in $\forall x(P(x))$ and
$\exists x(P(x))$.  The free variable $y$ in $L(x,y)$ becomes
bound in $\forall y(L(x,y))$ and in $\exists y(L(x,y))$.

Note that $\forall y(L(x,y))$ is still an open statement, since
it contains $x$ as a free variable.\index{quantifier!on a two-place predicate}
Therefore, it is possible to
apply the quantifier $\forall x$ or $\exists x$ to $\forall y(L(x,y))$,
giving $\forall x\big(\forall y(L(x,y))\big)$ and
$\exists x\big(\forall y(L(x,y))\big)$.  Since all the variables are
bound in these expressions, they are propositions.  If $L(x,y)$ represents
``$x$~loves~$y$,'' then $\forall y(L(x,y))$ is something like ``$x$~loves
everyone,''  and $\exists x\big(\forall y(L(x,y))\big)$ is the
proposition, ``There is someone who loves everyone.''  Of course, we
could also have started with $\exists x(L(x,y))$: ``There is someone
who loves~$y$.''  Applying $\forall y$ to this gives 
$\forall y\big(\exists x(L(x,y))\big)$,
which means ``For every person, there is someone who loves that person.''
Note in particular that $\exists x\big(\forall y(L(x,y))\big)$ and
$\forall y\big(\exists x(L(x,y))\big)$ do \emph{not} mean the same thing.
Altogether, there are eight different propositions that can
be obtained from $L(x,y)$ by applying quantifiers, with six distinct
meanings among them.

(From now on, I will leave out parentheses when there is no ambiguity.
For example, I will write $\forall x\, P(x)$ instead of $\forall x(P(x))$
and $\exists x\,\exists y\,L(x,y)$ instead of
$\exists x\big(\exists y(L(x,y))\big)$.  Furthermore, I will
sometimes give predicates and elements names that are complete words
instead of just letters, as in  $Red(x)$ and $Loves(john,mary)$.
This might help to make examples more readable.)

\medbreak

In predicate logic, the operators and laws of Boolean algebra\index{Boolean algebra!in predicate logic} still
apply.  For example, if $P$ and $Q$ are one-place predicates and
$a$ is an element in the domain, then $P(a)\IMP Q(a)$
is a proposition, and it is logically equivalent to $\NOT P(a)\OR Q(a)$.
Furthermore, if $x$ is a variable, then $P(x)\IMP Q(x)$ is an open
statement, and $\forall x(P(x)\IMP Q(x))$ is a proposition.
So are $P(a)\AND(\exists x\,Q(x))$ and $(\forall x\,P(x))\IMP(\exists xP(x))$.
Obviously, predicate logic can be very expressive.  Unfortunately,
the translation between predicate logic and English sentences is not
always obvious.

Let's look one more time at the proposition ``Roses are red.''
If the domain consists of roses, this translates into
predicate logic as $\forall x\, Red(x)$.  However, the sentence makes
more sense if the domain is larger---for example if it
consists of all flowers.  Then ``Roses are red'' has to be read as
``All flowers which are roses are red,'' or ``For any flower,
if that flower is a rose, then it is red.'' The last form translates
directly into logic as $\forall x\big(Rose(x)\IMP Red(x)\big)$.
Suppose we want to say that all red roses are pretty.  The phrase
``red rose'' is saying both that the flower is a rose and that it is
red, and it must be translated as a conjunction, $Rose(x)\AND Red(x)$.
So, ``All red roses are pretty'' can be rendered as
$\forall x\big((Rose(x)\AND Red(x))\IMP Pretty(x)\big)$.

Here are a few more examples of translations from predicate logic
to English.  Let $H(x)$ represent ``$x$~is happy,'' let
$C(y)$~represent ``$y$~is a computer,'' and let $O(x,y)$~represent
``$x$~owns~$y$.''  (The domain for $x$ consists of 
people, and the domain for $y$ consists of inanimate objects.)
Then we have the following translations:

\begin{itemize}
\setlength{\itemsep}{0pt plus 1 pt}
\setlength{\parsep}{0pt plus 1 pt}
\item Jack owns a computer: $\exists x\big(O(jack,x)\AND C(x)\big)$.
(That is, there is at least one thing such that Jack owns that thing and that thing
is a computer.)
\item Everything Jack owns is a computer: $\forall x\big(O(jack,x)\IMP C(x)\big)$.
\item If Jack owns a computer, then he's happy:\\*
\hspace*{0.5in}$\big(\exists y(O(jack,y)\AND C(y))\big)\IMP H(jack)$.
\item Everyone who owns a computer is happy:\\*
\hspace*{0.5in} $\forall x\big(\,\big(\exists y(O(x,y)\AND C(y)\big)\IMP H(x)\big)\,\big)$.
\item Everyone owns a computer: $\forall x\,\exists y\big(C(y)\AND O(x,y)\big)$.
(Note that this allows each person to own a different computer.
The proposition $\exists y\,\forall x\big(C(y)\AND O(x,y)\big)$
would mean that there is a single computer which is owned by
everyone.)
\item Everyone is happy: $\forall xH(x)$.
\item Everyone is unhappy: $\forall x(\NOT H(x))$.
\item Someone is unhappy: $\exists x(\NOT H(x))$.
\item At least two people are happy:
 $\exists x \exists y\big(H(x) \AND H(y) \AND (x\ne y)\big)$.  (The stipulation
 that $x\ne y$ is necessary because two different variables can refer to
 the same element.  The proposition $\exists x\exists y(H(x)\AND H(y))$ is
 true even if there is only one happy person.)
\item There is exactly one happy person:\\*
 \hspace*{0.5 in}$\big(\exists x H(x)\big)) \AND \big(\forall y \forall z((H(y)\AND H(z))\IMP (y=z))\big)$.
 (The first part of this conjunction says that there is at least one happy person.
 The second part says that if $y$ and $z$ are both happy people, then they are actually
 the same person. That is, it's not possible to find two \emph{different} people who
 are happy.)
\end{itemize}

\medskip

To calculate in predicate logic, we need a notion of logical equivalence.
Clearly, there are pairs of propositions in predicate logic that mean the same
thing.  Consider the propositions $\NOT(\forall x H(x))$ and $\exists x(\NOT H(x))$, where
$H(x)$ represents ``$x$~is happy.'' The first of these propositions means
``Not everyone is happy,'' and the second means ``Someone is not happy.''
These statements have the same truth value:  If not everyone is happy, then someone is
unhappy and vice versa.  But logical equivalence is much stronger than just
having the same truth value.  In propositional logic, logical equivalence
is defined in terms of propositional variables:  two compound propositions
are logically equivalent if they have the same truth values for all possible
truth values of the propositional variables they contain.  In predicate logic, two
formulas are logically equivalent\index{logical equivalence!in predicate logic}
if they have the same truth value for all
possible predicates.

Consider $\NOT(\forall x P(x))$ and $\exists x(\NOT P(x))$.\index{negation!of quantified statements}
These formulas make
sense for any predicate $P$, and for any predicate $P$ they have the same truth
value.  Unfortunately, we can't---as we did in propositional logic---just check
this fact with a truth table: there are no subpropositions, connected by
$\AND$, $\OR$, etc, out of which to build a table.  So, let's reason it out:
To say $\NOT(\forall x P(x))$ is true is just to say that it is not the case that
$P(x)$ is true for all possible elements $x$.  So, there must be some element $a$
for which $P(a)$ is false.  Since $P(a)$ is false, $\NOT P(a)$ is true.
But saying that there is an $a$ for which $\NOT P(a)$ is true is just saying
that $\exists x(\NOT P(x))$ is true.  So, the truth of $\NOT(\forall x P(x))$
implies the truth of $\exists x (\NOT P(x))$.  On the other hand, if 
$\NOT(\forall x P(x))$ is false, then $\forall x P(x)$ is true.  Since $P(x)$
is true for every~$x$, $\NOT P(x)$ is false for every~$x$; that is, there is no
element $a$ for which the statement $\NOT P(a)$ is true.
But this just means that the statement $\exists x(\NOT P(x))$
is false.  In any case, then, the truth values of $\NOT(\forall x P(x))$ and
$\exists x(\NOT P(x))$ are the same.  Since this is true for any predicate $P$,
we will say that these two formulas are logically equivalent and write
$\NOT(\forall x P(x)) \equiv \exists x(\NOT P(x))$.

\fig
  {F-predlogic}
  {Four important rules of predicate logic.  $P$ can be any one-place predicate,
   and $Q$ can be any two-place predicate.  The first two rules are called
   DeMorgan's Laws for predicate logic.}
  {\begin{tabular}{|c|}
     \hline
     \strut$\NOT\,(\forall x P(x)) \equiv \exists x(\NOT P(x))$\\
     \hline
     \strut$\NOT\,(\exists x P(x)) \equiv \forall x(\NOT P(x))$\\
     \hline
     \strut$\forall x \forall y Q(x,y) \equiv \forall y \forall x Q(x,y)$\\
     \hline
     \strut$\exists x \exists y Q(x,y) \equiv \exists y \exists x Q(x,y)$\\
     \hline
  \end{tabular}
 }

A similar argument would show that $\NOT(\exists x P(x)) \equiv \forall x(\NOT P(x))$.
These two equivalences, which explicate the relation between negation and quantification,
are known as DeMorgan's Laws\index{DeMorgan's Laws} for predicate logic.  (They are closely related to
DeMorgan's Laws for propositional logic; see the exercises.)  These
laws can be used to help simplify expressions.  For example, 
\begin{align*}
  \NOT\,\forall y (R(y)\OR Q(y)) &\equiv \exists y(\NOT(R(y)\OR Q(y)))\\
       &\equiv \exists y((\NOT R(y))\AND(\NOT Q(y))\\
\end{align*}
It might not be clear exactly why this qualifies as a ``simplification,''
but it's generally considered simpler to have the negation operator applied
to basic propositions such as $R(y)$, rather than to quantified expressions
such as \hbox{$\forall y (R(y)\OR Q(y))$}.
For a more complicated example:
\begin{align*}
  \NOT\,\exists x\big(P(x)&\AND (\forall y (Q(y)\IMP Q(x)))\big)\\
            &\equiv\forall x\big(\NOT\big(P(x)\AND (\forall y (Q(y)\IMP Q(x)))\big)\\
            &\equiv\forall x\big((\NOT P(x))\OR (\NOT \forall y (Q(y)\IMP Q(x)))\big)\\
            &\equiv\forall x\big((\NOT P(x))\OR (\exists y(\NOT (Q(y)\IMP Q(x))))\big)\\
            &\equiv\forall x\big((\NOT P(x))\OR (\exists y(\NOT (\NOT Q(y)\OR Q(x))))\big)\\
            &\equiv\forall x\big((\NOT P(x))\OR (\exists y(\NOT\NOT Q(y)\AND \NOT Q(x)))\big)\\
            &\equiv\forall x\big((\NOT P(x))\OR (\exists y(Q(y)\AND \NOT Q(x)))\big)\\
\end{align*}
DeMorgan's Laws are listed in Figure~\ref{F-predlogic} along with two
other laws of predicate logic.  The other laws allow you to interchange
the order of the variables when two quantifiers of the same type
(both $\exists$ or $\forall$) occur together. 

To define logical equivalence in predicate logic more formally,
we need to talk about formulas that contain predicate variables,\index{variable}
that is, variables that act as place-holders for arbitrary predicates
in the same way that propositional variables are place-holders for
propositions and element variables are place-holders for
elements.  With this in mind, we can define logical equivalence
and the closely related concept of tautology for predicate logic.

\begin{definition}
Let $\mathscr{P}$ be a formula of predicate logic which contains one or more
predicate variables.  $\mathscr{P}$~is said to be a \nw{tautology}
if it is true whenever all the predicate variables that it contains are replaced
by actual predicates.  Two formulas $\mathscr{P}$ and $\mathscr{Q}$ are
said to be \nw[logical equivalence!in predicate logic]{logically equivalent} if $\mathscr{P}\IFF\mathscr{Q}$ is
a tautology, that is if $\mathscr{P}$ and $\mathscr{Q}$ always have the same
truth value when the predicate variables they contain are replaced by actual
predicates.  The notation $\mathscr{P}\equiv\mathscr{Q}$ asserts that
$\mathscr{P}$ is logically equivalent to $\mathscr{Q}$.
\end{definition}




\begin{exercises}

\problem Simplify each of the following propositions.  In your answer, the
$\NOT$ operator should be applied only to individual predicates.
\pparts{
   \NOT\,\forall x (\NOT P(x)) & \NOT\,\exists x(P(x)\AND Q(x))\cr
   \NOT \,\forall z(P(z)\IMP Q(z))&
      \NOT\big((\forall x P(x))\AND \forall y(Q(y))\big) \cr
   \NOT\, \forall x \exists y P(x,y)&
      \NOT\,\exists x (R(x)\AND \forall y S(x,y))\cr
    \NOT\,\exists y(P(y)\IFF Q(y))&
       \NOT \big(\forall x (P(x)\IMP (\exists y Q(x,y)))\big) \cr
}

\problem Give a careful argument to show that the second of DeMorgan's laws for
predicate calculus,
$\NOT(\forall x P(x)) \equiv \exists x(\NOT P(x))$, is valid.

\problem Find the negation of each of the following propositions.
Simplify the result; in your answer, the
$\NOT$ operator should be applied only to individual predicates.
\ppart $\NOT$$\exists n (\forall s C(s,n))$
\ppart $\NOT$$\exists n (\forall s (L(s,n) \IMP P(s)))$
\ppart $\NOT$$\exists n (\forall s (L(s,n) \IMP (\exists x \exists y \exists z Q(x,y,z))))$.
\ppart $\NOT$$\exists n (\forall s (L(s,n) \IMP (\exists x \exists y \exists z (s=xyz \AND 
R(x,y) \AND T(y) \AND U(x,y,z))))$.

\problem Suppose that the domain for a predicate $P$
contains only two elements.  Show that $\forall x P(x)$ is equivalent to
a conjunction of two simple propositions, and $\exists x P(x)$ is equivalent
to a disjunction.  Show that in this case, DeMorgan's Laws for propositional
logic and DeMorgan's Laws for predicate logic actually say exactly the same
thing.  Extend the results to a domain that contains exactly
three elements.

\problem Let $H(x)$ stand for ``$x$ is happy,'' where the domain 
consists of people.  Express the proposition ``There are exactly three happy
people'' in predicate logic.

\problem Let $T(x,y)$ stand for ``$x$~has taken~$y$,'' where the
domain for $x$ consists of students and the domain
for $y$ consists of math courses (at your school).
Translate each of the following propositions into an unambiguous English sentence:
\pparts{ \forall x\,\forall y \,T(x,y) & \forall x \,\exists y \,T(x,y) & \forall y \,\exists x \,T(x,y)\cr
         \exists x\,\exists y \,T(x,y) & \exists x \,\forall y \,T(x,y) & \exists y \,\forall x \,T(x,y)\cr
}

\problem Let $F(x,t)$ stand for ``You can fool person $x$ at time~$t$.''
Translate the following sentence into predicate logic:
``You can fool some of the people all of the time, and you can fool
all of the people some of the time, but you can't fool all of the
people all of the time.''

\problem Translate each of the following sentences into a proposition 
using predicate logic.  Make up any predicates you need.  State what
each predicate means and what its domain is.
\ppart All crows are black.
\ppart Any white bird is not a crow.
\ppart Not all politicians are honest.
\ppart All green elephants have purple feet.
\ppart There is no one who does not like pizza.
\ppart Anyone who passes the final exam will pass the course.
\ppart If $x$ is any positive number, then there is a number $y$ such that
$y^2=x$.

\problem The sentence ``Someone has the answer to every question'' is
ambiguous.  Give two translations of this sentence into predicate logic,
and explain the difference in meaning.

\problem The sentence ``Jane is looking for a dog'' is ambiguous.
One meaning is that there is some particular dog---maybe the one she lost---that
Jane is looking for.  The other meaning is that Jane is looking for any old
dog---maybe because she wants to buy one.  Express the first meaning in
predicate logic.  Explain why the second meaning is \emph{not}
expressed by $\forall x(Dog(x)\IMP LooksFor(jane,x))$.  In fact, the
second meaning cannot be expressed in predicate logic.  Philosophers
of language spend a lot of time thinking about things like this.
They are especially fond of the sentence ``Jane is looking for a unicorn,''
which is not ambiguous when applied to the real world.  Why is that?



\end{exercises}




