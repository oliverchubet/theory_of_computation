\section{Predicates and Quantifiers}\label{S-logic-4}

In propositional logic,\index{propositional logic} we can let $p$ stand for ``Roses are red'' and
$q$ stand for ``Violets are blue.''  Then $p\land q$ will stand for
``Roses are red and violets are blue.''  But we lose a lot in the
translation into logic.  Since propositional logic only deals with
truth values, there's nothing we can do with $p$ and $q$ in propositional
logic that has anything to do with roses, violets, or color.
To apply logic to such things, we need \nw[predicate]{predicates}.
The type of logic that uses predicates is called \nw{predicate
logic}, or, when the emphasis is on manipulating and reasoning
with predicates, \nw{predicate calculus}.

A predicate is a kind of incomplete proposition, which becomes
a proposition when it is applied to some element (or, as we'll see later,
to several elements).  In the proposition ``the rose is red,'' the
predicate is \emph{is red}.  By itself, ``is red'' is not a proposition.
Think of it as having an empty slot, that needs to be filled in
to make a proposition: ``---~is red.''  In the proposition
``the rose is red,'' the slot is filled by the element ``the rose,''
but it could just as well be filled by other elements:
``the barn is red''; ``the wine is red''; ``the banana is red.''
Each of these propositions uses the same predicate, but they are
different propositions and they can have different truth values.

\medbreak

Let's go back to the proposition with which we started this section:
``Roses are red.''  This sentence is more difficult to handle than
it might appear.  We still can't express it properly in logic.
The problem is that this proposition is not saying something about
some particular element.  It really says that \emph{all} roses are red 
(which happens to be a false statement, but that's what it means).
Predicates can only be applied to individual elements.

Many other sentences raise similar difficulties:
``All persons are mortal.''  ``Some roses are red, but no roses are black.''
``All math courses are interesting.''  ``Every prime number greater than two
is odd.''  Words like \emph{all}, \emph{no}, \emph{some}, and \emph{every}
are called \nw{quantifiers}.\index{quantifier!in English}  We need to be able to express similar concepts
in logic.


\begin{definition}
Suppose that $P$ is a one-place predicate.  Then $\forall x(P(x))$ is
a proposition, which is true if and only if $P(a)$ is true for every
element $a$ in the domain for~$P$.  And $\exists x(P(x))$
is a proposition which is true if and only if there is at least one
element, $a$, in the domain for $P$ for which $P(a)$ is
true.  The $\forall$ symbol is called the \nw{universal quantifier},
and $\exists$ is called the \nw{existential quantifier}.
\end{definition}

(From now on, I will leave out parentheses when there is no ambiguity.
For example, I will write $\forall x\, P(x)$ instead of $\forall x(P(x))$
and $\exists x\,\exists y\,L(x,y)$ instead of
$\exists x\big(\exists y(L(x,y))\big)$.  Furthermore, I will
sometimes give predicates and elements names that are complete words
instead of just letters, as in  $Red(x)$ and $Loves(john,mary)$.
This might help to make examples more readable.)

\medbreak

To calculate in predicate logic, we need a notion of logical equivalence.
Clearly, there are pairs of propositions in predicate logic that mean the same
thing.  Consider the propositions $\NOT(\forall x H(x))$ and $\exists x(\NOT H(x))$, where
$H(x)$ represents ``$x$~is happy.'' The first of these propositions means
``Not everyone is happy,'' and the second means ``Someone is not happy.''
These statements have the same truth value:  If not everyone is happy, then someone is
unhappy and vice versa.  But logical equivalence is much stronger than just
having the same truth value.  In propositional logic, logical equivalence
is defined in terms of propositional variables:  two compound propositions
are logically equivalent if they have the same truth values for all possible
truth values of the propositional variables they contain.  In predicate logic, two
formulas are logically equivalent\index{logical equivalence!in predicate logic}
if they have the same truth value for all
possible predicates.

Consider $\NOT(\forall x P(x))$ and $\exists x(\NOT P(x))$.\index{negation!of quantified statements}
These formulas make
sense for any predicate $P$, and for any predicate $P$ they have the same truth
value.  Unfortunately, we can't---as we did in propositional logic---just check
this fact with a truth table: there are no subpropositions, connected by
$\land$, $\lor$, etc, out of which to build a table.  So, let's reason it out:
To say $\NOT(\forall x P(x))$ is true is just to say that it is not the case that
$P(x)$ is true for all possible elements $x$.  So, there must be some element $a$
for which $P(a)$ is false.  Since $P(a)$ is false, $\NOT P(a)$ is true.
But saying that there is an $a$ for which $\NOT P(a)$ is true is just saying
that $\exists x(\NOT P(x))$ is true.  So, the truth of $\NOT(\forall x P(x))$
implies the truth of $\exists x (\NOT P(x))$.  On the other hand, if 
$\NOT(\forall x P(x))$ is false, then $\forall x P(x)$ is true.  Since $P(x)$
is true for every~$x$, $\NOT P(x)$ is false for every~$x$; that is, there is no
element $a$ for which the statement $\NOT P(a)$ is true.
But this just means that the statement $\exists x(\NOT P(x))$
is false.  In any case, then, the truth values of $\NOT(\forall x P(x))$ and
$\exists x(\NOT P(x))$ are the same.  Since this is true for any predicate $P$,
we will say that these two formulas are logically equivalent and write
$\NOT(\forall x P(x)) \equiv \exists x(\NOT P(x))$.

\fig
  {F-predlogic}
  {Four important rules of predicate logic.  $P$ can be any one-place predicate,
   and $Q$ can be any two-place predicate.  The first two rules are called
   DeMorgan's Laws for predicate logic.}
  {\begin{tabular}{|c|}
     \hline
     \strut$\NOT\,(\forall x P(x)) \equiv \exists x(\NOT P(x))$\\
     \hline
     \strut$\NOT\,(\exists x P(x)) \equiv \forall x(\NOT P(x))$\\
     \hline
     \strut$\forall x \forall y Q(x,y) \equiv \forall y \forall x Q(x,y)$\\
     \hline
     \strut$\exists x \exists y Q(x,y) \equiv \exists y \exists x Q(x,y)$\\
     \hline
  \end{tabular}
 }

A similar argument would show that $\NOT(\exists x P(x)) \equiv \forall x(\NOT P(x))$.
These two equivalences, which explicate the relation between negation and quantification,
are known as DeMorgan's Laws\index{DeMorgan's Laws} for predicate logic.  (They are closely related to
DeMorgan's Laws for propositional logic; see the exercises.)  These
laws can be used to help simplify expressions.  For example, 
\begin{align*}
  \NOT\,\forall y (R(y)\lor Q(y)) &\equiv \exists y(\NOT(R(y)\lor Q(y)))\\
       &\equiv \exists y((\NOT R(y))\land(\NOT Q(y))\\
\end{align*}
It might not be clear exactly why this qualifies as a ``simplification,''
but it's generally considered simpler to have the negation operator applied
to basic propositions such as $R(y)$, rather than to quantified expressions
such as \hbox{$\forall y (R(y)\lor Q(y))$}.
For a more complicated example:
\begin{align*}
  \NOT\,\exists x\big(P(x)&\land (\forall y (Q(y)\IMP Q(x)))\big)\\
            &\equiv\forall x\big(\NOT\big(P(x)\land (\forall y (Q(y)\IMP Q(x)))\big)\\
            &\equiv\forall x\big((\NOT P(x))\lor (\NOT \forall y (Q(y)\IMP Q(x)))\big)\\
            &\equiv\forall x\big((\NOT P(x))\lor (\exists y(\NOT (Q(y)\IMP Q(x))))\big)\\
            &\equiv\forall x\big((\NOT P(x))\lor (\exists y(\NOT (\NOT Q(y)\lor Q(x))))\big)\\
            &\equiv\forall x\big((\NOT P(x))\lor (\exists y(\NOT\NOT Q(y)\land \NOT Q(x)))\big)\\
            &\equiv\forall x\big((\NOT P(x))\lor (\exists y(Q(y)\land \NOT Q(x)))\big)\\
\end{align*}
DeMorgan's Laws are listed in Figure~\ref{F-predlogic} along with two
other laws of predicate logic.  The other laws allow you to interchange
the order of the variables when two quantifiers of the same type
(both $\exists$ or $\forall$) occur together. 

To define logical equivalence in predicate logic more formally,
we need to talk about formulas that contain predicate variables,\index{variable}
that is, variables that act as place-holders for arbitrary predicates
in the same way that propositional variables are place-holders for
propositions and element variables are place-holders for
elements.  With this in mind, we can define logical equivalence
and the closely related concept of tautology for predicate logic.

\begin{definition}
Let $\mathscr{P}$ be a formula of predicate logic which contains one or more
predicate variables.  $\mathscr{P}$~is said to be a \nw{tautology}
if it is true whenever all the predicate variables that it contains are replaced
by actual predicates.  Two formulas $\mathscr{P}$ and $\mathscr{Q}$ are
said to be \nw[logical equivalence!in predicate logic]{logically equivalent} if $\mathscr{P}\IFF\mathscr{Q}$ is
a tautology, that is if $\mathscr{P}$ and $\mathscr{Q}$ always have the same
truth value when the predicate variables they contain are replaced by actual
predicates.  The notation $\mathscr{P}\equiv\mathscr{Q}$ asserts that
$\mathscr{P}$ is logically equivalent to $\mathscr{Q}$.
\end{definition}




\begin{exercises}

\problem Simplify each of the following propositions.  In your answer, the
$\NOT$ operator should be applied only to individual predicates.
\pparts{
   \NOT\,\forall x (\NOT P(x)) & \NOT\,\exists x(P(x)\land Q(x))\cr
   \NOT \,\forall z(P(z)\IMP Q(z))&
      \NOT\big((\forall x P(x))\land \forall y(Q(y))\big) \cr
   \NOT\, \forall x \exists y P(x,y)&
      \NOT\,\exists x (R(x)\land \forall y S(x,y))\cr
    \NOT\,\exists y(P(y)\IFF Q(y))&
       \NOT \big(\forall x (P(x)\IMP (\exists y Q(x,y)))\big) \cr
}

\problem Give a careful argument to show that the second of DeMorgan's laws for
predicate calculus,
$\NOT(\forall x P(x)) \equiv \exists x(\NOT P(x))$, is valid.

\problem Find the negation of each of the following propositions.
Simplify the result; in your answer, the
$\NOT$ operator should be applied only to individual predicates.
\ppart $\NOT$$\exists n (\forall s C(s,n))$
\ppart $\NOT$$\exists n (\forall s (L(s,n) \IMP P(s)))$
\ppart $\NOT$$\exists n (\forall s (L(s,n) \IMP (\exists x \exists y \exists z Q(x,y,z))))$.
\ppart $\NOT$$\exists n (\forall s (L(s,n) \IMP (\exists x \exists y \exists z (s=xyz \land 
R(x,y) \land T(y) \land U(x,y,z))))$.

\problem Suppose that the domain for a predicate $P$
contains only two elements.  Show that $\forall x P(x)$ is equivalent to
a conjunction of two simple propositions, and $\exists x P(x)$ is equivalent
to a disjunction.  Show that in this case, DeMorgan's Laws for propositional
logic and DeMorgan's Laws for predicate logic actually say exactly the same
thing.  Extend the results to a domain that contains exactly
three elements.

\problem Let $H(x)$ stand for ``$x$ is happy,'' where the domain 
consists of people.  Express the proposition ``There are exactly three happy
people'' in predicate logic.

\problem Let $T(x,y)$ stand for ``$x$~has taken~$y$,'' where the
domain for $x$ consists of students and the domain
for $y$ consists of math courses (at your school).
Translate each of the following propositions into an unambiguous English sentence:
\pparts{ \forall x\,\forall y \,T(x,y) & \forall x \,\exists y \,T(x,y) & \forall y \,\exists x \,T(x,y)\cr
         \exists x\,\exists y \,T(x,y) & \exists x \,\forall y \,T(x,y) & \exists y \,\forall x \,T(x,y)\cr
}

\problem Let $F(x,t)$ stand for ``You can fool person $x$ at time~$t$.''
Translate the following sentence into predicate logic:
``You can fool some of the people all of the time, and you can fool
all of the people some of the time, but you can't fool all of the
people all of the time.''

\problem Translate each of the following sentences into a proposition 
using predicate logic.  Make up any predicates you need.  State what
each predicate means and what its domain is.
\ppart All crows are black.
\ppart Any white bird is not a crow.
\ppart Not all politicians are honest.
\ppart All green elephants have purple feet.
\ppart There is no one who does not like pizza.
\ppart Anyone who passes the final exam will pass the course.
\ppart If $x$ is any positive number, then there is a number $y$ such that
$y^2=x$.

\problem The sentence ``Someone has the answer to every question'' is
ambiguous.  Give two translations of this sentence into predicate logic,
and explain the difference in meaning.

\problem The sentence ``Jane is looking for a dog'' is ambiguous.
One meaning is that there is some particular dog---maybe the one she lost---that
Jane is looking for.  The other meaning is that Jane is looking for any old
dog---maybe because she wants to buy one.  Express the first meaning in
predicate logic.  Explain why the second meaning is \emph{not}
expressed by $\forall x(Dog(x)\IMP LooksFor(jane,x))$.  In fact, the
second meaning cannot be expressed in predicate logic.  Philosophers
of language spend a lot of time thinking about things like this.
They are especially fond of the sentence ``Jane is looking for a unicorn,''
which is not ambiguous when applied to the real world.  Why is that?



\end{exercises}




