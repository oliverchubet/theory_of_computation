\section{Application: Programming with Functions}\label{S-sets-5}

Functions are fundamental in computer programming,\index{function!in computer programming}
although not everything in programming that goes by the name of ``function''
is a function according to the mathematical definition.

In computer programming, a function is a routine that is given 
some data as input and that will calculate and return an
answer based on that data.  For example, in the C++ programming
language, a function that calculates the square of an integer
could be written
\begin{verbatim}
            int square(int n) {
               return n*n;
            }
\end{verbatim}
In C++, \textit{int} is a data type.  From the mathematical
point of view, a data type is a set.  The data type \textit{int}
is the set of all integers that can be represented as 32-bit 
binary numbers.  Mathematically, then, $\textit{int}\SUB\Z$.
(You should get used to the fact that sets and functions can
have names that consist of more than one character, since
it's done all the time in computer programming.)
The first line of the above function definition,
``\verb=int square(int n)='', says that we are defining
a function named square whose range is \textit{int}
and whose domain is \textit{int}.  In the usual notation for
functions, we would express this as $\textit{square}\colon \textit{int}\to\textit{int}$,
or possibly as $\textit{square}\in{\textit{int}}^{\textit{\small{int}}}$,
where ${\textit{int}}^{\textit{\small{int}}}$ is the set of all
functions that map the set \textit{int} to the set \textit{int}.

The first line of the function, \verb=int square(int n)=, is called
the \nw{prototype} of the function.  The prototype specifies the
name, the domain, and the range of the function and so carries
exactly the same information as the notation ``$f\colon A\to B$''.
The ``$n$'' in ``\verb=int square(int n)='' is a name for
an arbitrary element of the data type \textit{int}.  In computer
jargon, $n$ is called a \nw{parameter} of the function.
The rest of the definition of \textit{square} tells the computer
to calculate the value of $\textit{square}(n)$ for any $n\in\textit{int}$
by multiplying $n$ times~$n$.  The statement ``\verb=return n*n=''
says that $n*n$ is the value that is computed, or ``returned,''
by the function.  (The $*$ stands for multiplication.)

C++ has many data types in addition to \textit{int}.  There is
a boolean data type named \textit{bool}.  The values of type
\textit{bool} are \textit{true} and \textit{false}.  Mathematically,
\textit{bool} is a name for the set $\{\textit{true},\,\textit{false}\}$.
The type \textit{float} consists of real numbers, which can
include a decimal point.  Of course, on a computer, it's not
possible to represent the entire infinite set of real numbers,
so \textit{float} represents some subset of the mathematical set
of real numbers.  There is also a data type whose values are
strings of characters, such as ``Hello world'' or ``xyz152QQZ''.
The name for this data type in C++ is \textit{string}.  All these
types, and many others, can be used in functions.  For example,
in C++, $m\,\%\,n$ is the remainder when the integer $m$ is
divided by the integer $n$.  We can define a function to test
whether an integer is even as follows:
\begin{verbatim}
            bool even(int k) {
               if ( k % 2 == 1 )
                  return false;
               else
                  return true;
            }
\end{verbatim}
You don't need to worry about all the details here, but you should
understand that the prototype, \verb=bool even(int k)=,
says that \textit{even} is a function from the set \textit{int}
to the set \textit{bool}.  That is,
$\textit{even}\colon\textit{int}\to\textit{bool}$.  Given
an integer $N$, $\textit{even}(N)$ has the value \textit{true}
if $N$ is an even integer, and it has the value \textit{false}
if $N$ is an odd integer.

A function can have more than one parameter.  For example, we might
define a function with prototype \verb=int index(string str, string sub)=.
If $s$ and $t$ are strings, then \textit{index}$(s,t)$ would be the
\textit{int} that is the value of the function at the ordered pair
$(s,t)$.  We see that the domain of index is the cross product
$\textit{string}\times\textit{string}$, and we can write
$\textit{index}\colon \textit{string}\times\textit{string}\to\textit{int}$
or, equivalently, $\textit{index}\in\textit{int}^{\textit{string}\times\textit{string}}$.

\medbreak

Not every C++ function is actually a function in the mathematical
sense.  In mathematics, a function must associate a single value in
its range to each value in its domain.  There are two things
that can go wrong:  The value of the function might not be defined
for every element of the domain, and the function might associate
several different values to the same element of the domain.
Both of these things can happen with C++ functions.

In computer programming, it is very common for a ``function'' to be
undefined for some values of its parameter.  In mathematics,
a \nw{partial function}\index{function!partial} from a set $A$ to
a set $B$ is defined to be a function from a subset of $A$ to~$B$.
A partial function from $A$ to $B$ can be undefined for some
elements of $A$, but when it is defined for some $a\in A$,
it associates just one element of $B$ to~$a$.  Many functions
in computer programs are actually partial functions.  (When 
dealing with partial functions, an ordinary function, which is
defined for every element of its domain, is sometimes referred to
as a \nw{total function}.  Note that---with the mind-boggling
logic that is typical of mathematicians---a total function is
a type of partial function, because a set is a subset of itself.)

It's also very common for a ``function'' in a computer program
to produce a variety of values for the same value of its parameter.
A common example is a function with prototype
\verb=int random(int N)=, which returns a random integer between
1 and~$N$.  The value of \textit{random}(5) could be 1, 2, 3, 4, or~5.
This is not the behavior of a mathematical function!

Even though many functions in computer programs are not really
mathematical functions, I will continue to refer to them as
functions in this section.  Mathematicians will just have to stretch
their definitions a bit to accommodate the realities of computer
programming.

\medbreak

In most programming languages, functions are not first-class
objects\index{first-class object}.  That is, a function cannot
be treated as a data value in the same way as a \textit{string}
or an \textit{int}.  However, C++ does take a step in this
direction.  It is possible for a function to be a parameter
to another function.  For example, consider the function
prototype
\begin{verbatim}
            float sumten( float f(int) )
\end{verbatim}
This is a prototype for a function named \textit{sumten} whose
parameter is a function.  The parameter is specified by the
prototype ``\verb=float f(int)=''.  This means that the parameter
must be a function from \textit{int} to \textit{float}.  The parameter
name, $f$, stands for an arbitrary such function.  Mathematically,
$f\in \textit{float}^{\textit{int}}$, and so
$\textit{sumten}\colon \textit{float}^{\textit{int}}\to\textit{float}$.

My idea is that \textit{sumten}($f$) would compute
$f(1)+f(2)+\cdots+f(10)$.  A more useful function would
be able to compute $f(a)+f(a+1)+\cdots+f(b)$ for any integers
$a$ and~$b$.  This just means that $a$ and $b$ should be
parameters to the function.  The prototype for the improved
function would look like
\begin{verbatim}
            float sum( float f(int), int a, int b )
\end{verbatim}
The parameters to \textit{sum} form an ordered triple in which
the first coordinate is a function and the second and third
coordinates are integers.  So, we could write
\[\textit{sum}\colon \textit{float}^{\textit{int}}
       \times\textit{int}\times\textit{int}\to\textit{float}\]
It's interesting that computer programmers deal routinely
with such complex objects.

\medskip

One thing you can't do in C++ is write a function that creates
new functions from scratch.  The only functions that exist
are those that are coded into the source code of the program.
There are programming languages that do allow new functions to
be created from scratch while a program is running.  In such
languages, functions are first-class objects.  These languages
support what is called \nw{functional programming}.  

One of the most accessible languages that supports functional programming
is JavaScript, a language that is used on Web pages.  (Although
the names are similar, JavaScript and Java are only distantly
related.)  In JavaScript, the function that computes the
square of its parameter could be defined as
\begin{verbatim}
            function square(n) {
               return n*n;
            }
\end{verbatim}
This is similar to the C++ definition of the same function, but
you'll notice that no type is specified for the parameter $n$ or
for the value computed by the function.  Given this definition
of \text{square}, \textit{square}($x$) would be legal for any
$x$ of any type.  (Of course, the value of \textit{square}($x$)
would be undefined for most types, so \textit{square} is
a \emph{very} partial function, like most functions in JavaScript.)
In effect, all possible data values in JavaScript are bundled
together into one set, which I will call \textit{data}.
We then have $\textit{square}\colon \textit{data}\to \textit{data}$.\footnote{Not
all functional programming languages lump data types together
in this way.  There is a functional programming language
named Haskell, for example, that is as strict about types as C++.
For information about Haskell, see http://www.Haskell.org/.}

In JavaScript, a function really is a first-class object.  We can 
begin to see this by looking at an alternative definition of the
function \textit{square}:
\begin{verbatim}
            square = function(n) { return n*n; }
\end{verbatim}
Here, the notation ``\verb=function(n) { return n*n; }='' creates
a function that computes the square of its parameter, but it doesn't
give any name to this function.  This function object is then
assigned to a variable named \textit{square}.  The value of
\textit{square} can be changed later, with another assignment
statement, to a different function or even to a different type
of value.  This notation for creating function objects can
be used in other places besides assignment statements.  Suppose,
for example, that a function with prototype
\verb=function sum(f,a,b)= has been defined in a JavaScript
program to compute $f(a)+f(a+1)+\cdots+f(b)$.  Then
we could compute $1^2+2^2+\cdots+100^2$ by saying
\begin{verbatim}
            sum( function(n) { return n*n; }, 1, 100 )
\end{verbatim}
Here, the first parameter is the function that computes
squares.  We have created and used this function
without ever giving it a name.

It is even possible in JavaScript for a function to return
another function as its value.  For example,
\begin{verbatim}
            function monomial(a, n) {
               return ( function(x) { a*Math.pow(x,n); } );
            }
\end{verbatim}
Here, \verb=Math.pow(x,n)= computes $x^n$, so for any
numbers $a$ and $n$, the value of \textit{monomial}($a$,$n$) is 
a function that computes $ax^n$.  Thus,
\begin{verbatim}
            f = monomial(2,3);
\end{verbatim}
would define $f$ to be the function that satisfies $f(x)=2x^3$,
and if \textit{sum} is the function described above, then
\begin{verbatim}
            sum( monomial(8,4), 3, 6 )
\end{verbatim}
would compute $8*3^4+8*4^4+8*5^4+8*6^4$.  In fact, \textit{monomial}
can be used to create an unlimited number of new functions
from scratch.  It is even possible to write \textit{monomial}(2,3)(5)
to indicate the result of applying the function \textit{monomial}(2,3)
to the value 5.  The value represented by \textit{monomial}(2,3)(5)
is $2*5^3$, or 250.  This is real functional programming and
might give you some idea of its power.



\begin{exercises}

\problem For each of the following C++ function prototypes, translate the
prototype into a standard mathematical function specification, such
as $\textit{func}\colon\textit{float}\to\textit{int}$.
\tparts {
   \texttt{int strlen(string s)}\cr
   \texttt{float pythag(float x, float y)}\cr
   \texttt{int round(float x)}\cr
   \texttt{string sub(string s, int n, int m)}\cr
   \texttt{string unlikely( int f(string) )}\hidewidth\cr
   \texttt{int h( int f(int), int g(int) )}\hidewidth\cr
}

\problem Write a C++ function prototype for a function that
belongs to each of the following sets.

\smallskip

\pparts {
   \textit{string}^{\textit{string}}\cr\noalign{\smallskip}
   \textit{bool}^{\textit{float}\times\textit{float}}\cr\noalign{\smallskip}
   \textit{float}^{ \textit{int}^{\textit{int}} }\cr
}

\problem It is possible to define new types in C++.  For example, the
definition
\begin{verbatim}
               struct point {
                  float x;
                  float y;
               }
\end{verbatim}
defines a new type named \textit{point}.  A value of type \textit{point}
contains two values of type \textit{float}.  What mathematical operation
corresponds to the construction of this data type?  Why?

\problem Let \textit{square}, \textit{sum} and \textit{monomial}
be the JavaScript functions described in this section.  What is the
value of each of the following?
\tparts {
   \textit{sum}(\textit{square}, 2, 4)\cr
   \textit{sum}(\textit{monomial}(5,2), 1, 3)\cr
   \textit{monomial}(\textit{square}(2), 7)\cr
   \textit{sum}(function($n$) $\{$ return $2*n$; $\}$, 1, 5)\cr
   \textit{square}(\textit{sum}(\textit{monomial}(2,3), 1, 2))
}


\problem Write a JavaScript function named \textit{compose}
that computes the composition of two functions.  That
is, \textit{compose}($f$,$g$) is $f\circ g$, where
$f$ and $g$ are functions of one parameter.  Recall that
$f\circ g$ is the function defined by $(f\circ g)(x)=f(g(x))$.

\end{exercises}






