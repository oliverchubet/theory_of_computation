\section{The Limits of Computation}\label{S-turing-3}

Recursively enumerable languages are languages that can be defined by computation.
We have seen that there are many different models of compu\-tation---Turing machines,
two-tape Turing machines, grammars, computer programs---but they all lead
to the same class of languages.  In fact, every computational method for
specifying languages that has ever been developed produces only recursively
enumerable languages.  There is something about these languages---some pattern
or property---that makes them ``computable,''  and it is some intrinsic
property of the languages themselves, not some peculiarity of any given
model of computation.

This is especially interesting since most languages are not recursively enumerable.
Given an alphabet $\Sigma$, there are uncountably many languages over $\Sigma$, but
only countably many of them are recursively enumerable.  The rest---the vast
majority---are not recursively enumerable.  What can we say about
all these non-recursively-enumerable languages?  If the language $L$ is not
recursively enumerable, then there is no algorithm for listing the members of
$L$.  It might be possible to define $L$ by specifying some property that
all its members satisfy, but that property can't be computable.  That is, there
can be no computer program or Turing machine that tests whether a given
string $w$ has the property, since if there were, then we could write a
program that lists the members of $L$.

So, even though almost every language is non-recursively-enumerable, it's
difficult to find a particular language that is not recursively enumerable.
Nevertheless, in this section we will find one such language.  At that same
time, we will find an example of a language that is recursively enumerable
but not recursive.  And we will discover some interesting limitations to
the power of computation.

\medskip

The examples that we will look at in this section involve Turing
machines that work with other Turing machines as data.  For this to
work, we need a symbolic representation of Turing machines---a
representation that can be written on the tape of another Turing
machine.  This will let us create two machines:  First,
a Turing machine that can generate Turing machines
on demand by writing their symbolic representations on its tape.
We will design a Turing machine $G$ to do this.  And second,
a Turing machine that can simulate the computation of other
Turing machines whose descriptions are written on its tape.


In order to do all this, we must put some limitations on
the states and alphabetic symbols that can be used in the Turing machines
that we consider.
Clearly, given any Turing machine, we can change the names of the
states without changing the behavior of the machine.  So, without any
loss of generality, we can assume that all states have names chosen
from the list: $h$, $q$, $q'$, $q''$, $q'''$, $q''''$,~\dots.
We assume that $h$ is the halt state and $q$ is the start state.
Note that there is an infinite number of possible states, but any
given Turing machine will only use finitely many states from this
list.  

As for the alphabets of the Turing machines, I want to look at
Turing machines whose alphabets include the symbols 0, 1, $a$, and of
course~\b.
These are the symbols that the machines will use for input and output.
The alphabets can also include other symbols.  We will assume that
these auxiliary symbols are chosen from the list: $a'$, $a''$, $a'''$,
$a''''$,~\dots.  Given a Turing machine whose alphabet includes
the symbols 0, 1, $a$, and \b, we can rename any other symbols in its
alphabet using names from this list.  This renaming will not
affect any of the behavior that we are interested in.

Now suppose we have one of these standard Turing machines---one
whose states are chosen from the list $h$, $q$, $q'$, $q''$, $q'''$,~\dots,
whose start state is $q$, and whose symbols are chosen from the list
\b, 0, 1, $a$, $a'$, $a''$, $a'''$,~\dots.  Such a machine can be
completely encoded as a string of symbols over the alphabet
$\{h,q,L,R,\b,0,1,a,{}',\text{\tt\$}\}$.  A transition rule
such as $\delta(q'',0)=(a''',L,q)$ can be encoded as a
string $q''0a'''Lq$.  To encode a complete machine, simply encode
each of its transition rules in this way and join them together in a string,
separated by {\tt\$}'s.  We now have the symbolic representation for
Turing machines that we need.

Note that a string over the alphabet $\{h,q,L,R,\b,0,1,a,{}',\text{\$}\}$
might or might not encode a Turing machine.  However, it is a simple
matter to check whether such a string is the code for a Turing machine.
We can imagine the following process:  Generate all the strings over the
alphabet $\{h,q,L,R,\b,0,1,a,{}',\text{\$}\}$.  Check each string
to see whether it encodes a Turing machine.  If so, add the string
to an output list. In this way, we can generate a list of
all strings that encode standard Turing machines.  In effect,
the standard Turing machines, or at least their symbolic representations,
form a recursively enumerable set.  Let $T_0$ be the
machine encoded by the first string in this list of
standard Turing machines; let $T_1$ be
the machine encoded by the second string; let $T_2$ be the
machine encoded by the third string; and so on.  The list
$T_0$, $T_1$, $T_2$,~\dots, includes every standard Turing machine.
Furthermore, given $n\in\N$, we can find the symbolic representation
for $T_n$ by generating strings in the list until we have $n+1$ strings.
Furthermore---and this is the essential point---we can use a Turing
machine to do all these calculations.  In fact, there is
a Turing machine that, when run with input $a^n$, will halt with
the string representation of $T_n$ written on its tape as output.
The Turing machine that does this is $G$, the first of the
two machines that we need.

The second machine that we need will be called $U$.  It is a
so-called \nw{Universal Turing Machine}.  The single Turing machine
$U$ can simulate the computation of any standard Turing machine, $T$,
on any input.  Both the symbolic representation of $T$ and that of
the input string are written to $U$'s tape, separated by a
space.  As $U$ simulates the computation of $T$, it will need
some way to keep track of what state $T$ is in and of
the position of $T$ on its (simulated) tape.  It does this
by writing the current state of $T$ on its tape, following
$T$'s input string, and by adding a special symbol, such as \at,
to the input string to mark $T$'s position.  When $U$ is first
started, it begins by adding the \at\ to the beginning of the
input string and writing a $q$ after the string to represent
the start state of $T$.  It is then relatively straightforward
for $U$ to simulate the computation of $T$.  For each step
in the computation of $T$, it can determine the current state
of $T$ (which is recorded on $U$'s tape) and the symbol which
$T$ is currently reading (which is on $U$'s tape, after the \at).
$U$ searches the symbolic representation of $T$ for the
rule that tells $T$ what to do in this situation.  Using
this rule, $U$ can update its representation of $T$'s state,
position, and tape to reflect the result of applying the rule.
If the new state of $T$ is the halt state, then $U$ also halts.
Otherwise, it goes on to simulate the next step in $T$'s computation.
Note that when $U$ is given $T$ and an input string $w$ as
input, $U$ will halt if and only if $T$ halts on input $w$.
(Obviously, this is a very inefficient simulation, but we
are not concerned with efficiency here.)

So, we have our two machines, $G$ and $U$.
After all this setup, we are finally in a position to look at
the major theorem that we have been working towards.

\begin{theorem}
Let $T_0$, $T_1$, $T_2$, \dots, be the standard Turing machines,
as described above.  Let $K$ be the language over the alphabet $\{a\}$
defined by $$K=\{a^n\st\,T_n\text{ halts when run with input }a^n\}.$$
Then $K$ is a recursively enumerable language, but $K$ is not
recursive.  The complement $$\overline{K}=\{a^n\st\,T_n\text{ does
not halt when run with input }a^n\}.$$
is a language that is not recursively enumerable.
\end{theorem}

First note that if both $K$ and $\overline{K}$ were recursively
enumerable, then $K$ would be recursive, by Theorem~\ref{T-re}.
So, once we show that $K$ is recursively enumerable but not
recursive, it follows immediately that $\overline{K}$ cannot
be recursively enumerable.  That is, the second part of the
theorem follows from the first.

To show that $K$ is recursively enumerable, it suffices to find
a Turing machine, $M$, that accepts $K$.  That is, when run
with input $a^n$, for $n\in\N$, $M$ should halt if and only if
$a^n\in K$.  We can build $M$ from the Turing machines $G$ and $U$
which were introduced above.  When started with input $a^n$, 
$M$ should proceed as follows:  
First copy the input.  Run $G$ on the
first copy of $a^n$.  This will produce a symbolic description
of the Turing machine $T_n$.  Now run $U$ to simulate the 
computation of $T_n$ on input $a^n$.  This simulation will end
if and only if $T_n$ halts when run with input $a^n$, that is, if and only
if $a^n\in K$.  The Turing machine $M$ that performs the computation
we have described accepts the language $K$.
This proves that $K$ is recursively enumerable.

To show that $K$ is not recursive, we need to show that
there is \textit{no} Turing machine that decides $K$.  
Let $H$ be any Turing machine.  We must show that
no matter what $H$ does, it
does not decide the language $K$.  We must do this without
knowing anything more about $H$ that the fact that is it
a Turing machine.  To say that $H$ decides
$K$ would mean that for any $n\in\N$, when $H$ is
run with input $a^n$, $H$ will halt with output 1 if $a^n\in K$
and will halt with output 0 if $a^n\not\in K$.  To show that
$H$ does not decide $K$ we need to show that there is some
$n\in\N$ such that when $H$ is run with input $a^n$, 
$H$ either fails to halt or else halts but gives the wrong 
output.  Note in particular that we only need to find \textit{one}
$n$ for which $H$ does not give the correct result.
As we try to find $n$, we have nothing much to work with
but $H$ itself.\looseness=-1

To find $n$, we construct a Turing machine $M$ that is a simple
variation on $H$.  When $M$ is run on any input, it duplicates the behavior
of $H$ on that input until $H$ halts (if it ever does).  At that point, $M$ should
check $H$'s output.  If $H$ has halted with output $1$, then 
$M$ should go into an infinite loop, so that $M$ never halts in this case.
Otherwise, if the output of $H$ is not $1$, then $M$ should halt.
Now, we can assume that $M$ is one
of the standard Turing machines, say $M=T_n$.  (If $M$ is not
already one of these machines, it is because it uses different names
for its states and symbols.  Renaming the states and symbols will
produce an equivalent machine with the same behavior as $M$,
and we can replace $M$ with this standard machine.)  

We now have a Turing machine $T_n = M$ which has the following behavior
when it is run with input $a^n$ (note that the $n$ here is the same $n$ as
in $T_n$):
If $H$ halts with output 1 on input $a^n$, then $T_n$
will fail to halt on input $a^n$.  If $H$ halts with output 0
on input $a^n$, then $T_n$ fails to halt on input $a^n$.  (What $T_n$
might do in other cases is not relevant here.)

Remember that we are trying to show that $H$ does not decide the language
$K$.  I claim that, in fact, $H$ does not give the correct answer for $a^n$.  When $H$
is run with input $a^n$, it is supposed to halt with output 1 if $a^n\in K$,
and it is supposed to halt with output 0 if $a^n\not\in K$.  Recall that
$a^n\in K$ if and only if $T_n$ halts when run with input $a^n$.

Suppose that we run $H$ with input $a^n$.
If $H$ does not halt with output 0 or 1, then it has certainly not given the 
correct answer for $a^n$.  Now, suppose that $H$ halts with output 1 on input $a^n$.
In this case, by the properties of $T_n$ given above, we know that $T_n$ does not
halt on input $a^n$.  But that means, by definition of $K$, 
that $a^n\not\in K$.  By halting with output
1 in this case, $H$ has given the wrong answer for $a^n$.  Finally, suppose that
$H$ halts with output 0 on input $a^n$.  We then know that $T_n$ halts
on input $a^n$.  But that means that $a^n\in K$.  Again, by halting with output
0 in this case, $H$ has given the wrong answer for $a^n$.  So, in no case will
$H$ give the correct answer for $a^n$.  
This means that
$H$ does not decide the language $K$, because $H$ gives
an incorrect answer when it is run with the particular input $a^n$.
$H$ does not decide $K$, and since
$H$ was an arbitrary Turing machine, we see that there is
no Turing machine at all that decides the language $K$.  Thus,
$K$ is not a recursive language, as the theorem claims. 

\medbreak

To decide the language $K$ would be to solve the following
problem:  Given a Turing machine $T_n$, decide whether or
not $T_n$ will halt when it is run with input $a^n$.  This
problem is called the \nw{Halting Problem}.  We have shown
that there is no Turing machine that solves this problem.
Given the equivalence of Turing machines and computer programs,
we can also say that there is no computer program that
solves the halting problem.  We say that the halting problem
is \nw[computational unsolvability]{computationally unsolvable}.

The halting problem is just one
of many problems that cannot be solved by Turing machines or
computer programs.  In fact, almost any interesting yes/no
question that can be asked about Turing machines or programs
is in this class:  Does this Turing machine halt for all possible
inputs in $\Sigma^*$?  Given this input, will this program
ever halt?  Do these two programs (or Turing machines) have
the same output for each possible input?  Will this Turing
machine ever halt if it is started on a blank tape?
All these problems are computationally unsolvable in the
sense that there is no Turing machine or computer program
that will answer them correctly in all cases. The
existence of such problems is a real limitation on the
power of computation.

