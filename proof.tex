\section{Proof}\label{S-proof}
Mathematics is unique in that it claims a certainty
that is beyond all possible doubt or argument.  A mathematical proof
shows how some result follows by logic alone from a given set of
assumptions, and once the result has been proven, it is as solid as
the foundations of logic themselves.
Of course, mathematics achieves this certainty by restricting itself
to an artificial, mathematical world, and its application to the
real world does not carry the same degree of certainty.

Within the world of mathematics, consequences follow from assumptions
with the force of logic, and a proof is just a way of pointing out
logical consequences.  There is an old mathematical joke about this:

This mathematics professor walks into class one day and says
``We'll start today with this result, which is obvious,'' and
he writes it on the board.  Then, he steps back and looks at the
board for a while.  He walks around the front of the room, stares
into space and back at the board.  This goes on till the end of
class, and he walks out without saying anything else.  The next
class period, the professor walks into the room with a big smile,
writes the same result on the board, turns to the class and
says, ``I was right.  It is obvious.''\index{obviousness}

For of course, the fact that mathematical results follow logically
does not mean that they are obvious in any normal sense.  Proofs are
convincing once they are discovered, but finding them is often
very difficult.  They are written in a language and style
that can seem obscure to the uninitiated.
Often, a proof builds on a long series of definitions
and previous results, and while each step along the way might be
``obvious,'' the end result can be surprising and powerful.
This is what makes the search for proofs worthwhile.

In the rest of this chapter, we'll look at some approaches and techniques
that can be used for proving mathematical results, including two
important proof techniques known as proof by contradiction and
mathematical induction.  Along the way, we'll encounter a few new
definitions and notations.  Hopefully, you will be left with
a higher level of confidence for exploring the mathematical world
on your own.

\medskip

One of the most important pieces of advice to keep in mind is,
``Use the definition.''  In the world of mathematics, terms
mean exactly what they are defined to mean and nothing more.
Definitions allow very complex ideas to be summarized as
single terms.  When you are trying to prove things about
those terms, you generally need to ``unwind'' the definitions.
When you are trying to prove something about equivalence relations
in Chapter~\ref{C-sets},
you can be pretty sure that you will need to use the
fact that equivalence relations, by definition, are symmetric, reflexive,
and transitive.  (And, of course,
you'll need to know how the term ``relation'' is defined
in the first place!
You'll get nowhere if you work from the idea that ``relations'' are something
like your aunt and uncle.)

More advice along the same line is to check whether you are
using the assumptions of the theorem.  An assumption that
is made in a theorem is called an \nw{hypothesis}.  The hypotheses
of the theorem state conditions whose truth will guarantee the
conclusion of the theorem.  To prove the theorem means to assume
that the hypotheses are true, and to show, under that assumption,
that the conclusion must be true.  It's likely (though not
guaranteed) that you will need to use the hypotheses explicitly 
at some point in the proof, as we did in our example above.\footnote{Of 
course, if you set out to
discover new theorems on your own, you aren't given the hypotheses
and conclusion in advance, which makes things quite a bit harder---and
more interesting.}  Also, you should keep in mind that any
result that has already been proved is available to be used
in your proof.

\medbreak

A proof is a logical argument, based on the rules of logic.
Since there are really not all that many basic rules of logic,
the same patterns keep showing up over and over.  Let's look
at some of the patterns.

The most common pattern arises in the attempt to prove that
something is true ``for all'' or ``for every'' or ``for any''
element in a given category.  In terms of logic, the statement
you are trying to prove is of the form $\forall x\,P(x)$.
In this case, the most likely way to begin the proof is
by saying something like, ``Let $x$ be an arbitrary element in 
the domain.  We want to show that $P(x)$.''  In the
rest of the proof, $x$ refers to some unspecified but definite
element in the domain.  Since $x$ is arbitrary,
proving $P(x)$ amounts to proving $\forall x\,P(x)$.  You only
have to be careful that you don't use any facts about $x$ beyond
what you have assumed.  For example, in our proof above, we cannot
make any assumptions about the integer $n$ except that it is
even; if we had made such assumptions, then the proof would have
been incorrect, or at least incomplete.

Sometimes, you have to prove that an element exists that satisfies
certain stated properties.  Such a proof is called an
\nw{existence proof}.  In this case, you are attempting to
prove a statement of the form $\exists x\,P(x)$.  The way to
do this is to find an example, that is, to find a specific
element $a$ for which $P(a)$ is true.  One way to prove
the statement ``There is an even prime number'' is to find
a specific number that satisfies this description.  
The same statement could also
be expressed ``Not every prime number is odd.''  This statement
has the form $\NOT(\forall x\,P(x))$, which is equivalent
to the statement $\exists x\,(\NOT P(x))$.  
An example that proves the statement $\exists x\,(\NOT P(x))$
also proves $\NOT(\forall x\,P(x))$.  Such an example is
called a \nw{counterexample} to the statement $\forall x\,P(x)$:
A counterexample proves that the statement $\forall x\,P(x)$ is false.
The number 2 is a counterexample to the statement ``All prime numbers
are odd.''  In fact, 2 is the only counterexample to this
statement; many statements have multiple counterexamples.

Note that we have now discussed how to prove and disprove
universally quantified statements, and how to prove existentially
quantified statements.  How do you {\em disprove} $\exists x\,P(x)$?
Recall that $\NOT \exists x\,P(x)$ is logically equivalent to
$\forall x\,(\NOT P(x))$, so to disprove $\exists x\,P(x)$ you need
to prove $\forall x\,(\NOT P(x))$.

Many statements, like that in our example above, 
have the logical form of an implication, $p\IMP q$.
(More accurately, they
are of the form ``$\forall x\, (P(x) \IMP Q(x))$", but as discussed above
the
strategy for proving such a statement is to prove $P(x) \IMP Q(x)$ 
for an arbitrary element $x$ of the
domain.)  The statement
might be ``For all natural numbers $n$, if $n$ is even then $n^2$ is even,'' or ``For
all strings $x$, if $x$ is in the language $L$ then $x$ is generated by the grammar $G$,''
or ``For all elements $s$, if $s \in A$ then $s \in B$.''  Sometimes the implication is
implicit rather than explicit: for example, ``The sum of two rationals is rational'' is
really short for ``For any numbers $x$ and $y$, if $x$ and $y$ are rational then $x+y$
is rational.''
A proof of such a statement often begins something like this:
``Assume that $p$.  We want to show that $q$.''  In the rest of
the proof, $p$ is treated as an assumption that is known to be
true.  %As discussed above, the logical reasoning behind this is that 
%you are essentially proving that 
%\begin{center}
%\argument{$p$}{$q$}
%\end{center}
%is a valid argument.
Another way of thinking about it is to remember
that $p\IMP q$ is
automatically true in the case where $p$ is false, so there is no
need to handle that case explicitly.  In the remaining case, when $p$ is
true, we can show that $p\IMP q$ is true by showing that the truth of $q$
follows from the truth of $p$.

A statement of the form $p\AND q$ can be proven by proving
$p$ and $q$ separately.  A statement of the form $p\OR q$
can be proved by proving the logically equivalent statement
$(\NOT p)\IMP q$: to prove
$p\OR q$, you can assume that $p$ is false and prove, under
that assumption, that $q$ is true.  For example, the
statement ``Every integer is either even or odd'' is
equivalent to the statement ``Every integer that is not even
is odd.''

Since $p\IFF q$ is equivalent
to $(p\IMP q)\AND(q\IMP p)$, a statement of the form $p\IFF q$
is often proved by giving two proofs, one of
$p\IMP q$ and one of $q\IMP p$.  In English,
$p\IFF q$ can be stated in several forms such as
``$p$ if and only if $q$'', ``if $p$ then $q$ and conversely,''
and ``$p$ is necessary and sufficient for $q$.''  The phrase
``if and only if'' is so common in mathematics that it is
often abbreviated \nw{iff}.

You should also keep in mind that you can prove $p\IMP q$
by displaying a chain of valid implications $p\IMP r\IMP s\IMP \cdots\IMP q$.
Similarly, $p\IFF q$ can be proved with a chain of valid
biconditionals $p\IFF r\IFF s\IFF \cdots\IFF q$.

\medbreak

%We'll turn to a few examples, but first  
%here is some terminology that we will use throughout our sample proofs:
%
%\begin{itemize}
%
%\item The \nw{natural numbers} (denoted $\N$) are the numbers $0,1,2,\ldots$.  Note that the
%sum and product of natural numbers are natural numbers.
%
%\item The \nw{integers} (denoted $\Z$) are the numbers $0, -1, 1, -2, 2, -3, 3, \ldots$.
%Note that the sum, product, and difference of integers are integers.
%
%\item The \nw{rational numbers} (denoted $\Q$)
%are all numbers that can be written in the form $\frac{m}{n}$
%where $m$ and $n$ are integers and $n\not=0$.  So $\frac{1}{3}$ and $\frac{-65}{7}$ are
%rationals; so, less obviously, are 6 and $\frac{\sqrt{27}}{\sqrt{12}}$ since $6=\frac{6}{1}$
%(or, for that matter, $6=\frac{-12}{-2}$), and $\frac{\sqrt{27}}{\sqrt{12}} = 
%\sqrt{\frac{27}{12}} = \sqrt{\frac{9}{4}} = \frac{3}{2}$.  Note the restriction that the
%number in the denominator cannot be 0: $\frac{3}{0}$ is not a number at all, rational
%or otherwise; it is an undefined quantity.  Note also that the sum, product, difference,
%and quotient of rational numbers are rational numbers (provided you don't attempt to divide
%by~0.)
%
%\item The \nw{real numbers} (denoted $\R$) are numbers
%that can be written in decimal form, possibly with an infinite number of
%digits after the decimal point.  Note that the sum, product, difference,
%and quotient of real numbers are real numbers (provided you don't attempt to divide
%by~0.)
%
%\item The \nw{irrational numbers} are real numbers that are not rational, i.e.\ that cannot
%be written as a ratio of integers.  Such numbers include $\sqrt{3}$ (which we will
%prove is not rational) and $\pi$ (if anyone ever told you that $\pi = \frac{22}{7}$,
%they lied---$\frac{22}{7}$ is only an {\em approximation} of the value of $\pi$).
%
%\item An integer $n$ is \nw{divisible by $m$} iff $n=mk$ for some integer $k$. (This can also
%be expressed by saying that $m$ {\nw evenly divides} $n$.) So
%for example, $n$ is divisible by $2$ iff $n=2k$ for some integer $k$; $n$ is divisible
%by 3 iff $n=3k$ for some integer $k$, and so on.  Note that if $n$ is {\em not} divisible
%by 2, then $n$ must be 1 more than a multiple of 2 so $n=2k+1$ for some integer $k$.
%Similarly, if $n$ is not divisible by 3 then $n$ must be 1 or 2 more than a multiple of 3,
%so $n=2k+1$ or $n=2k+2$ for some integer $k$.
%
%\item An integer is \nw{even} iff it is divisible by 2 and \nw{odd} iff it is not.
%
%\item An integer $n>1$ is \nw{prime} if it is divisible by exactly two positive integers, namely 1 and itself.
%Note that a number must be greater than 1 to even have a chance of being termed ``prime''.
%In particular, neither 0 nor 1 is prime.
%
%\end{itemize}
%
%
%\medskip
%
%Let's look now at another example: prove that the sum of any two rational numbers is
%rational. 
%\begin{proof}
%We start by assuming that $x$ and $y$ are arbitrary rational numbers.
%Here's a formal proof that the inference rule
%\begin{center}
%\argument{$x$ is rational \\ $y$ is rational}{$x+y$ is rational}
%\end{center}
%is a valid rule of inference:
%
%\breakSixByNine
%
%\begin{center}
%\begin{tabular}{r@{\ \ }l@{\qquad}l}
%1.&$x$ is rational                                                 & premise\\
%2.&if $x$ is rational, then $x=\frac{a}{b}$                        & \\
%  & \ \ \ for some integers $a$ and $b\not=0$                      & definition of rationals \\
%3.&$x=\frac{a}{b}$ for some integers $a$ and $b\not=0$             & from 1,2 (\textit{modus ponens}) \\
%4.&$y$ is rational                                                 & premise\\
%5.&if $y$ is rational, then $y=\frac{c}{d}$ for                    & \\
%  & \ \ \ some integers $c$ and $d\not=0$                          & definition of rational\\
%6.&$y=\frac{c}{d}$ for some $c$ and $d\not=0$                      & from 4,5 (\textit{modus ponens})\\
%7.&$x=\frac{a}{b}$ for some $a$ and $b\not=0$ and                  & \\
%  & \ \ \ $y=\frac{c}{d}$ for some $c$ and $d\not=0$               & from 3,6\\
%8.&if $x=\frac{a}{b}$ for some $a$ and $b\not=0$ and               & \\
%  & \ \ \ $y=\frac{c}{d}$ for $c$ and $d\not=0$ then               & \\
%  & \ \ \ $x+y = \frac{ad+bc}{bd}$where $a,b,c,d$                  & \\
%  & \ \ \ are integers and $b,d \not=0$                            & basic algebra\\
%9.&$x+y = \frac{ad+bc}{bd}$ for some $a,b,c,d$                     & \\
%  & \ \ \ where $b,d \not=0$                                       & from 7,8 (\textit{modus ponens})\\
%10.&if $x+y = \frac{ad+bc}{bd}$ for some $a,b,c,d$                 & \\
%  & \ \ \  where $b,d \not=0$ then $x+y = \frac{m}{n}$             & \\
%  & \ \ \  where $m,n$ are integers and $n\not= 0$                 & properties of integers\\
%11.&$x+y = \frac{m}{n}$ where $m$ and $n$                          & \\
%   & \ \ \ are integers and $n\not= 0$                             & from 9,10 (\textit{modus ponens})\\
%12.&if  $x+y = \frac{m}{n}$ where $m$ and $n$ are                  & \\
%   & \ \ \ integers and $n\not= 0$                                 & \\
%   & \ \ \ then $x+y$ is rational                                  & definition of rational\\
%13.&$x+y$ is rational                                              & from 11,12 (\textit{modus ponens})\\
%\end{tabular}
%\end{center}
%So the rule of inference given above is valid.
%Since $x$ and $y$ are arbitrary rationals, we have proved that the rule is valid for all
%rationals, and hence the sum of any two rationals is rational.
%\end{proof}
%
%Again, a more informal presentation would look like: 
%\begin{proof}
%Let $x$ and $y$ be arbitrary rational
%numbers.  By the definition of rational, there are integers $a,b\not=0,c,d\not=0$ such
%that $x=\frac{a}{b}$ and $y=\frac{c}{d}$.  Then $x+y = \frac{ad+bc}{bd}$; we know 
%$ad+bc$ and $bd$ are integers since the sum and product of integers are integers, and
%we also know $bd\not=0$ since neither $b$ nor $d$ is 0.  So we have written
%$x+y$ as the ratio of two integers, the denominator being non-zero.  Therefore, by
%the definition of rational numbers, $x+y$ is rational.  Since $x$ and $y$ were arbitrary
%rationals, the sum of any two rationals is rational. 
%\end{proof}
%
%\medskip
%
%And one more example: we will prove that any 4-digit number $d_1d_2d_3d_4$ is 
%divisible by 3 iff the sum of the four digits is divisible by 3.
%
%\begin{proof}
%This statement is of the form $p \IFF q$; recall that $p \IFF q$
%is logically equivalent to $(p\IMP q) \wedge
%(q \IMP p)$.  So we need to prove for any 4-digit number $d_1d_2d_3d_4$ that (1) if
%$d_1d_2d_3d_4$ is divisible by~3 then $d_1+d_2+d_3+d_4$ is divisible by~3, and (2)
%if $d_1+d_2+d_3+d_4$ is divisible by~3 then $d_1d_2d_3d_4$ is divisible by~3.
%So let $d_1d_2d_3d_4$ be an arbitrary 4-digit number.
%
%(1) Assume $d_1d_2d_3d_4$ is divisible by 3, i.e. $d_1d_2d_3d_4=3k$ for some integer
%$k$.  The number $d_1d_2d_3d_4$ is actually $d_1 \times 1000 + d_2 \times 100 +
%d_3 \times 10 + d_4$, so we have the equation 
%$$d_1 \times 1000 + d_2 \times 100 +
%d_3 \times 10 + d_4 = 3k.$$  
%Since $1000=999+1$, $100=99+1$, and $10=9+1$, this
%equation can be rewritten 
%$$999d_1 + d_1 + 99d_2 + d_2 +9d_3 + d_3 + d_4 = 3k.$$
%Rearranging gives
%\begin{align*}
%d_1 + d_2 +d_3 +d_4 &= 3k - 999d_1 - 99d_2 - 9d_3 \\
%                    &= 3k - 3(333d_1) - 3(33d_2) - 3(3d_3).
%\end{align*}
%We can now factor a 3
%from the right side to get 
%$$d_1 + d_2 +d_3 +d_4 = 3(k - 333d_1 - 33d_2 - d_3).$$
%Since $(k - 333d_1 - 33d_2 - d_3)$ is an integer, we have shown that $d_1 + d_2 +d_3 +d_4$
%is divisible by 3.
%
%(2) Assume $d_1 + d_2 +d_3 +d_4$
%is divisible by 3. Consider the number $d_1d_2d_3d_4$.  As remarked above,
%$$d_1d_2d_3d_4 = d_1 \times 1000 + d_2 \times 100 +
%d_3 \times 10 + d_4$$ 
%so 
%\begin{align*}d_1d_2d_3d_4 &= 999d_1 + d_1 + 99d_2 + d_2 + 9d_3 + d_3 + d_4\\
%                             &= 999d_1 + 99d_2 + 9d_3 + (d_1 + d_2 +d_3 +d_4).
%\end{align*} 
%We assumed that
%$d_1 + d_2 +d_3 +d_4 = 3k$ for some integer $k$, so we can substitute into the
%last equation to get 
%$$d_1d_2d_3d_4 = 999d_1 + 99d_2 + 9d_3 + 3k = 3(333d_1 +
%33d_2 + 3d_3 + k).$$  
%Since the quantity in parentheses is an integer, we have
%proved that $d_1d_2d_3d_4$ is divisible by 3.
%
%In (1) and (2) above, the number $d_1d_2d_3d_4$ was an arbitrary 4-digit integer,
%so we have proved that  for all 4-digit integers,
%$d_1d_2d_3d_4$ is  divisible by 3 iff the sum of the four digits is divisible by 3.
%\end{proof}
%
%
%\medskip
%
%Now suppose we wanted to prove the statement ``For all integers $n$, $n^2$ is even if
%and only if $n$ is even.''  We have already proved half of this
%statement (``For all integers $n$, if $n$ is even then $n^2$ is even''), so
%all we need to do is prove the statement
%``For all integers $n$, if $n^2$ is even then $n$ is even'' and we'll be done.
%Unfortunately, this is not as straightforward as it seems: suppose we started
%in our standard manner and let $n$
%be an arbitrary integer and assumed that $n^2=2k$ for some integer $k$.  Then we'd be
%stuck!  Taking the square root of both sides would give us $n$ on the left but would
%leave a $\sqrt{2k}$ on the right. This quantity is not of the form $2k'$ for any
%integer $k'$; multiplying it by $\frac{\sqrt{2}}{\sqrt{2}}$ would give $2\frac{\sqrt{k}}
%{\sqrt{2}}$ but there is no way for us to prove that $\frac{\sqrt{k}}
%{\sqrt{2}}$ is an integer.  So we've hit a dead end.  What do we do now?
%
%The answer is that we need a different proof technique.  The proofs we have written
%so far are what are called \nw{direct proofs}: to prove $p \IMP q$ you assume
%$p$ is true and prove that the truth of $q$ follows.
%Sometimes, when a direct proof of $p \IMP q$ fails, an \nw{indirect proof} will
%work.  Recall that the {\em contrapositive} of the implication $p \IMP q$
%is the implication $\NOT q \IMP \NOT p$, and that this proposition is
%logically equivalent to $p \IMP q$. An indirect proof of 
%$p \IMP q$, then, is a direct proof of the contrapositive 
%$\NOT q \IMP \NOT p$.  In our current example, instead of proving
%``if $n^2$ is even then $n$ is even" directly, we can prove its contrapositive
%``if $n$ is not even (i.e. $n$ is odd) then $n^2$ is not even (i.e. $n^2$ is odd.)''
%The proof of this contrapositive is a routine direct argument which we leave to the
%exercises.
%
%\begin{exercises}
%
%\problem Find a natural number $n$ for which $n^2+n+41$ is not prime.
%
%\problem Show that the propositions $p\OR q$ and $(\NOT p)\IMP q$
%are logically equivalent.
%
%\problem Show that the proposition $(p\OR q)\IMP r$ is equivalent
%to $(p\IMP r)\AND(q\IMP r)$.
%%  Explain how this
%%fact is used in the proof of Theorem~\ref{T-image}.
%
%\problem Determine whether each of the following statements is
%true.  If it true, prove it.  If it is false, give a counterexample.
%\ppart Every prime number is odd.
%\ppart Every prime number greater than 2 is odd.
%\ppart If $x$ and $y$ are integers with $x<y$, then there is an integer
%$z$ such that $x<z<y$.
%\ppart If $x$ and $y$ are real numbers with $x<y$, then there is a real number
%$z$ such that $x<z<y$.
%
%\problem Suppose that $r$, $s$, and $t$ are integers, such that $r$ evenly divides $s$ and
%$s$ evenly divides $t$.  Prove that $r$ evenly divides $t$.
%
%\problem Prove that for all integers $n$, if $n$ is odd then $n^2$ is odd.
%
%\problem\label{divby3}Prove that an integer $n$ is divisible by 3 iff $n^2$ is divisible
%by 3.  (Hint: give an indirect proof of ``if $n^2$ is divisible by 3 then
%$n$ is divisible by 3.'')
%
%\problem Prove or disprove each of the following statements.
%\ppart The product of two even integers is even.
%\ppart The product of two integers is even only if both integers are even.
%\ppart The product of two rational numbers is rational.
%\ppart The product of two irrational numbers is irrational.
%\ppart For all integers $n$, if $n$ is divisible by 4 then $n^2$ is
%divisible by 4.
%\ppart For all integers $n$, if $n^2$ is divisible by 4 then $n$ is
%divisible by 4.
%
%\end{exercises}
%
%
%
