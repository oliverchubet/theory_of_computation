\section{Finite-State Automata and Regular Languages}\label{S-fsa-3}

We know now that our two models for mechanical language recognition actually
recognize the same class of languages.  The question still remains: do they
recognize the same class of languages as the class generated mechanically by regular
expressions?  The answer turns out to be ``yes".  There are two parts to proving
this: first that every language generated can be recognized, and second that
every language recognized can be generated.

\begin{theorem}\label{retonfa}
Every language generated by a regular expression can be recognized by an NFA.
\end{theorem}

\begin{proof} The proof of this theorem is a nice example of a proof by induction on
the structure of regular expressions.  The definition of regular expression is
inductive: $\Phi$, $\varep$, and $a$ are the simplest regular expressions,
and then more complicated regular expressions can be built from these.  We will
show that there are NFAs that accept the languages generated by the simplest
regular expressions, and then show how those machines can be put together to
form machines that accept languages generated by more complicated regular
expressions.

Consider the regular expression $\Phi$.  $L(\Phi) = \{\}$.  Here is a machine
that accepts $\{\}$: 

\fsafig{12}

Consider the regular expression $\varep$.  $L(\varep) = \{\varepsilon\}$.  
Here is a machine that accepts $\{\varepsilon\}$:

\fsafig{13}

Consider the regular expression $a$.  $L(a) = \{a\}$.  Here is a
machine that accepts $\{a\}$:

\fsafig{14}

Now suppose that you have NFAs that accept the languages generated by the
regular expressions $r_1$ and $r_2$.  Building a machine that accepts $L(r_1 \REOR 
r_2)$ is fairly straightforward: take an NFA $M_1$ that accepts $L(r_1)$ and an
NFA $M_2$ that accepts $L(r_2)$.  Introduce a new state $q_{new}$, connect
it to the start states of $M_1$ and $M_2$ via $\varepsilon$-transitions, and
designate it as the start state of the new machine.  No other transitions are
added.  The final states of $M_1$ together with the final states of $M_2$ are
designated as the final states of the new machine.  It should be fairly clear
that this new machine accepts exactly those strings accepted by $M_1$ together
with those strings accepted by $M_2$: any string $w$ that was accepted by $M_1$
will be accepted by the new NFA by starting with an $\varep$-transition to the
old start state of $M_1$ and then following the accepting path through $M_1$;
similarly, any string accepted by $M_2$ will be accepted by the new machine;
these are the only strings that will be accepted by the new machine, as on any
input $w$ all the new machine can do is make an $\varep$-move to $M_1$'s (or
$M_2$'s) start state, and from there $w$ will only be accepted by the new
machine if it is accepted by $M_1$ (or $M_2$).  Thus, the new machine accepts
$L(M_1) \cup L(M_2)$, which is $L(r_1) \cup L(r_2)$, which is exactly the
definition of $L(r_1 \REOR  r_2)$.

\fsafig{15}

(A pause before we continue: note that for the simplest regular expressions,
the machines that we created to accept the languages generated by the regular
expressions were in fact DFAs.  In our last case above, however, we needed
$\varep$-transitions to build the new machine, and so if we were trying to
prove that every regular language could be accepted by a DFA, our proof would
be in trouble.  THIS DOES NOT MEAN that the statement ``every regular language
can be accepted by a DFA" is false, just that we can't prove it using this kind
of argument, and would have to find an alternative proof.)

Suppose you have machines $M_1$ and $M_2$ that accept $L(r_1)$ and $L(r_2)$
respectively.  To build a machine that accepts $L(r_1)L(r_2)$ proceed as
follows.  Make the start state $q_{01}$ of $M_1$ be the start state of the new
machine.  Make the final states of $M_2$ be the final states of the new machine.
Add $\varep$-transitions from the final states of $M_1$ to the start state
$q_{02}$ of
$M_2$.

\fsafig{16}

It should be fairly clear that this new machine accepts exactly those strings of
the form $xy$ where $x\in L(r_1)$ and $y \in L(r_2)$: first of all, any string
of this form will be accepted because $x\in L(r_1)$ implies there is a path that
consumes $x$ from
$q_{01}$ to a final state of $M_1$; a $\varep$-transition moves to $q_{02}$; 
then $y \in L(r_2)$ implies there is a path that consumes $y$ from $q_{02}$ to a
final state of $M_2$; and the final states of $M_2$ are the final states of the
new machine, so $xy$ will be accepted.  Conversely, suppose $z$ is accepted by
the new machine.  Since the only final states of the new machine are in the old
$M_2$, and the only way to get into $M_2$ is to take a $\varep$-transition from
a final state of $M_1$, this means that $z=xy$ where $x$ takes the machine from
its start state to a final state of $M_1$, a $\varep$-transition occurs, and
then $y$ takes the machine from $q_{02}$ to a final state of $M_2$.  Clearly,
$x\in L(r_1)$ and $y \in L(r_2)$. 

We leave the construction of an NFA that accepts $L(r^*)$ from an NFA that 
accepts $L(r)$ as an exercise.

\end{proof}

\smallskip

\begin{theorem}\label{T-DFAeqReg}
Every language that is accepted by a DFA or an NFA is generated by a regular 
expression.
\end{theorem}

Proving this result is actually fairly involved and not very illuminating. 
Before presenting a proof, we will give an illustrative example of how one
might actually go about extracting a regular expression from an NFA or a DFA.
You can go on to read the proof if you are interested.

\begin{example}
Consider
the DFA shown below:

\fsafig{17}

Note that there is a loop from state $q_2$ back to state $q_2$: any number of
$a$'s will keep the machine in state $q_2$, and so we label the transition with
the regular expression $a^*$.  We do the same thing to the transition labeled
$b$ from $q_0$.  (Note that the result is no longer a DFA, but that doesn't
concern us, we're just interested in developing a regular expression.)

\fsafig{18}

Next we note that there is in fact a loop from $q_1$ to $q_1$ via $q_0$.  A
regular expression that matches the strings that would move around the loop is
$ab^*a$.  So we add a transition labeled $ab^*a$ from $q_1$ to
$q_1$, and remove the now-irrelevant $a$-transition from $q_1$ to $q_0$.  (It is
irrelevant because it is not part of any other loop from $q_1$ to 
$q_1$.)

\fsafig{19}
  
Next we note that there is also a loop from $q_1$ to $q_1$ via $q_2$.  A
regular expression that matches the strings that would move around the loop is
$ba^*b$.  Since the transitions in the loop are the only transitions to or from
$q_2$, we simply remove $q_2$ and replace it with a transition from $q_1$ to
$q_1$.

\fsafig{20}

It is now clear from the diagram that strings of the form $b^*a$ get you to
state $q_1$, and any number of repetitions of strings that match $ab^*a$ or
$ba^*b$ will keep you there.  So the machine accepts $L(b^*a(ab^*a\REOR ba^*b)^*)$. 
\end{example}

%It is a fact that every DFA or NFA is equivalent to an NFA whose start state is
%not accepting, and whose final states number exactly one.  Any such machine can
%be massaged, pictorially speaking, into a "machine" that has exactly two
%states---a non-accepting start state and an accepting second state---and whose
%transition arcs are labeled by regular expressions.  The idea is perhaps best
%illustrated by means of an example:   ***** check these claims



\begin{proof}[Proof of Theorem~\ref{T-DFAeqReg}]
We prove that the language accepted by a DFA is regular.  The proof for NFAs
follows from the equivalence between DFAs and NFAs.

Suppose that $M$ is a DFA, where $M=(Q,\Sigma,q_0,\delta,F)$.  Let $n$ be the
number of states in $M$, and write $Q=\{q_0,q_1,\dots,q_{n-1}\}$.  We want
to consider computations in which $M$ starts in some state $q_i$, reads a string
$w$, and ends in state $q_k$.  In such a computation, $M$ might go through a
series of intermediates states between $q_i$ and $q_k$:
$$q_i\longrightarrow p_1\longrightarrow p_2 \cdots\longrightarrow p_r\longrightarrow q_k$$
We are interested in computations in which all of the intermediate states---$p_1,p_2,\dots,p_r$---are
in the set $\{q_0,q_1,\dots,q_{j-1}\}$, for some number~$j$.
We define $R_{i,j,k}$ to be the set of all strings $w$ in $\Sigma^*$ that are consumed
by such a computation.  That is, $w\in R_{i,j,k}$ if and only if when $M$ starts in state
$q_i$ and reads $w$, it ends in state $q_k$, and all the intermediate states between
$q_i$ and $q_k$ are in the set $\{q_0,q_1,\dots,q_{j-1}\}$.
$R_{i,j,k}$ is a language over $\Sigma$.  We show that $R_{i,j,k}$ for
$0\le i < n$, $0\le j \le n$, $0\le k < n$.

Consider the language $R_{i,0,k}$.  For $w\in R_{i,0,k}$, the set of allowable intermediate
states is empty.  Since there can be no intermediate states,
it follows that there can be at most one step in the computation that
starts in state $q_i$, reads $w$, and ends in state $q_k$.  So, $|w|$ can be at most one.
This means that $R_{i,0,k}$ is finite, and hence is regular.  (In fact,
$R_{i,0,k}=\{a\in\Sigma\st \delta(q_i,a)=q_k\}$, for $i\ne k$, and
$R_{i,0,i}=\{\varep\}\cup\{a\in\Sigma\st \delta(q_i,a)=q_i\}$.  Note that in many
cases, $R_{i,0,k}$ will be the empty set.)

We now proceed by induction on $j$ to show that $R_{i,j,k}$ is regular for all $i$ and $k$.
We have proved the base case, $j=0$.  Suppose that $0\le j< n$ we already know that $R_{i,j,k}$
is regular for all $i$ and all $k$.  We need to show that $R_{i,j+1,k}$ is regular for all $i$ and $k$.
In fact, 
$$R_{i,j+1,k}=R_{i,j,k}\cup \left( R_{i,j,j}R_{j,j,j}^*R_{j,j,k}\right)$$
which is regular because $R_{i,j,k}$ is regular for all $i$ and $k$, and because the union, concatenation,
and Kleene star of regular languages are regular.

To see that the above equation holds, consider a string $w\in\Sigma^*$.
Now, $w\in R_{i,j+1,k}$ if and only if when $M$ starts in state $q_i$ and reads $w$,
it ends in state $q_k$, with all intermediate states in the computation in the set
$\{q_0,q_1,\dots,q_j\}$.  Consider such a computation.  There are two
cases: Either $q_j$ occurs as an intermediate state in the computation, or it does not.
If it does \textbf{not} occur, then all the intermediate states are in the set
$\{q_0,q_1,\dots,q_{j-1}\}$, which means that in fact $w\in R_{i,j,k}$.
If $q_j$ \textbf{does} occur as an intermediate state in the computation, then we can break the
computation into phases, by dividing it at each point where $q_j$ occurs
as an intermediate state.  This breaks $w$ into a concatenation $w=xy_1y_2\cdots y_rz$.
The string $x$ is consumed in the first phase of the computation, during which $M$
goes from state $q_i$ to the first occurrence of $q_j$; since the intermediate states
in this computation are in the set $\{q_0,q_1,\dots,q_{j-1}\}$, $x\in R_{i,j,j}$.
The string $z$ is consumed by the last phase of the computation, in which $M$
goes from the final occurrence of $q_j$ to $q_k$, so that $z\in R_{j,j,k}$.
And each string $y_t$ is consumed in a phase of the computation in which $M$ goes
from one occurrence of $q_j$ to the next occurrence of $q_j$, so that $y_r\in R_{j,j,j}$.
This means that $w=xy_1y_2\cdots y_rz\in R_{i,j,j}R_{j,j,j}^*R_{j,j,k}$.

We now know, in particular, that $R_{0,n,k}$ is a regular language for all $k$.
But $R_{0,n,k}$ consists of all strings $w\in\Sigma^*$ such that when $M$ starts
in state $q_0$ and reads $w$, it ends in state $q_k$ (with \textbf{no} restriction
on the intermediate states in the computation, since every state of $M$ is in
the set $\{q_0,q_1,\dots,q_{n-1}\}$).
To finish the proof that $L(M)$ is regular, it is only necessary to note that
$$L(M)=\bigcup_{q_k\in F} R_{0,n,k}$$
which is regular since it is a union of regular languages.
This equation is true since
a string $w$ is in $L(M)$ if and only if when $M$ starts in state $q_0$ and reads $w$,
in ends in some accepting state $q_k\in F$. This is the same as saying
$w\in R_{0,n,k}$ for some $k$ with $q_k\in F$.
\end{proof}


\bigskip

We have already seen that if two languages $L_1$ and $L_2$ are
regular, then so are $L_1 \cup L_2$, $L_1L_2$, and $L_1^*$ 
(and of course $L_2^*$).  We have not yet seen, however, how the 
common
set operations intersection and complementation affect regularity.
Is the complement of a regular language regular?  How about the
intersection of two regular languages?

Both of these questions can be answered by thinking of regular
languages in terms of their acceptance by DFAs.  Let's consider
first the question of complementation.  Suppose we have an arbitrary
regular language $L$.  We know there is a DFA $M$ that accepts $L$.
Pause a moment and try to think of a modification that you could make
to $M$ that would produce a new machine $M'$ that accepts $\overline{L}$....
Okay, the obvious thing to try is to make $M'$ be a copy of $M$ 
with all final states of $M$ becoming non-final states of $M'$ and
vice versa.  This is in fact exactly right: $M'$ does in fact accept
$\overline{L}$.  To verify this, consider an arbitrary string $w$.  The
transition functions for the two machines $M$ and $M'$ are identical, so $\dstar
(q_0, w)$ is the same state in both $M$ and $M'$; if that state is
accepting in $M$ then it is non-accepting in $M'$, so if $w$ is
accepted by $M$ it is not accepted by $M'$; if the state is
non-accepting in $M$ then it is accepting in $M'$, so if $w$ is
not accepted by $M$ then it is accepted by $M'$. Thus $M'$ accepts
exactly those strings that $M$ does not, and hence accepts $\overline{L}$.  

It is worth pausing for a moment and looking at the above argument
a bit longer.  Would the argument have worked if we had looked at an
arbitrary language $L$ and an arbitrary $N$FA $M$ that accepted $L$?
That is, if we had built a new machine $M'$ in which the final and
non-final states had been switched, would the new NFA $M'$ accept
the complement of the language accepted by $M$?  The answer is
``not necessarily".  Remember that acceptance in an NFA is determined
based on whether or not at least one of the states reached by a
string is accepting.  So any string $w$ with the property that
$\pstar(q_0, w)$ contains both accepting and non-accepting states of $M$
would be accepted both by $M$ and by $M'$.

Now let's turn to the question of intersection.  Given two regular
languages $L_1$ and $L_2$, is $L_1 \cap L_2$ regular?  Again, it is
useful to think in terms of DFAs: given machines $M_1$ and $M_2$
that accept $L_1$ and $L_2$, can you use them to build a new
machine that accepts $L_1 \cap L_2$? The answer is yes, and the
idea behind the construction bears some resemblance to that behind
the NFA-to-DFA construction.  
We want a new machine where transitions reflect the transitions
of both $M_1$ and $M_2$ simultaneously, and we want to accept a
string $w$ only if that those sequences of transitions lead to 
final states in both $M_1$ and $M_2$. So we associate the
states of our new machine $M$ with pairs of states from $M_1$
and $M_2$.  For each state $(q_1,q_2)$ in the new machine and input symbol $a$,
define $\delta((q_1,q_2),a)$ to be the state 
$(\delta_1(q_1,a), \delta_2(q_2,a))$.
The start state $q_0$ of $M$ is
$(q_{01}, q_{02})$, where $q_{0i}$ is the start state
of $M_i$.  The final states of $M$ are the the states of the form $(q_{f1},
q_{f2})$ where $q_{f1}$ is an accepting state of $M_1$ and $q_{f2}$ is an
accepting state of $M_2$.  You should convince yourself that $M$ accepts a
string $x$ iff $x$ is accepted by both $M_1$ and $M_2$.

The results of the previous section and the preceding discussion are summarized
by the following theorem:

\begin{theorem}\label{closure} 
The intersection  of two
regular languages is a regular language.  

The union of two
regular languages is a regular language.  

The concatenation of two
regular languages is a regular language.  

The complement of a regular language is a regular language.

The Kleene closure of a regular language is a regular language.
\end{theorem}
 

\begin{exercises}
\problem Give a DFA that accepts the intersection of the languages accepted by
the machines shown below.  (Suggestion: use the construction discussed in the
chapter just before Theorem~\ref{closure}.)

\fsafig{5ex}

\problem Complete the proof of Theorem~\ref{retonfa} by showing how to modify a
machine that accepts $L(r)$ into a machine that accepts $L(r^*)$.
\problem Using the construction described in Theorem~\ref{retonfa}, build an NFA
that accepts $L((ab\REOR a)^*(bb))$.
\problem Prove that the reverse of a regular language is regular.
\problem Show that for any DFA or NFA, there is an NFA with exactly one final
state that accepts the same language.
\problem Suppose we change the model of NFAs to allow NFAs to have multiple
start states.  Show that for any ``NFA" with multiple start states, there is an
NFA with exactly one start state that accepts the same language.
\problem Suppose that $M_1=(Q_1,\Sigma,q_1,\delta_1,F_1)$ and 
$M_2=(Q_2,\Sigma,q_2,\delta_2,F_2)$ are DFAs over the alphabet $\Sigma$.  It is possible
to construct a DFA that accepts the langauge $L(M_1)\cap L(M_2)$ in a single step.
Define the DFA $$ M = (Q_1\times Q_2, \Sigma, (q_1,q_2), \delta, F_1\times F_2)$$
where $\delta$ is the function from $(Q_1\times Q_2)\times\Sigma$ to $Q_1\times Q_2$
that is defined by: $\delta((p1,p2),\sigma))=(\delta_1(p_1,\sigma),\delta_2(p_2,\sigma))$.
Convince yourself that this definition makes sense.  (For example, note that
states in $M$ are pairs $(p_1,p_2)$ of states, where $p_1\in Q_1$ and $p_2\in Q_2$,
and note that the start state $(q_1,q_2)$ in $M$ is in fact a state in $M$.)
Prove that $L(M)=L(M_1)\cap L(M_2)$, and explain why this shows that the intersection of
any two regular languages is regular.  This proof---if you can get past the
notation---is more direct than the one outlined above.

\end{exercises}


