\section{Counting Past Infinity}\label{S-sets-6}

As children, we all learned to answer the question ``How many?''
by counting with numbers: 1, 2, 3, 4,~\dots.  But the question of
``How many?'' was asked and answered long before the abstract concept of
number was invented.  The answer can be given in terms of ``as many as.''
How many cousins do you have?  As many cousins as I have fingers on both
hands.  How many sheep do you own?  As many sheep as there are notches
on this stick.  How many baskets of wheat must I pay in taxes?
As many baskets as there are stones in this box.  The question
of how many things are in one collection of objects is answered
by exhibiting another, more convenient, collection of objects
that has just as many members.  

In set theory, the idea of one set having just as many members
as another set is expressed in terms of \nw{one-to-one
correspondence}.  A one-to-one correspondence between two sets
$A$ and $B$ pairs each element of $A$ with an element of $B$
in such a way that every element of $B$ is paired with one and
only one element of~$A$.  The process of counting, as it is learned
by children, establishes a one-to-one correspondence between a set of 
$n$ objects and the set of numbers from 1 to~$n$.  The rules
of counting are the rules of one-to-one correspondence:
Make sure you count every object, make sure you don't count
the same object more than once.  That is, make sure that
each object corresponds to \emph{one} and \emph{only one} number.
Earlier in this chapter,
we used the fancy name ``bijective function''\index{bijective function}
to refer to this idea, but we can now see it as
as an old, intuitive way of answering the question ``How many?''

\medbreak

In counting, as it is learned in childhood, the set $\{1,2,3,\dots,n\}$
is used as a typical set that contains $n$ elements.  In mathematics
and computer science, it has become more common to start counting with
zero instead of with one, so we define the following sets to use
as our basis for counting:
\begin{center}
\begin{tabular}{@{\qquad\qquad}ll}
   $N_0=\emptyset$,& a set with 0 elements\\
   $N_1=\{0\}$,& a set with 1 element\\
   $N_2=\{0,1\}$,& a set with 2 elements\\
   $N_3=\{0,1,2\}$,& a set with 3 elements\\
   $N_4=\{0,1,2,3\}$,& a set with 4 elements\\
\end{tabular}
\end{center}
\noindent and so on.  In general, $N_n=\{0,1,2,\dots,n-1\}$ for each $n\in\N$.
For each natural number $n$, $N_n$ is a set with $n$ elements.
Note that if $n\not= m$, then there is no one-to-one correspondence
between $N_n$ and $N_m$.  This is obvious, but like many obvious things
is not all that easy to prove rigorously, and we omit the argument here. 

\begin{theorem}
For each $n\in\N$, let $N_n$ be the set $N_n=\{0,1,\dots,n-1\}$.
If $n\not=m$, then there is no bijective function from $N_m$ to $N_n$.
\end{theorem}


We can now make the following definitions:

\begin{definition}
A set $A$ is said to be \nw[finite set]{finite} if there is a one-to-one
correspondence between $A$ and $N_n$ for some natural number~$n$.  We then
say that $n$ is the \nw{cardinality} of~$A$.  The notation $|A|$ is
used to indicate the cardinality of~$A$.  That is, if $A$ is a finite set,
then $|A|$ is the natural number $n$ such that there is a one-to-one
correspondence between $A$ and $N_n$.  A set that is not finite is
said to be \nw[infinite set]{infinite}.  That is, a set $B$ is infinite
if for every $n\in \N$, there is \emph{no} one-to-one correspondence between
$B$ and~$N_n$.
\end{definition}

Fortunately, we don't always have to count every element in a set individually
to determine its cardinality.  Consider, for example, the set $A\times B$,
where $A$ and $B$ are finite sets.  If we already know $|A|$ and~$|B|$,
then we can determine $|A\times B|$ by computation, without explicit counting of elements.
In fact, $|A\times B|=|A|\cdot |B|$.  The cardinality of the cross product
$A\times B$ can be computed by multiplying the cardinality of $A$ by
the cardinality of $B$.  To see why this is true, think of how you might
count the elements of $A\times B$.  You could put the elements into piles,
where all the ordered pairs in a pile have the same first coordinate.  There
are as many piles as there are elements of $A$, and each pile contains
as many ordered pairs as there are elements of~$B$.  That is, there
are $|A|$ piles, with $|B|$ items in each.  By the definition of multiplication,
the total number of items in all the piles is $|A|\cdot|B|$.
A similar result holds for the cross product of more that two finite sets.
For example, $|A\times B\times C|=|A|\cdot |B|\cdot |C|$.

It's also easy to compute $|A\cup B|$ in the case where $A$ and $B$ are
disjoint finite sets.  (Recall that two sets $A$ and $B$ are said to be
disjoint if they have no members in common, that is, if $A\cap B=\emptyset$.)
Suppose $|A|=n$ and $|B|=m$.  If we wanted to count the elements of $A\cup B$,
we could use the $n$ numbers from 0 to $n-1$ to count the elements of $A$
and then use the $m$ numbers from $n$ to $n+m-1$ to count the elements of
$B$.  This amounts to a one-to-one correspondence between $A\cup B$
and the set $N_{n+m}$.  We see that $|A\cup B|=n+m$.  That is,
for disjoint finite sets $A$ and $B$, $|A\cup B|=|A|+|B|$.

What about $A\cup B$, where $A$ and $B$ are not disjoint?  We have to be
careful not to count the elements of $A\cap B$ twice.  After counting
the elements of $A$, there are only $|B|-|A\cap B|$ new elements in $B$ that
still need to be counted.  So we see that for any two finite sets $A$ and $B$,
$|A\cup B|=|A|+|B|-|A\cap B|$.

What about the number of subsets of a finite set $A$?  What is the relationship
between $|A|$ and $|\POW(A)|$?  The answer is provided by the following theorem.

\begin{theorem}\label{T-subsetct}
A finite set with cardinality $n$  has $2^n$ subsets.
\end{theorem}
\begin{proof}
Let $P(n)$ be the statement ``Any set with cardinality $n$ has $2^n$
subsets.''  We will use induction to show that $P(n)$ is true
for all $n\in N$.

Base case:  For $n=0$, $P(n)$ is the statement that a set with
cardinality 0 has $2^0$ subsets.  The only set with 0 elements is
the empty set.  The empty set has exactly 1 subset, namely itself.
Since $2^0=1$, $P(0)$ is true.

Inductive case:  Let $k$ be an arbitrary element of $\N$, and
assume that $P(k)$ is true.  That is, assume that any set with
cardinality $k$ has $2^k$ elements.  (This is the induction
hypothesis.)  We must show that $P(k+1)$
follows from this assumption.  That is, using the assumption
that any set with cardinality $k$ has $2^k$ subsets, we must show that any
set with cardinality $k+1$ has $2^{k+1}$ subsets.

Let $A$ be an arbitrary set with cardinality $k+1$. 
We must show that $|\POW(A)|=2^{k+1}$.   Since $|A|>0$,
$A$ contains at least one element.  Let $x$ be
some element of $A$, and let $B=A\SETDIFF\{x\}$.  The cardinality of
$B$ is $k$, so we have by the induction hypothesis that $|\POW(B)|=2^k$.
Now, we can divide the subsets of $A$ into two classes: subsets of
$A$ that do not contain $x$ and subsets of $A$ that do contain $x$.
Let $Y$ be the collection of subsets of $A$ that do not contain $x$,
and let $X$ be the collection of subsets of $A$ that do contain $x$.
$X$ and $Y$ are disjoint, since it is impossible for a given subset
of $A$ both to contain and to not contain~$x$. It follows that
$|\POW(A)|=|X\cup Y|=|X|+|Y|$.

Now, a member of $Y$ is a subset of $A$ that does not contain $x$.
But that is exactly the same as saying that a member of $Y$ is
a subset of $B$. So $Y=\POW(B)$, which we know contains $2^k$ members.
As for $X$, there is a one-to-one correspondence between $\POW(B)$
and $X$.  Namely, the function $f\colon\POW(B)\to X$ defined by
$f(C)=C\cup \{x\}$ is a bijective function.  (The proof of this
is left as an exercise.)  From this, it follows that
$|X|=|\POW(B)|=2^k$.  Putting these facts together, we see that
$|\POW(A)|=|X|+|Y|=2^k+2^k=2\cdot2^k=2^{k+1}$.  This completes the
proof that $P(k)\IMP P(k+1)$.
\end{proof}

We have seen that
the notation $A^B$ represents the set of all functions from $B$ to~$A$.
Suppose $A$ and $B$ are finite, and that $|A|=n$ and $|B|=m$.
Then $\left|A^B\right|=n^m=|A|^{|B|}$.  (This fact is one of the
reasons why the notation $A^B$ is reasonable.)
One way to see this is to note that there is a one-to-one correspondence
between $A^B$ and a cross product $A\times A\times\cdots A$, where the
number of terms in the cross product is $m$.  (This will be shown in
one of the exercises at the end of this section.)  It follows that
$\left|A^B\right|=|A|\cdot|A|\cdots|A|=n\cdot n\cdots n$, where the 
factor $n$ occurs $m$ times in the product.  This product is,
by definition,~$n^m$.   
  

This discussion about computing cardinalities is summarized in the following
theorem:
\begin{theorem}
Let $A$ and $B$ be finite sets.  Then
\begin{itemize}
   \item $|A\times B|=|A|\cdot |B|$.
   
   \item $|A\cup B|= |A|+|B|-|A\cap B|$.
   
   \item If $A$ and $B$ are disjoint then $|A\cup B|= |A|+|B|$.
   
   \item $\left|A^B\right|=|A|^{|B|}$.
   
   \item $|\POW(A)|=2^{|A|}$.
\end{itemize}
\end{theorem}
When it comes to counting and computing cardinalities, this theorem is
only the beginning of the story.  There is an entire large and deep branch of
mathematics known as \nw{combinatorics} that is devoted mostly to the problem
of counting.  But the theorem is already enough to answer many questions
about cardinalities.

For example, suppose that $|A|=n$ and $|B|=m$.  We can form the set
$\POW(A\times B)$, which consists of all subsets of $A\times B$.  Using
the theorem, we can compute that $|\POW(A\times B)|=2^{|A\times B|}=2^{|A|\cdot|B|}=2^{nm}$.
If we assume that $A$ and $B$ are disjoint, then we can compute that
$\left|A^{A\cup B}\right|=|A|^{|A\cup B|}=n^{n+m}$.

To be more concrete, let $X=\{a,b,c,d,e\}$ and let
$Y=\{c,d,e,f\}$ where $a$, $b$, $c$, $d$, $e$, and $f$ are distinct.
Then $|X\times Y|=5\cdot 4=20$ while $|X\cup Y|=5 + 4 - |\{c,d,e\}| = 6$
and $\left|X^Y\right|=5^4=625$.

We can also answer some simple practical questions.  Suppose that in
a restaurant you can choose one appetizer and one main course.  What is the
number of possible meals?  If $A$ is the set of possible appetizers and $C$ is the
set of possible main courses, then your meal is an ordered pair belonging
to  the set $A\times C$.  The number of possible meals is $|A\times C|$,
which is the product of the number of appetizers and the number of main courses.

Or suppose that four different prizes are to be awarded, and that the set of people
who are eligible for the prizes is~$A$.  Suppose that $|A|=n$.
How many different ways are there
to award the prizes?  One way to answer this question is to view a way of
awarding the prizes as a function from the set of prizes to the set of people.
Then, if $P$ is the set of prizes, the number of different ways of awarding
the prizes is $\left|A^P\right|$.  Since $|P|=4$ and $|A|=n$, this is $n^4$.  Another
way to look at it is to note that the people who win the prizes form
an ordered tuple $(a,b,c,d)$, which is an element of $A\times A\times A\times A$.
So the number of different ways of awarding the prizes is
$|A\times A\times A\times A|$, which is $|A|\cdot |A|\cdot |A|\cdot |A|$.  This
is $|A|^4$, or $n^4$, the same answer we got before.\footnote{This discussion assumes that one person can 
receive any number of prizes.
What if the prizes have to go to four different people?   This question
takes us a little farther into combinatorics than I would like to go,
but the answer is not hard.  The first award can be given to any of $n$
people.  The second prize goes to one of the remaining $n-1$ people.  There
are $n-2$ choices for the third prize and $n-3$ for the fourth.  The
number of different ways of awarding the prizes to four different people
is the product $n(n-1)(n-2)(n-3)$.}

\medbreak

So far, we have only discussed finite sets.  $\N$, the set of natural
numbers $\{0,1,2,3,\dots\}$, is an example of an infinite set.  There
is no one-to-one correspondence between $\N$ and any of the finite sets~$N_n$.  
Another example of an infinite set is the set of even natural numbers,
$E=\{0,2,4,6,8,\dots\}$.  There is a natural sense in which the sets $\N$
and $E$ have the same number of elements.  That is, there is a one-to-one
correspondence between them.  The function $f\colon \N\to E$ defined
by $f(n)=2n$ is bijective.  We will say that $\N$ and $E$ have the same
cardinality,\index{cardinality} even though that cardinality is not
a finite number.  Note that $E$ is a proper subset of $\N$.  That is,
$\N$ has a proper subset that has the same cardinality as~$\N$.

We will see that not all infinite sets have the same cardinality.  When it
comes to infinite sets, intuition is not always a good guide.  Most people
seem to be torn between two conflicting ideas.  On the one hand, they think, it seems
that a proper subset of a set should have fewer elements than the set itself.
On the other hand, it seems that any two infinite sets should have the same
number of elements.  Neither of these is true, at least if we define
having the same number of elements in terms of one-to-one correspondence.

A set $A$ is said to be \nw{countably infinite} if there is a one-to-one
correspondence between $\N$ and $A$.  A set is said to be
\nw[countable set]{countable} if it is either finite or countably infinite.
An infinite set that is not countably infinite is said to be
\nw[uncountable set]{uncountable}.  If $X$ is an uncountable set, then there
is no one-to-one correspondence between $\N$ and $X$.  

The idea of ``countable infinity''
is that even though a countably infinite set cannot be counted
in a finite time, we can imagine counting all the elements of $A$,
one-by-one, in an infinite process.  A bijective function
$f\colon\N\to A$ provides such an infinite listing:
$(f(0),f(1),f(2),f(3),\dots)$.  Since $f$ is onto, this infinite list
includes all the elements of $A$.  In fact, making such a list effectively
shows that $A$ is countably infinite, since the list amounts to a bijective
function from $\N$ to $A$.  For an uncountable set, it is
impossible to make a list, even an infinite list, that contains all the
elements of the set.

Before you start believing in uncountable sets, you should ask for
an example.  In Chapter~\ref{C-proof}, we worked with the infinite
sets $\Z$ (the integers), $\Q$ (the rationals), $\R$ (the reals), and
$\R\SETDIFF\Q$ (the irrationals).  Intuitively, these are all ``bigger''
than $\N$, but as we have already mentioned, intuition is a poor guide
when it comes to infinite sets.  Are any of $\Z$, $\Q$, $\R$, and $\R\SETDIFF\Q$
in fact uncountable?

It turns out that both $\Z$ and $\Q$ are only countably infinite.
The proof that $\Z$ is countable is left as an exercise; we 
will show here that the set of non-negative
rational numbers is countable.  (The fact that $\Q$ itself is countable 
follows easily from this.)  The reason is that it's possible to make
an infinite list containing all the non-negative rational numbers.  Start the
list with all the non-negative rational numbers $n/m$ such that $n+m=1$.  There
is only one such number, namely $0/1$.  Next come numbers with
$n+m=2$.  They are $0/2$ and $1/1$, but we leave out $0/2$ since
it's just another way of writing $0/1$, which is already in the
list.  Now, we add the numbers with $n+m=3$, namely
$0/3$, $1/2$, and $2/1$.  Again, we leave out $0/3$, since it's
equal to a number already in the list.  Next come numbers
with $n+m=4$.  Leaving out $0/4$ and $2/2$ since they are already
in the list, we add $1/3$ and $3/1$ to the list.  We continue
in this way, adding numbers with $n+m=5$, then numbers with
$n+m=6$, and so on.  The list looks like
\[
    \left( {\,0\,\over 1}, {\,1\,\over 1}, {\,1\,\over 2}, {\,2\,\over 1},
            {\,1\,\over 3}, {\,3\,\over 1}, {\,1\,\over 4}, {\,2\,\over 3},
              {\,3\,\over 2}, {\,4\,\over 1}, {\,1\,\over 5},
               {\,5\,\over 1}, {\,1\,\over 6}, {\,2\,\over 5},\dots
    \right)
\]
This process can be continued indefinitely, and every non-negative rational
number will eventually show up in the list.  So we get a
complete, infinite list of non-negative rational numbers.  This shows that
the set of non-negative rational numbers is in fact countable.

On the other hand, $\R$ is uncountable.  It is not possible to
make an infinite list that contains every real number.  It is
not even possible to make a list that contains every real
number between zero and one.  Another way of saying this is that
every infinite list of real numbers between zero and one, no 
matter how it is constructed, leaves something out.  To see why
this is true, imagine such a list, displayed in an infinitely long
column.  Each row contains one number, which has an infinite
number of digits after the decimal point.  Since it is a number
between zero and one, the only digit before the decimal point is
zero.  For example, the list might look like this:
\begin{center}
\begin{tabular}{l}
    0.\textbf{9}0398937249879561297927654857945\dots\\
    0.1\textbf{2}349342094059875980239230834549\dots\\
    0.22\textbf{4}00043298436234709323279989579\dots\\
    0.500\textbf{0}0000000000000000000000000000\dots\\
    0.7774\textbf{3}449234234876990120909480009\dots\\
    0.77755\textbf{5}55588888889498888980000111\dots\\
    0.123456\textbf{7}8888888888888888800000000\dots\\
    0.3483544\textbf{0}009848712712123940320577\dots\\
    0.93473244\textbf{4}47900498340999990948900\dots\\
    \qquad $\vdots$
\end{tabular}
\end{center}
This is only (a small part of) one possible list.  How can we be certain
that \emph{every} such list leaves out some real number between
zero and one?  The trick is to look at the digits shown in bold face.
We can use these digits to build a number that is not in the
list.  Since the first number in the list has a 9 in the first
position after the decimal point, we know that this number
cannot equal any number of, for example, the form 0.4\dots.  Since the second
number has a 2 in the second position after the decimal point,
\emph{neither} of the first two numbers in the list is
equal to any number that begins with 0.44\dots.  Since the third
number has a 4 in the third position after the decimal point,
\emph{none} of the first three numbers in the list is equal to any
number that begins 0.445\dots.  We can continue to construct
a number in this way, and we end up with a number that is different
from every number in the list.  The $n^{th}$ digit of the number
we are building must differ from the $n^{th}$ digit of the $n^{th}$ number 
in the list.  These are the digits shown in bold face in the 
above list.  To be definite, I use a 5 when the corresponding
boldface number is 4, and otherwise I use a~4.  For the list
shown above, this gives a number that begins 0.44544445\dots.
The number constructed in this way is not in the given list,
so the list is incomplete.
The same construction clearly works for any list of real numbers
between zero and one.  No such list can be a complete listing
of the real numbers between zero and one, and so there can be
no complete listing of all real numbers.  We conclude that
the set $\R$ is uncountable.

The technique used in this argument is called \nw{diagonalization}.
It is named after the fact that the bold face digits in the above
list lie along a diagonal line.   This proof was discovered by
a mathematician named Georg Cantor, who caused quite a fuss in the
nineteenth century when he came up with the idea that there are
different kinds of infinity.  Since then, his notion of using
one-to-one correspondence to define the cardinalities of infinite sets
has been accepted.  Mathematicians now consider it almost intuitive
that $\N$, $\Z$, and $\Q$ have the same cardinality while $\R$
has a strictly larger cardinality.

\medbreak
\begin{theorem}
Suppose that $X$ is an uncountable set, and that $K$ is a countable
subset of $X$.  Then the set $X\SETDIFF K$ is uncountable.
\end{theorem}
\begin{proof}
Let $X$ be an uncountable set.  Let $K\SUB X$, and suppose that
$K$ is countable.  Let $L=X\SETDIFF K$.  We want to show that
$L$ is uncountable.  Suppose that $L$ is
countable.  We will show that this assumption leads to a contradiction.

Note that $X=K\cup(X\SETDIFF K)=K\cup L$.  You will show in Exercise~11
of this section that the union of two countable
sets is countable.  Since $X$ is the union of the countable
sets $K$ and $L$, it follows that $X$ is countable.  But this
contradicts the fact that $X$ is uncountable.  This contradiction
proves the theorem. 
\end{proof}

In the proof, both $q$ and $\NOT q$ are shown to follow from the
assumptions, where $q$ is the statement ``$X$ is countable.''  The 
statement $q$ is shown to follow from the assumption that $X\SETDIFF K$ 
is countable.  The statement $\NOT q$ is true by assumption.
Since $q$ and $\NOT q$ cannot both be true, at least one of the
assumptions must be false.  The only assumption that can be false
is the assumption that $X\SETDIFF K$ is countable.

This theorem, by the way, has the following easy corollary.
(A \nw{corollary} is a theorem that follows easily from another,
previously proved theorem.)

\begin{corrolary}
The set of irrational real numbers is uncountable.
\end{corrolary}
\begin{proof}
Let $I$ be the set of irrational real numbers.  By
definition, $I=\R\SETDIFF\Q$.  We have already shown that
$\R$ is uncountable and that $\Q$ is countable, so the result
follows immediately from the previous theorem.
\end{proof}

\medbreak

You might still think that $\R$ is as big as things get, that is,
that any infinite set is in one-to-one correspondence with $\R$ or
with some subset of~$\R$.  In fact, though, if $X$ is any set
then it's possible to find a set that has strictly larger
cardinality than $X$.  In fact, $\POW(X)$ is such a set.
A variation of the diagonalization technique can be used to
show that there is no one-to-one correspondence between 
$X$ and $\POW(X)$.  Note that this is obvious for finite
sets, since for a finite set $X$, $|\POW(X)| = 2^{|X|}$,
which is larger than $|X|$.  The point of the theorem is that
it is true even for infinite sets.

\begin{theorem}\label{T-cardinality}
Let $X$ be any set.  Then there is no one-to-one correspondence
between $X$ and $\POW(X)$.
\end{theorem}
\begin{proof}
Given an arbitrary function $f\colon X\to \POW(X)$, we can show that
$f$ is not onto.  Since a one-to-one correspondence is both
one-to-one and onto, this shows that $f$ is not a one-to-one
correspondence.

Recall that $\POW(X)$ is the set of subsets of $X$.  So, for
each $x\in X$, $f(x)$ is a subset of $X$.  We have to show that
no matter how $f$ is defined, there is some subset of $X$ that
is not in the image of $f$.

Given $f$, we define $A$ to be the set $A=\{x\in X\st x\not\in f(x)\}$.
The test ``$x\not\in f(x)$'' makes sense because $f(x)$ is a set.
Since $A\SUB X$, we have that $A\in\POW(X)$.  However, $A$ is not in the image
of $f$.  That is, for every $y\in X$, $A\not=f(y)$.\footnote{In fact, we have
constructed $A$ so that the sets $A$ and $f(y)$ differ in at least
one element, namely $y$ itself.  
This is where the ``diagonalization'' comes in.}
To see why this
is true, let $y$ be any element of $X$.  There are two cases
to consider.  Either $y\in f(y)$ or $y\not\in f(y)$.  We show that
whichever case holds, $A\not=f(y)$.  If it is true that $y\in f(y)$,
then by the definition of $A$, $y\not\in A$.  Since $y\in f(y)$ but
$y\not\in A$, $f(y)$ and $A$ do not have the same elements and 
therefore are not equal.  On the other hand, suppose that
$y\not\in f(y)$.  Again, by the definition of $A$, this implies that
$y\in A$.  Since $y\not\in f(y)$ but $y\in A$, $f(y)$ and $A$ do not 
have the same elements and therefore are not equal.  In either
case, $A\not=f(y)$.  Since this is true for any $y\in X$,
we conclude that $A$ is not in the image of $f$ and therefore
$f$ is not a one-to-one correspondence.
\end{proof}


From this theorem, it follows that there is no one-to-one
correspondence between $\R$ and $\POW(\R)$.  The cardinality of $\POW(\R)$
is strictly bigger than the cardinality of~$\R$. But it doesn't 
stop there.  $\POW(\POW(\R))$ has an even bigger cardinality,
and the cardinality of $\POW(\POW(\POW(\R)))$ is bigger still.
We could go on like this forever, and we still won't have exhausted
all the possible cardinalities.  If we let ${\mathbb X}$ be the infinite
union $\R\cup\POW(\R)\cup\POW(\POW(\R))\cup\cdots$, then ${\mathbb X}$
has larger cardinality than any of the sets in the union.
And then there's $\POW({\mathbb X})$, $\POW(\POW({\mathbb X}))$, ${\mathbb X}\cup\POW({\mathbb X})\cup
\POW(\POW({\mathbb X}))\cup\cdots$.  There is no end to this.
There is no upper limit on possible cardinalities, not even an infinite
one!  We have counted past infinity.

We have seen that $|\R|$ is strictly larger than $|\N|$.
We end this section with what might look like
a simple question:  Is there a subset of $\R$ that is neither in
one-to-one correspondence with $\N$ nor with $\R$?  That is, is
the cardinality of $\R$ the \emph{next} largest cardinality after
the cardinality of $\N$, or are there other cardinalities intermediate
between them?  This problem was unsolved for quite a while, and
the solution, when it was found, proved to be completely unexpected.  
It was shown that both ``yes'' and
``no'' are consistent answers to this question!  That is, the
logical structure built on the system of axioms that had been
accepted as the basis of set theory was not extensive enough to
answer the question.  It is possible to extend the system in
various ways.  In some extensions, the answer is yes.  In others,
the answer is no.  You might object, ``Yes, but which answer is
true for the \emph{real} real numbers?''  Unfortunately, it's 
not even clear whether this question makes sense, since
in the world of mathematics, the real numbers are just part of
a structure built from a system of axioms.  And it's not at
all clear whether the ``real numbers'' exist in some sense in the real world.
If all this sounds like it's a bit of a philosophical muddle,
it is.  That's the state of things today at the foundation of
mathematics.

\begin{exercises}

\problem Suppose that $A$, $B$, and $C$ are finite sets which are
pairwise disjoint.  (That is, $A\cap B=A\cap C=B\cap C=\emptyset$.)
Express the cardinality of each of the following sets in terms
of $|A|$, $|B|$, and $|C|$.  Which of your answers depend on
the fact that the sets are pairwise disjoint?
\pparts{
   \POW(A\cup B)&      A\times (B^C)&     \POW(A)\times\POW(C)\cr
   A^{B\times C}&      (A\times B)^C&     \POW(A^B)\cr
   (A\cup B)^C&        (A\cup B)\times A& A\times A\times B\times B\cr
}

\problem Suppose that $A$ and $B$ are finite sets which are not necessarily
disjoint.  What are all the possible values for $|A\cup B|\,$?

\problem Let's say that an ``identifier'' consists of one or two
characters.  The fist character is one of the twenty-six letters
(A, B, \dots, C). The second character, if there is one, is either a letter or
one of the ten digits (0, 1, \dots, 9).  How many different identifiers
are there?   Explain your answer in terms of unions and cross products.

\problem Suppose that there are five books that you might bring along to
read on your vacation.  In how many different ways can you decide which
books to bring, assuming that you want to bring at least one?  Why?

\problem Show that the cardinality of a finite set is well-defined.
That is, show that if $f$ is a bijective function from a set $A$ to
$N_n$, and if $g$ is a bijective function from $A$ to $N_m$, then
$n=m$.

\problem Finish the proof of Theorem~\ref{T-subsetct} by proving the
following statement:  Let $A$ be a non-empty set, and let $x\in A$.  
Let $B=A\SETDIFF\{x\}$.
Let $X=\{C\SUB A\st x\in C\}$.  Define $f\colon\POW(B)\to X$
by the formula $f(C)=C\cup\{x\}$.  Show that $f$ is a bijective
function.

\problem Use induction on the cardinality of $B$ to show that for
any finite sets $A$ and $B$, $\left|A^B\right|=|A|^{|B|}$.
(Hint:  For the case where $B\not=\emptyset$, choose $x\in B$,
and divide $A^B$ into classes according to the value of $f(x)$.)

\problem Let $A$ and $B$ be finite sets with $|A|=n$ and $|B|=m$.
List the elements of $B$ as $B=\{b_0,b_1,\dots,b_{m-1}\}$.
Define the function ${\mathscr F}\colon A^B\to A\times A\times\cdots\times A$,
where $A$ occurs $m$ times in the cross product, by
${\mathscr F}(f)=\big(f(b_0), f(b_1), \dots,~f(b_{m-1})\big)$.
Show that ${\mathscr F}$ is a one-to-one correspondence.

\problem Show that $\Z$, the set of integers, is countable by finding
a one-to-one correspondence between $\N$ and $\Z$.

\problem Show that the set $\N\times\N$ is countable.

\problem Complete the proof of Theorem 2.9 as follows:
\ppart Suppose that $A$ and $B$ are countably infinite sets.
Show that $A\cup B$ is countably infinite.
\ppart Suppose that $A$ and $B$ are countable sets.
Show that $A\cup B$ is countable.

\problem Prove that each of the following statements is true.
In each case, use a proof by contradiction.
\ppart Let $X$ be a countably infinite set, and let $N$ be a finite
subset of $X$.  Then $X\SETDIFF N$ is countably infinite.
\ppart Let $A$ be an infinite set, and let $X$ be a subset of $A$.
Then at least one of the sets $X$ and $A\SETDIFF X$ is infinite.
\ppart Every subset of a finite set is finite.

\problem Let $A$ and $B$ be sets and let $\perp$ be an entity that is \emph{not}
a member of $B$.  Show that there is a one-to-one correspondence between
the set of functions from $A$ to $B\cup\{\perp\}$ and the set of
partial functions from $A$ to~$B$.  (Partial functions were defined
in Section~\ref{S-sets-5}.  The symbol ``$\perp$'' is sometimes used in
theoretical computer science to represent the value ``undefined.'')

\end{exercises}


